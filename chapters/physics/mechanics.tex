\section{Mechanics}\label{sec:mechanics}

Here we are interested in the relationship between force, matter and motion, as seen from a Newtonian perspective, focusing on motion (\textbf{kinematics} or ``\textit{the geometry of motion}'') and forces (\textbf{dynamics}).
When referring to classical mechanics, we usually mean Newton's laws of motion, which are described briefly.

\subsection{First Law}\label{subsec:first-law}
%%%%%%%%%%%%%%%%%%%%%%%%%%%%%%%%%%%%%%%%%%%%%%%%%%%%%%%%%%%%%%%%%%%%%%%%%%%%%%%%%%%%%%%%%%%%%%%%%%%%%%%%%%%%%%%%%%%%%%%%

``\textit{A body remains at rest, or in motion at a constant speed in a straight line, except insofar as it is acted upon by a force.}''

This law expresses the principle of \textbf{\gls{inertia}}: the natural behavior of a body is to move in a straight line at constant speed.
A body's motion preserves the status quo, but external forces can disturb this.

\subsection{Second Law}\label{subsec:second-law}
%%%%%%%%%%%%%%%%%%%%%%%%%%%%%%%%%%%%%%%%%%%%%%%%%%%%%%%%%%%%%%%%%%%%%%%%%%%%%%%%%%%%%%%%%%%%%%%%%%%%%%%%%%%%%%%%%%%%%%%%

``\textit{The net force on a body is equal to the body's instantaneous acceleration multiplied by its instantaneous mass or, equivalently, the rate at which the body's momentum changes with time.}'' ($F = ma$)

This law is about motion, or as we call it nowadays \textbf{\gls{momentum}}.
It depends upon the amount of matter contained in a body, the speed at which that body is moving, and the direction in which it is moving.
In modern notation, the momentum of a body is the product of its mass and its velocity.
The forces acting on a body add as \textbf{\gls{vector}s}: Quantities with both magnitude (amount of motion) and direction (of motion).
So the total force on a body depends upon both the magnitudes and the directions of the individual forces.
When the net force on a body is equal to zero, then by Newton's second law, the body does not accelerate, and it is said to be in \textbf{mechanical equilibrium}.
Momentum is conserved in a closed system, meaning that the total momentum before an event --such as a collision-- is equal to the total momentum after the event, as long as no external forces are acting on the system.

\subsection{Third Law}\label{subsec:third-law}
%%%%%%%%%%%%%%%%%%%%%%%%%%%%%%%%%%%%%%%%%%%%%%%%%%%%%%%%%%%%%%%%%%%%%%%%%%%%%%%%%%%%%%%%%%%%%%%%%%%%%%%%%%%%%%%%%%%%%%%%

``\textit{To every action, there is always opposed an equal reaction; or, the mutual actions of two bodies upon each other are always equal, and directed to contrary parts.}''

This relates to the conservation of momentum.

\subsection{Additional Laws}\label{subsec:additional-laws}
%%%%%%%%%%%%%%%%%%%%%%%%%%%%%%%%%%%%%%%%%%%%%%%%%%%%%%%%%%%%%%%%%%%%%%%%%%%%%%%%%%%%%%%%%%%%%%%%%%%%%%%%%%%%%%%%%%%%%%%%

\noindent \textit{Uniformly accelerated motion}, also known as: \textbf{free fall}.
When a body falls (in the absence of air resistance), it will accelerate at a constant rate.
The speed with which it falls is proportional to the elapsed time, and the acceleration is the same for all bodies, independent of their mass (``\textit{law of universal gravitation}'').

\vspace{5pt}
\noindent \textit{Uniform circular motion}, which contains the \textbf{centripetal} force and is required to sustain the acceleration towards the center.
The \textbf{centrifugal} force is, on the other hand, an inertial (fictitious/pseudo) force that is directed radially away from the axis of rotation.

\vspace{5pt}
\noindent \textit{Harmonic motion}, as shown by a spring-mass system or a pendulum.
