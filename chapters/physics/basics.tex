\section{Basics}\label{sec:basics}

\subsection{Energy}\label{subsec:energy}
%%%%%%%%%%%%%%%%%%%%%%%%%%%%%%%%%%%%%%%%%%%%%%%%%%%%%%%%%%%%%%%%%%%%%%%%%%%%%%%%%%%%%%%%%%%%%%%%%%%%%%%%%%%%%%%%%%%%%%%%

The word ``energy'' is just too often misused in the spiritual world as some metaphysical, psycho-telepathic \textbf{mystery}.
In physics, we define it simply as ``\textit{the capacity for doing work}'', and as such different forms exist: potential (position), kinetic (movement), thermal (heat), nuclear (atom), electrical (charges), chemical, and so on.
All forms of energy are associated with motion: Any given body has kinetic energy if it's in motion.
A tensioned device such as a bow, a spring, or your tendons, though at rest, have the potential for creating motion; it contains potential energy.
Energy can neither be created nor destroyed, but only transformed from one form to another, which is stated in the first law of thermodynamics: energy conservation.

% TODO vs force

\subsection{Momentum versus Inertia}\label{subsec:momentum-versus-inertia}
%%%%%%%%%%%%%%%%%%%%%%%%%%%%%%%%%%%%%%%%%%%%%%%%%%%%%%%%%%%%%%%%%%%%%%%%%%%%%%%%%%%%%%%%%%%%%%%%%%%%%%%%%%%%%%%%%%%%%%%%

Momentum and inertia are related concepts, yet they refer to different properties of objects in motion.

\textbf{Momentum} measures an object's quantity of motion, which is calculated based on an object's mass and its velocity.
The result is a vector quantity.

\textbf{Inertia}, on the other hand, is about an object's tendency to resist changes while being in motion.
The more massive an object is, the greater its inertia (``laziness'') and the more force is required to change its motion, to start or stop its moving, or change its direction.

In summary, momentum quantifies the amount of motion of an object, considering both its mass and velocity, while inertia describes an object's resistance to changes in its state of motion, primarily due to its mass.

% TODO (shared) center of mass/gravity
% TODO angular momentum
