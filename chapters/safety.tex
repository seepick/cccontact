\chapter{Safety}\label{ch:safety}

\begin{wrapfigure}{R}{0.3\textwidth}
    \centering
    \includegraphics[width=0.25\textwidth]{images/safety}
\end{wrapfigure}

We sometimes move in order to keep our body healthy; yet it sometimes happen that while we practice, we injure our body, destroy our health.
It seems like a terrible contradiction, that the one thing which is supposed to heal, suddenly harms.
And that is why safety is of utmost importance when practicing CI -also because it sometimes, just like acrobatics, includes some more dangerous techniques.
Not only the safety of yourself, but we also want to ensure the safety of the other people we are dancing with, which lead us to safety rules there.
And lastly, not only physical, but also psychological safety needs to be addressed, whether in individual or the group's emotional and social safety.

\section{Is it safe?}\label{sec:is-it-safe?}
%%%%%%%%%%%%%%%%%%%%%%%%%%%%%%%%%%%%%%%%%%%%%%%%%%%%%%%%%%%%%%%%%%%%%%%%%%%%%%%%%%%%%%%%%%%%%%%%%%%%%%%%%%%%%%%%%%%%%%%%

Safety in obvious terms of ``free of injuries'' should always come first when practicing any potentially risky movement style.
Furthermore, respecting and adhering the local, specific sub-\textbf{cultural norms} which will also lead to emotional/psychological safety within the group.
Besides all these concerns about safety, it is safe to say that CI can be practiced by anyone: professionally trained dancers, recreational movers, athletes, dancers of all abilities and ages.

Yet, if you ask yourself whether CI is safe, then the short answer is very simple: No.
It can be ``safe enough'' though, which is what we aim for.
It is for sure less safe than Judo or MMA, but more safe than poetry or running.
It is a high intense for of dancing, and also more risky due to the unknown of your partner what they are bringing.
Safety just can't be reached in the outside world, meeting each other;
neither injured (physically) nor psychologically in pain.
So that's why it's important to be aware of the existing risks, yet expanding the boundaries of what's possible.

\section{Consent}\label{sec:consent}
%%%%%%%%%%%%%%%%%%%%%%%%%%%%%%%%%%%%%%%%%%%%%%%%%%%%%%%%%%%%%%%%%%%%%%%%%%%%%%%%%%%%%%%%%%%%%%%%%%%%%%%%%%%%%%%%%%%%%%%%

Consent is mainly about being able to express as well as hearing (and appropriately responding to) a ``yes'' and a ``no'', which are both possibly (and preferably) expressed non-verbally.

Of course this starts already when engaging in a dance with someone, but primarily is important when it comes to more advanced lifts.
It could be due to lack of trust into each other, fear of being lifted, considering the other person's weight as too heavy, or other reasons.
There are many elegant ways out, so that we can hold ourselves responsible for withstanding our boundaries instead of making others accountable for crossing them, ultimately ``self-dis-empowering'' ourselves.

Whenever a gentle hint by movement does not have the desired result, we can of course always cough, or lastly simply speak and be very explicit about our wishes.

\section{Technical Safety}\label{sec:technical-safety}
%%%%%%%%%%%%%%%%%%%%%%%%%%%%%%%%%%%%%%%%%%%%%%%%%%%%%%%%%%%%%%%%%%%%%%%%%%%%%%%%%%%%%%%%%%%%%%%%%%%%%%%%%%%%%%%%%%%%%%%%

There are many, many things which can lead to dramatic injuries, especially when being performed by non-advanced students.
Yet, for the sake of simplicity and conciseness, we will limit the list to a few, the most common ones, here:

\begin{itemize}
    \item \textbf{Head above ass}: When lifting a partner, the shoulder always by higher than the pelvis, as otherwise the person will slide down overhead.
    \item \textbf{Don't grab}: When encompassing the partner's limb for example, that part of his body will be immovable and thus prevents him from using it as ``landing gear'' when needed, leading to a potentially severe injury.
    Additionally, it removes the agency of the other person and is considered to be simply rude in the CI scene.
    There is a fine line between ``polite manipulation'' by pushing a bit, guiding, and even offering a welcomed structure/base for the other person, and a forceful, dominant directedness, which as well can be ok with some dance partners (you know well) and they consented to it.
    Yet, to be fully safe, especially with strangers or in unknown places, refrain from too much manipulation (or pedipulation, meaning using the feet to change the shape of your partner).
    And as always: Once you reach a higher level of experience and skill, many of those rules can be (intentionally) broken.
    \item \textbf{Don't interlock}: When people perform a back lift and interlock the arms (and God forbid simultaneously lowering the head below the pelvis line) and they fall, the flyer might get into a state of panic and tenses up his arms, while the base will fall and not being able to use his arms, thus falling on his face with the weight of the lifted partner.
    \item \textbf{Don't jump}, especially when being lifted.
    Jumping and other techniques leading to (uncontrolled) collisions was an essential part of oldschool CI, yet in today's development more emphasis is being put on safety.
    The impact of jumping on a partner can lead to injuries, especially on the spine, that's why we prefer to pour our weight softly on our partner, so that he can adjust, shift his weight and readjust his structure.
    \item \textbf{Momentum} is debatable, yet if safety has the highest priority, then it should be avoided as it takes away reversibility because there is no control anymore.
\end{itemize}

For any more risky technique, it's always a good idea to have someone spotting you, someone we usually refer to as a ``\gls{guardianangel}'', who is trying to catch if someone falls onto the ground, usually more concerned about the head and less about the feet.
Don't grab, yet stay ready to engage in case something goes wrong and grab them well to make them land as softly as possible.

\section{Intimacy}\label{sec:intimacy}
%%%%%%%%%%%%%%%%%%%%%%%%%%%%%%%%%%%%%%%%%%%%%%%%%%%%%%%%%%%%%%%%%%%%%%%%%%%%%%%%%%%%%%%%%%%%%%%%%%%%%%%%%%%%%%%%%%%%%%%%

The psychological aspect of safety usually revolves around the topics of intimacy, sensuality and sexuality,
for example when using the entire body and rolling over it.
From the outside (imagine people who don't know the practice) this might seem very intimate, but for is experienced dancers its pure physics and exploration.

Not only on the physical level, the proximity in touch, but also the sensual/sexual level (which is NOT in its core), can be frightening.
Advice: Switch of the mind necessary to change that.

Sensuality is a wonderful thing, but it is not really appropriate to go into ``melting'' into your partner, falling in love, or even sexuality, and expressing this via caressing touch, cuddles and somewhat ``Tantric behavior'' during Contact Improvisation.
This boundary is often crossed as the positive bonding effects of touch easily leads to more.
At this moment it is advised to separate, to ``cool down'' and re-engage with the dance, instead of the partner.
Caressing the skin of another person, to ``cuddle up'' and all of it is great, amazing, and please do more of it, but please refrain from doing so on the CI dance floor.

If sensuality/sexuality still happens, the teacher will take that off very fast; or the jam facilitator, if present.
Cuddles, and cuddle puddles, will often happen and are being tolerated most of the time, yet they are not intentional!
If your intention is to engage in cuddles, the group will ``notice'' that behavior, and they might not want to dance with you anymore because of that.
There is of course no strict rule about it, no prohibition, and in some jams they totally tolerate it, even few invite it, but mostly it just doesn't happen.

Often there is a partner bodywork at the end; a typical thing dancers would do.
It is meant to be a more technical, medical, or sports massage, than a personal, emotional, intimate touch session.

In classes, it happens very, very, very rarely that certain individuals will look for satisfying their sexual needs by ``rubbing their horny body on someone else's''.
The teacher will take care of a safe space for participants, and stop any form of sexual intended or predatory behavior right away, declaring it as a clear ``no-go''.
That's also why people should go first to a class to learn the norms and values, instead dropping in a jam right away.
At jams without a supervisor no one knows how often such behavior might occur; maybe relatively more often?
Most of the time there is an organizer though, which can repeat clear rules and guidelines in an opening circle, minimizing the probability of such non-welcome behavior.
Comparing CI to Ecstatic Dance or Tantric Dance, in CI are for sure way way less ``horny people'' showing up.

\section{Society and Touch}\label{sec:society-and-touch}
%%%%%%%%%%%%%%%%%%%%%%%%%%%%%%%%%%%%%%%%%%%%%%%%%%%%%%%%%%%%%%%%%%%%%%%%%%%%%%%%%%%%%%%%%%%%%%%%%%%%%%%%%%%%%%%%%%%%%%%%

In a very strictly gender-role based society, e.g.: where no same-sex touch is permitted, CI would have big problems to become popular.
Our Western World is seen as a non-touch society, yet there are bubbles where it is allowed and even encouraged; e.g. couple dancing, although not the entire body but only holding or embracing arms and the back.
Because of the lack of (connection via) touch, and us being \textbf{social mammals}, these bubbles might get more and more attraction somewhere in the (near) future.
Especially people who don't have others close to them, like relatives, loved ones, partners, or in extreme circumstances like during COVID; we feel again that basic human need for touch.
But be aware that this is not the aim of CI, it's a very nice additional \textbf{side effect}.
People who only come for that side effect, won't stay long in CI\@.
There are plenty other places where you can get touch for that purpose; CI is not being one of them.

Using CI in order to ``heal the world'' seems a bit\ldots let's say utopian, or maybe even naive.
This approach is more well-fitted for younger people, with lots of enthusiasm, and still being overly eager.
That eagerness comes because CI can indeed change one's life.

``\textit{If everyone would do CI, there would be no wars anymore, and it would be the end of capitalism}'', says the young.
``\textit{CI is a very, very rich, beneficial and amazing dance form, and it might not change the world, but it did change me}'', says the old.
The more we \textbf{spread it}, the better.

\section{Respectful Behavior}\label{sec:respectful-behavior}
%%%%%%%%%%%%%%%%%%%%%%%%%%%%%%%%%%%%%%%%%%%%%%%%%%%%%%%%%%%%%%%%%%%%%%%%%%%%%%%%%%%%%%%%%%%%%%%%%%%%%%%%%%%%%%%%%%%%%%%%

This will be a short list based on observations and personal experiences which seem to be worth mentioning and of course there is no claim of completeness or anything similar.

\begin{itemize}
    \item \textbf{Starting slowly} helps us to get to know each other, before we rush into someone's living room.
    It's like a handshake, to get a glimpse of who the other person is, where he is in regard to skill and experience and then take it from there.
    \item \textbf{Eye gazing} is another wonderful tool to deeply connect with another human being, yet during Contact Improvisation we refrain from doing so for too long.
     Your partners will tell you that this is not the way.
     We use our peripheral vision to see, and usually don't constantly dance front-to-front.
     Again: We fall in love with the dance, if with anything, and not with the partner.
    \item Speaking about \textbf{front}: It is totally fine to roll over the front side of your partner, and there is no need to avoid certain body parts (breasts, genital area).
    Once we are able to depersonalize the partner, seeing it as a physical object and letting go of the stories, there is no need to avoid certain body parts, and rather develop a more innocent and reality bound perspective.
\end{itemize}

\section{Etiquette}\label{sec:etiquette}
%%%%%%%%%%%%%%%%%%%%%%%%%%%%%%%%%%%%%%%%%%%%%%%%%%%%%%%%%%%%%%%%%%%%%%%%%%%%%%%%%%%%%%%%%%%%%%%%%%%%%%%%%%%%%%%%%%%%%%%%

In every subculture, there are certain norms and rules established which are not written down anywhere.
They somehow float in the minds of the people, implicitly without any mentioning.
Yet if they are broken once, there will be consequences of the group as a punishment for misbehavior.
I personally find it useful to try to capture those and make them explicit, so we can avoid unintentional misbehavior and establish as much harmony as possible.
Just know, that every scene, maybe even every group around a specific teacher, has their own set of normative rules, thus this list is just a possible set of many.

\begin{itemize}
    \item \textbf{Talking} is considered to be a distraction during Contact Improvisation, especially during jams.
    If verbal communication is indeed needed, it can be so, there is no obligation for total silence, but be mindful whether it is truly necessary at the moment, or is it just some irrelevant chit-chat, and try to keep the volume low, keep it short in order to not distract the fellow dancers.
    Otherwise, you can always step off the dance floor and speak properly a bit further away if really needed.
    \item State your \textbf{name} when you dance with someone, at least at the very end, and ask for that person's name.
    It's just a nice way to acknowledge the other person and useful to furthermore possibly create a friendly bond.
    \item Don't \textbf{park} on the dance floor, meaning: Lying on the floor, with eyes closed, and being unaware of the surroundings and the other people.
    \item Although probably obvious, yet worth mentioning is \textbf{body hygiene}: Wear fresh clothes, make sure no intense bad breath, and have taken a shower before engaging in such an intimate movement form with one or more partners.
    Try to keep it at normal levels, or above to smell nice, yet no too strong perfume!
\end{itemize}

If you participate in a class, there will be more guidance, and it will be easier for you to spot what's ``right and wrong'' behavior; yet in a jam it might be a bit more difficult.
Take about 10 minutes at the beginning of a jam to sit, simply watch and observe the communal implicit norms.
Every jam is different, yet there is a common set of baseline rules as mentioned already above.

\section{Clothing}\label{sec:clothing}
%%%%%%%%%%%%%%%%%%%%%%%%%%%%%%%%%%%%%%%%%%%%%%%%%%%%%%%%%%%%%%%%%%%%%%%%%%%%%%%%%%%%%%%%%%%%%%%%%%%%%%%%%%%%%%%%%%%%%%%%

Appropriate \textbf{clothing} is found in every little group, sometimes for very rational, practical reasons, sometimes just in order to establish some kind of norm to agree to, a dress code.
If violated, it can even lead that people won't dance with you because of that, and not because of you as a person!

As such, it is recommended to wear long pants and sleeves to be able to slide on the ground and enough friction (and sweat soaking).
Clothes which don't limit movements like firm jeans.
Button free shirts as rolling on partners with your full weight might hurt them, and also the shirt might break, and also no zippers.
In short: Normal, dancing, movable sports clothes.

Obviously jewelry (like big earrings) are not advised as they can get trapped on the other person's body/clothes hurting someone.

In CI we need some friction for grip, and for that matter plastic fabrics will slide more (too much), thus cotton is better.
Your beloved Adidas pants are not preferred as they are too slippery;
sorry.
Speaking of grip: It's better to dance barefoot, without socks to avoid slipping (or socks with those plastic nobs on the sole).

Ultimately, we don't dress to impress but rather prefer to wear our cosy pyjamas.