\section{Safety}\label{sec:safety}

\begin{wrapfigure}{R}{0.3\textwidth}
    \centering
    \includegraphics[width=0.25\textwidth]{images/safety}
\end{wrapfigure}

Safety in obvious terms of ``free of injuries'' should always come first when practicing any potentially risky movement style.
Furthermore respecting and adhering the local, specific sub-\textbf{cultural norms} which will also lead to emotional/psychological safety within the group.

\subsection{Consent}\label{subsec:consent}

Consent is mainly about being able to express as well as hearing (and appropriately responding to) a ``yes'' and a ``no'', which are both possibly (and preferably) expressed non-verbally.

Of course this starts already when engaging in a dance with someone, but primarily is important when it comes to more advanced lifts.
It could be due to lack of trust into each other, fear of being lifted, considering the other person's weight as too heavy, or other reasons.
There are many elegant ways out, so that we can hold ourselves responsible for withstanding our boundaries instead of making others accountable for crossing them, ultimately ``self-dis-empowering'' ourselves.

Whenever a gentle hint by movement does not have the desired result, we can of course always cough, or lastly simply speak and be very explicit about our wishes.

\subsection{Technical Safety}\label{subsec:technical-safety}

There are many, many things which can lead to dramatic injuries, especially when being performed by non-advanced students.
Yet, for the sake of simplicity and conciseness, we will limit the list to a few, the most common ones, here:

\begin{itemize}
    \item \textbf{Head above ass}: When lifting a partner, the shoulder always by higher than the pelvis, as otherwise the person will slide down overhead.
    \item \textbf{Don't grab}: When encompassing the a partner's limb for example, that part of his body will be immovable and thus prevents him from using it as ``landing gear'' when needed, leading to a potentially severe injury.
    Additionally, it removes the agency of the other person and is considered to be simply rude in the CI scene.
    \item \textbf{Don't interlock}: When people perform a back lift and interlock the arms (and God forbid simultaneously lowering the head below the pelvis line) and they fall, the flyer might get into a state of panic and tenses up his arms, while the base will fall and not being able to use his arms, thus falling on his face with the weight of the lifted partner.
\end{itemize}

\subsection{Respectful Behavior}\label{subsec:respectful-behavior}

% TODO finish those

% sensuality; nice, but  not here (cuddles,arousing touch, tantric)
% eye gazing
% front side rolling: depersonalize partner, seeing body as physical object, let go of the stories; fall in love with dance not the other dancer.

\subsection{Etiquette}\label{subsec:etiquette}

% say your name, ask for name
% don't talk too much
% proper clothing (no jeans, buttons, ... we were pyjamas haha)
% don't smell bad (body and breath ;)


