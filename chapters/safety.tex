\section{Safety}\label{sec:safety}

\begin{wrapfigure}{R}{0.3\textwidth}
    \centering
    \includegraphics[width=0.25\textwidth]{images/safety}
\end{wrapfigure}

Safety in obvious terms of ``free of injuries'' should always come first when practicing any potentially risky movement style.
Furthermore, respecting and adhering the local, specific sub-\textbf{cultural norms} which will also lead to emotional/psychological safety within the group.
Besides all these concerns about safety, it is safe to say that CI can be practiced by anyone: professionally trained dancers, recreational movers, athletes, dancers of all abilities and ages.

\subsection{Consent}\label{subsec:consent}

Consent is mainly about being able to express as well as hearing (and appropriately responding to) a ``yes'' and a ``no'', which are both possibly (and preferably) expressed non-verbally.

Of course this starts already when engaging in a dance with someone, but primarily is important when it comes to more advanced lifts.
It could be due to lack of trust into each other, fear of being lifted, considering the other person's weight as too heavy, or other reasons.
There are many elegant ways out, so that we can hold ourselves responsible for withstanding our boundaries instead of making others accountable for crossing them, ultimately ``self-dis-empowering'' ourselves.

Whenever a gentle hint by movement does not have the desired result, we can of course always cough, or lastly simply speak and be very explicit about our wishes.

\subsection{Technical Safety}\label{subsec:technical-safety}

There are many, many things which can lead to dramatic injuries, especially when being performed by non-advanced students.
Yet, for the sake of simplicity and conciseness, we will limit the list to a few, the most common ones, here:

\begin{itemize}
    \item \textbf{Head above ass}: When lifting a partner, the shoulder always by higher than the pelvis, as otherwise the person will slide down overhead.
    \item \textbf{Don't grab}: When encompassing the a partner's limb for example, that part of his body will be immovable and thus prevents him from using it as ``landing gear'' when needed, leading to a potentially severe injury.
    Additionally, it removes the agency of the other person and is considered to be simply rude in the CI scene.
    \item \textbf{Don't interlock}: When people perform a back lift and interlock the arms (and God forbid simultaneously lowering the head below the pelvis line) and they fall, the flyer might get into a state of panic and tenses up his arms, while the base will fall and not being able to use his arms, thus falling on his face with the weight of the lifted partner.
\end{itemize}

\subsection{Respectful Behavior}\label{subsec:respectful-behavior}

This will be a short list based on observations and personal experiences which seem to be worth mentioning and of course there is no claim of completeness or anything similar.

\begin{itemize}
    \item \textbf{Sensuality} is a wonderful thing, but it is not really appropriate to go into ``melting'' into your partner, falling in love, or even sexuality, and expressing this via caressing touch, cuddles and somewhat ``Tantric behavior'' during contact improvisation.
    This boundary is often crossed as the positive bonding effects of touch easily leads to more.
    At this moment it is advised to separate, to ``cool down'' and re-engage with the dance, instead of the partner.
    Caressing the skin of another person, to ``cuddle up'' and all of it is great, amazing, and please do more of it, but please refrain from doing so on the CI dancefloor.
    \item \textbf{Eye gazing} is another wonderful tool to deeply connect with another human being, yet during contact improvisation we refrain from doing so.
    We use our peripheral vision to see, and usually don't constantly dance front-to-front.
    Again: We fall in love with the dance, if with anything, and not with the partner.
    \item Speaking about \textbf{front}: It is totally fine to roll over the front side of your partner, and there is no need to avoid certain body parts (breasts, genital area).
    Once we are able to depersonalize the partner, seeing it as a physical object and letting go of the stories, there is no need to avoid certain body parts, and rather develop a more innocent and reality bound perspective.
\end{itemize}

\subsection{Etiquette}\label{subsec:etiquette}

In every subculture, there are certain norms and rules established which are not written down anywhere.
They somehow float in the minds of the people, implicitly without any mentioning.
Yet if they are broken once, there will be consequences of the group as a punishment for misbehavior.
I personally find it useful to try to capture those and make them explicit, so we can avoid unintentional misbehavior and establish as much harmony as possible.
Just know, that every scene, maybe even every group around a specific teacher, has their own set of normative rules, thus this list is just a possible set of many.

\begin{itemize}
    \item \textbf{Talking} is considered to be a distraction during contact improvisation, especially during jams.
    If loud, verbal communication is indeed needed, it can be so, there is no obligation for total silence, but be mindful whether it is truly necessary at the moment, or is it just some irrelevant chit-chat, and try to keep the volume low in order to not distract the fellow dancers.
    \item State your \textbf{name} when you dance with someone, at least at the very end, and ask for that person's name.
    It's just a nice way to acknowledge the other person and useful to furthermore possibly create a friendly bond.
    \item Proper, and appropriate, \textbf{clothing} is found in every little group, sometimes even for very rational, practical reasons.
    As such, it is recommended to wear long pants and sleeves to be able to slide on the ground; clothes which don't limit movements like jeans; button free shirts as rolling on partners with your full weight might hurt them, and also the shirt might break.
    Ultimately, we don't dress to impress but rather prefer to wear our cosy pyjamas.
    \item Although probably obvious, yet worth mentioning is \textbf{body hygiene}: Wear fresh clothes, make sure no intense bad breath, and have taken a shower before engaging in such an intimate movement form with one or more partners.
\end{itemize}
