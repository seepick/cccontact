\section{Rolling Point of Contact}\label{sec:rolling-point-of-contact}

We use spiraling and rotating movement patterns to always \textbf{maintain contact} and amount of pressure in a rolling fashion, instead of sliding or even jumping the point of contact, e.g. directly from a hand to a shoulder, not passing ``through'' the whole arm.
Sliding or jumping (point of contact) is by no means wrong, but it is added later on deliberately by more advanced practitioners.

The contact will thereby follow a \textbf{predictable trajectory}, a pathway, which means both partners can anticipate the very next movement, which furthermore leads to a more ``fluid sensation'' in the dance.
There is no disengaging or re-engaging of the point of contact (at least not at the beginning), which sometimes can feel like little bumps during the dance, breaking this fluid sensation.
For this to happen, it is required to have a more agile body, bulging out body parts and bending/flexing wherever necessary to keep a clear rolling point of contact; something we refer to as an ``octopus''-quality.
It is also used to correct each other to find balance, to readjust and realign.

This is the second most important and also the second most often principle with which beginners struggle with.
Be aware that we don't aim for \textbf{intimate body-parts} (buttocks, breasts, genitals), yet we roll over them ``coincidentally'' without staying there.
For example, when my head would be at those parts, I will still try to avoid that if possible, even if we know each other very well.
We try to desexualize the human body --not the partner, but the body--, something which can be indeed difficult in our hyper-sexualized society.
The touch should be without any sexual intention, even if intimate parts are being rolled over, which will make us both feel comfortable; the ``how'' is more important than the ``what''.
We play with each other, just like kids are playing ``rough and tumble'' in the playground, before encountering the wonderful world of sexuality.
Just like a doctor touching breast tissue intending to find signs of cancer and without any sexual intention or gain some personal advantage; that's why it feels safe for the patient to receive that touch.
