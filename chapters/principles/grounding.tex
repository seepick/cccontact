\section{Grounding}\label{sec:grounding}

With grounding we are referring to some kind of sensation of (light) heaviness in the body, which makes the stance more \textbf{stable}, more robust and thus more connected to the ground.
Imaginary language like ``rooting'', and similar, are often used to describe this internal sensation, with its very realistic impact on the external.
This quality is the beginning of it all, without it, we can't go any further, as without a firm foundation, there is no house we can build upon.
To help to improve our groundedness we can use \textbf{visualizations} (roots growing into the ground), focusing our attention to where the sole of the feet have contact with the floor, breathing out and relaxing the muscular tension without collapsing in one's structure, and simply thinking about words which are associated with a grounded, firm, or stable quality.

It should not be confused with stiffness or rigidity, which so often lead to the illusion of groundedness and is achieved by simply contracting all muscles; something we don't want to do as it will remove the ability to adapt to the moment, remove our flexibility.

Lastly, because of the interconnection between \textbf{body-mind}, the fact that you become a more grounded mover physically, you also become a more grounded person mentally (emotionally stable, more resilient, durable).
