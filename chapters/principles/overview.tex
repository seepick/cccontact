\section{Overview}\label{sec:overview}

\subsection{Grammar versus Vocabulary}\label{subsec:grammar-versus-vocabulary}
%%%%%%%%%%%%%%%%%%%%%%%%%%%%%%%%%%%%%%%%%%%%%%%%%%%%%%%%%%%%%%%%%%%%%%%%%%%%%%%%%%%%%%%%%%%%%%%%%%%%%%%%%%%%%%%%%%%%%%%%

Many different systems --like with sports, dances or martial arts-- are focused around dozens or even hundreds of \textbf{techniques}, or call them words/vocabulary.
They are given to the student to learn by hard, including their names, and a precise definitions of what's the right way of doing it.
This is an approach that might work for many instances, but obviously has some serious disadvantages.
Think of when it comes to quickly responding, picking the right technique from many within a split of a second.
And more importantly is the ability for individual expression: using the system as an art-form to ``say the unsayable''.

In contrast to that, CI --along with many other systems as well-- is centered around a few principles, or call them the grammar.
Every technique that might be taught, studied and practiced, is a manifestation of those core principles.
There are no real moves to be learned, but more principles to be embodied and applied in any given moment.
They need to be adjusted to whatever needs are present while staying within the boundaries of the system.
Once they are well understood, you can free yourself from the limitations of specific techniques, and questions like whether something is ``right or wrong'' can be easily answered by asking your deeper understanding.
Yet, as it is with the mastery of any art: Once the principles are fully understood, they can be broken if desired so, as: ``\textit{You can do whatever you want, as long as you know what you are doing.}''
% TODO move this last sentence into the motto chapter

When you meet CI people from another culture, they might speak a different dialect, even using foreign words unknown to you.
But as long as you both use the same grammar, you will still understand and be able to interact with each other.

\subsection{Technique or Principle}\label{subsec:technique-or-principle} % TODO this subsection is duplicated... merge it with the one above

CI is a \textbf{principle-based system}, and as such it doesn't have a defined set of techniques which are universally taught and practiced.
Nevertheless, there are certain reoccurring movements (you could call them ``tricks'') which could be considered as techniques.
They should not be regarded as something to be followed too literally.
As long as you stay within the framework of the principles, any adoption can be judged as correct relative to the system.

The difference between technique and principle of CI is like the difference between \textbf{grammar} (universal and abstract) and \textbf{vocabulary} (specific and concrete) of language.
Even in the United States we will use the same grammar, but the words/vocabulary, and especially style/dialect will be slightly different from in the United Kingdom, Australia or anywhere else in the world; expect small hiccups to happen.
Nevertheless, we will be able to \textbf{understand} each other as the principles, the grammar stays the same.
And often it will be necessary to stay a few hours with a new teacher to embody his body language, from which the vocabulary comes.
Vocabulary, like lifts, is a rather \textbf{regional expression} of the application of the commonly shared principles.


\subsection{In Short}\label{subsec:in-short}
%%%%%%%%%%%%%%%%%%%%%%%%%%%%%%%%%%%%%%%%%%%%%%%%%%%%%%%%%%%%%%%%%%%%%%%%%%%%%%%%%%%%%%%%%%%%%%%%%%%%%%%%%%%%%%%%%%%%%%%%

As CI is an improvised partner dance --usually, but not necessarily done with two people/bodies--, it encourages the exploration together with the ground, while staying in constant physical contact.
The dance is supposed to move by itself, according to the participants aims and wishes.

In short, these are the basic principles used in CI, whereas the first two could be considered the physical essential ones and the other more about attitude and technique:

\begin{itemize}
    \setlength\itemsep{0em}
    \item \textbf{Sharing Weight}
    \item \textbf{Rolling Point of Contact}
    \item Exploration of \textbf{Physical Forces}
    \item \textbf{Spirals} - and other related movement patterns
\end{itemize}

Once all those principles are embodied, they will show up and surprise you when they happen and change your pathways.
They will also very much show up when in high velocity, when going into a risk engaged dance, dancing with a super high level of alertness and attention, jumping on each other, yet landing safely back on the ground.
