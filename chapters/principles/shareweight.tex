\section{Sharing Weight}\label{sec:sharing-weight}

The first and most important principle is trying to seek a deep --core-- connection between two bodies, something we call in our classes as an ``\textit{oomph}''-quality.
The body is \textbf{stable and grounded}, yet its limbs and joints are \textbf{soft and relaxed}; \textit{like an iron stick wrapped in cotton wool}.
Also, the contact is primarily on \gls{thebox}, the upper body, and less on the arms and legs.
It is different from actively pushing with muscular force, and also slightly different from leaning, by which you shift your center of gravity beyond a point of no-return.

The contact should lead to a sensation of the \textbf{ground underneath} your partner's feet, passing through their center, originating from the single point of contact which can be even as far from the ground as a hand.
We constantly try to search for the \gls{centergravity} of the other person's body, which might sound familiar to people with a background in martial arts such as Taijiquan.
This is also called a \textit{contact quality}, a result of grounding plus sharing weight, instead of a simple \textit{touch quality}, a soft feather stroking like a butterfly.
The ultimate goal is to maintain this quality throughout (almost) all time, and therefore also leading to acquiring the skill of \textbf{recognizing weight}.

Unfortunately, this is also usually one of the \textbf{most difficult} skills to acquire for beginners.
Reasons could be such as fear of falling, fear of imposing one's own weight on another person, being a burden and thoughts popup like: ``\textit{Am I too much? Am I too heavy for the other person?}''
A handshake or a tap on the shoulder is common in our society, leaning on someone not.
It's important to learn this principle, yet without stopping to breathe and without tensing up, which can be a massive struggle, especially for beginners.
Another very scary aspect for many people is when going to the ground.
Having lots of weight on you or giving (lots of) weight to that person on the ground is something which needs to be trained.
We don't only share our weight, but we are also sharing our balance, in order not to be off-balanced but ``\textbf{shared-balanced}''.

The more advanced you get, the more you have already acquired that skill, the more often you can choose to deliberately not give weight in the right moment.

\subsection{Pouring Weight}\label{subsec:pouring-weight}

Once we have succeeded to ``gain some weight by grounding'', we can use that to pour it into another person's body.
The emphasis here is to \textbf{slowly and gradually} increase the amount of pressure where the bodies have contact, instead of a quick and sudden shift, which will be difficult and fear evoking movement for your partner; ultimately even dangerous.
Instead, we want to ``announce'' that there is some weight approaching so that our partner can adjust and adapt posture and internal tension/structure to that poured weight.
