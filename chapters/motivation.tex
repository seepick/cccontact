\section{Motivation}\label{sec:motivation}

``\textit{If meditation is the process of being in the present moment, in the so-called here and now, then contact improvisation is the highest form of meditation.}''

Besides the obvious reason for dancers to expand their skill-set, non-professionals are most likely to start with CI for either simply pleasure of doing it, or for personal development purposes.

\subsection{The Power of Touch}\label{subsec:the-power-of-touch}

\begin{wrapfigure}{R}{0.3\textwidth}
\centering
\includegraphics[width=0.25\textwidth]{images/motivation}
\end{wrapfigure}

The tactile experience is most prominent during practicing CI. To be specific, it's the activation of nerve fibers in the skin, so called C-tactile afferents, through gentle and slow stroking with body-like temperature (``warm touch'' leads to release of serotonin, a hormone which makes us feel happy, regulates our emotions and associated with sympathy and interpersonal affection; ``cold touch'' is experienced when being socially excluded, as the skin temperature decreases).
This increases activity in the insular brain region, which is responsible for interoception (perception of the current body state and facilitation of the sense of self).
Through this touch, which can be soft to slide, or deep to the bones, is used to establish a nonverbal, two-way dialogue process, a form of communication.
In the UK alone, about 54\% (among 40,000 participants) want more touch in their lives!
We live in a low-touch society, and mistrusting strangers became a default.

This touch reduces stress (lowered cortisol levels, which is a biomarker for stress-related diseases), increases oxytocin levels (attachment, connectedness) and lower blood pressure.
If a person lacks touch, disorders like depression, a distorted body image, low self-esteem and heightened aggression and self-injurious behavior might occur.
It counteracts a feeling of loneliness and isolation, which is the case in many forms of mental disorders.
Touch releases endogenous opioids, which make us feel relaxed, good, and more resilient.

Yet, touch needs to happen with consent in order to be experienced as something positive.
To want to touch, and want to be touched.
Respecting each other's boundaries, and the possibility to step out of touch at any given moment.
A good CI facilitator will address those things, by, for example, introducing ``no'' exercises, and safety rules at an event.
Especially beginners, who are easily overwhelmed by touch and lust trust in others due to previous bad experiences can benefit from a gender-based group and sharing circles in a safe space to reflect upon.

\subsection{Psychological Health Benefits}\label{subsec:psychological-health-benefits}

It is said to have positive effects on stress relief, relaxation, well-being, happiness, joy, connectedness, empowerment and feeling more fearless.
Also, a clearer experience of a ``sense of self'' and the body; being aware of one's own existence.

The massage effect of ``sharing weight'' seems beneficial against anxiety, depression, ADHD, eating disorders (better sense of one's own body), autoimmune diseases, chronic fatigue syndrome, and more.
It also helps with the activation of the parasympathetic nervous system (``\textit{rest and digest}'').

As the attention is attracted to the ``shared zones of contact'', it brings us to the present moment, and thus becomes a mindfulness practice, distracting us from stressful thoughts.
There is no intention to reach a certain goal, but simply exploring movement and touch itself.
This requires us to be attentive, aware and present.

Finally, the personality gets strengthened, by being more aware of one's own needs, and distinguish oneself from the others.
Patterns can be broken, and a positive impact in resilience-building might occur.

\subsection{Social Bonding}\label{subsec:social-bonding}

CI helps us to relate better to others, to create strong social connections.
The experience of giving and receiving support via ``sharing weight'' for example.
The sensation goes beyond the pure physically into the emotional realm.
To trust each other, to feel safe again.
Presenting our vulnerabilities and instabilities, ultimately leading to a state of intimacy.

Reported experiences during a dance can be: playfulness, surprises, curiosity, flow state, feeling free, being alive; vitality, nourished, energized; connectedness; trust, closeness, deep communication, safe, secure.
The sensation of connectedness feels like being one new, ``third entity'', as it is not clear who gave the impulse to move in a certain way (quote Paxton).
By feeling the other person more, I feel my own boundaries better, thus I feel myself more, ``feeling more home in my own body''.
Yet, negative sensations might occur as well of course, like: boredom, insecurities, doubts, exhaustion, strenuous and being at one's own mercy.

\subsection{Soft Skills}\label{subsec:soft-skills}

Through the practice of CI, many soft skills are also trained and enhanced, such as: teamwork, problem-solving, communication, interpersonal skills, adaptability, dependability, and creative thinking; which are all not only important in our personal, but also in our work life.

In \textbf{teamwork}, we achieve a common goal while supporting the strengths of others.
While we dance with a partner, we are looking and seeing where all the other couples are and moving through and around them.
Our goal is to have an enjoyable dance while letting others have one too; by literally supporting our partner, and choosing together how to navigate the space, we form a small team.
The so-called ``big space'', the heightened awareness of all the people within a room/a jam, we all pay attention to each other, forming a big team.

Every contact is a little kinetic puzzle which requires \textbf{problem-solving} skills.
Like watching two improvisation actors saying ``yes and'' a little too often, and you wonder how they will get out of that; yet with CI the very same happens just physically.
Navigating through space, while watching not to bump into others or hit the wall, and all of that while maybe your partner is trying to balance on your body; all at the same time need to be somehow solved at the moment.

Good \textbf{communication} skills require deep, active listening; yet with CI we mostly stay in silence, so the attentive listening happens only non-verbally with the whole body.
There is never an agenda, a script, a predefined choreography; no one knows upfront what's going to happen, who is going to initiate what; yet surprising as it sounds, with good enough listening skills, there will always be clearly communicated what will happen next; by simply listening and responding.

\textbf{Empathy} is an essential part of interpersonal skills, which is practiced in CI by ``feeling the earth through our partner's feet''.
Although you might not know anything about your current dance partner, through this interaction you will experience so much about how they live in this world -bodies speak beyond words.
It's not only physical characteristics like size or weight, but also preferred, applied techniques and general movement choices.
Through this, we might not be aware of facts like country of origin or the family status, but on a different dimension.

There is such a vast variety in contact improvisation dancers that \textbf{adaptability} becomes actually a very common skill which gets acquired very fast due to its necessity.
Obviously the amount of weight one can share with the partner needs to be adjusted; the differences in height make certain techniques (lifts) maybe more or less easy, or even impossible; and of course the level of experience with body practices in general and CI specifically requires one to adapt to what's there - and possible - at the moment.

A fun contradiction in CI is, that on the one hand you have to be able to completely rely on your partner, building up trust (\textbf{dependability}), and on the other hand to always stay within your own center, being able to catch yourself at any given moment (\textbf{self-reliance}).
Always take care of yourself, while being hyper aware of where your partner is, how he is moving and where he is going; and also knowing what the other people are doing in the room, having one's awareness spread throughout the space.
Bear also in mind that you can't know what's going to happen, predictions are impossible due to the improvisation nature of CI; thus the name.

Lastly, you will also gain some more insights into yourself, knowing more about your personal boundaries and tolerance for risk.
How we want to be treated, and how to be heard; our tendency about taking over a conversation; realizing that we are not as heavy as we imagine ourselves to be.
We learn to be vulnerable and to trust others with our vulnerability (``\textit{When I fall, will you catch me?}'').
