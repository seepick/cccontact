\section{Requirements}\label{sec:requirements}

You might ask yourself for whom is CI fitted?
For whom is it a good idea to join, or not to join?
Are there any physical requirements and how to deal with if you feel a bit touch-averse towards strangers?

In general, it can be said that every-body is welcome, but not necessarily every behavior.
This art-form is not only for the young and well-trained, as there is no real requirement for acrobatic performance; the body just needs to be mobile, and the bones bear some weight.
It is not about \textbf{age}, neither about body ability; yet some techniques might not be possible to do, or would need some adaption.
Students even with the age of 70 would be able to do shoulder lifts, and that person was doing it for many years.
Some of the more crazy things, he mentioned, were better to keep for ``the next life'' though.

Sharing weight, being in contact with another body, exploring what's happening presentlly, where the weight is, the unconscious reaction to balance, \ldots this, anybody can do.
Even with a partially disabled body, with some \textbf{physical disability} like being in a wheelchair, it is possible to play together with a physically fit person.
Or also with blind people, as they are often also super in-tune with weight and different other aspects of perception.
It leads to a very different and interesting kind of exploration, and requires us to renegotiate, to re-explore what kind of communication we can play with.

Even with a huge \textbf{weight} difference, we just need to be more careful about which kinds of lifts to do, and who is carrying whom how.
We need to negotiate what's possible, and sometimes this means no shoulder lifts, and/or no rolling over.
Always respect the abilities of the dance and both bodies by checking each other out, slowly (!) and step-by-step, then it becomes \textit{safer}, yet not necessarily \textit{safe}!
When you are about to engage in more advanced (crazy/dangerous) techniques, always do it with a lot of care and a \gls{guardianangel} who will spot you.

In theory, in an ideal world, everyone is welcome, yet it is not advised for people with severe \textbf{touch aversion} to jump into CI, as a solution to their trauma in that field.
It is recommended to better ``tippy toe'' in different kind of forms first, and only then see if they want to continue with CI\@.
Mental challenges sometimes make people join only one (or even half a) class, or sometimes the teacher might even have doubts and expresses that discreetly.
People who take too many risks (the ``dangerous ones''), which are throwing themselves on unknown partners, won't be denied access to classes or jams, but dancers will most likely get aware of that and keep their distance to them, as it won't feel safe.
