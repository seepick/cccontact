\section{Joining}\label{sec:joining}

To start, it's always nice to check some \textbf{videos} (see the Resources chapter at the end of this book), but please don't be intimidated right away from what you are seeing.
The people you will see on those videos are usually already on a very high level, and that's not what you are going to encounter at first.

Join a \textbf{recommended teacher} and go first to some classes (see ``How to spot a good teacher'' further below).
It is not advisable to immediately go to jams, but only after 10--20 classes, to already have the basic principles embodied.
And don't give up if your very first experience seems bad; maybe you want to just try another class or another teacher.
Be also aware that CI is not only physical, so ask yourself: Why and what are you doing?
This will be very beneficial and inspiring for your first steps with CI\@.

\subsection{Good Teacher}\label{subsec:good-teacher}
%%%%%%%%%%%%%%%%%%%%%%%%%%%%%%%%%%%%%%%%%%%%%%%%%%%%%%%%%%%%%%%%%%%%%%%%%%%%%%%%%%%%%%%%%%%%%%%%%%%%%%%%%%%%%%%%%%%%%%%%

Finding the right teacher is special.
Don’t expect that it will work right from the beginning at the first attempt.
Take your time, have patience.

There are three realms to consider: The physical, the psychological, and you could call it the spiritual (attitude, personality) aspect which we will briefly cover.

\subsubsection{Physical Aspect}
% ----------------------------------------------------------------------------------------------------------------------

The teacher knows what he is teaching, and also being honest about his limitations in that knowledge, or simply put: \textbf{humbleness}.
How does he deal with his own weaknesses, flaws and mistakes?
Is he able to admit to not know something, to be mistaken, or simply, which is very human, to have messed up something?

What is his educational \textbf{background}, and for how long is he doing and teaching it?
Try to look him up on the internet, maybe there are even some videos from him.

\subsubsection{Psychological Aspect}\label{subsec:psychological-aspect}
% ----------------------------------------------------------------------------------------------------------------------

This includes the ability to \textbf{hold space}, in case when internal processes might happen.
Are common rules laid out transparently, and are they comprehensible and challengeable?

\textbf{Safety} is created by the nature of his intention: Does he want to inflate his ego, making himself bigger?
What happens if doubt is expressed, or if he is criticized in front of others?
Does he claim to have a status of a master or guru, or is he ``one of us'' and going for a drink with the students, taking off his teacher's hat?

Does he want to gain something from you outside the class, besides monetary compensation in a formally agreed transaction?
Are personal services and favors expected or even demanded, when you are in individual contact with the teacher?

Contact him personally.
As long as you are not intrusive and stay respectful, a clear, friendly and in-time response to your questions should be possible.

\subsubsection{Spiritual Aspect}\label{subsec:spiritual-aspect}
% ----------------------------------------------------------------------------------------------------------------------

Watch out for the presence of \textbf{competition}.
Does he genuinely want the student to be as good or even better as himself?
Basically, does he have a pure intention to teach, sharing his knowledge?
Whereas this sharing is also always happening two-way though,

It is a good sign if your teacher motivates you to also take a look at other teachers.
