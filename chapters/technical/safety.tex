\section{Safety}\label{sec:safety}

% TODO \chapterCoverImage{safety}

When thinking about safety, we might immediately think about \textbf{physical} safety:
To avoid injuries, disproportional risks, and dangerous situations.
Next to that, we also need to ensure our \textbf{psychological} safety:
The emotions we experience and how they are expressed, and especially when it comes to topics around intimacy and consent.
Lastly, there is also something like \textbf{social} safety: The way we behave and ought to behave; the implicit norms of a subculture, and simple etiquette like dresscode.

\subsection{Protection}\label{subsec:protection}
%%%%%%%%%%%%%%%%%%%%%%%%%%%%%%%%%%%%%%%%%%%%%%%%%%%%%%%%%%%%%%%%%%%%%%%%%%%%%%%%%%%%%%%%%%%%%%%%%%%%%%%%%%%%%%%%%%%%%%%%

Many of us exercise to keep our body \textbf{healthy}.
Yet it sometimes happens that while we are doing so, we injure our body, destroy our health, damage what we initially wanted to protect.
It seems like a terrible contradiction, that the one thing which is supposed to heal, suddenly harms.

That is why safety is of the utmost importance when practicing CI\@.
Just like acrobatics, there are sometimes some more dangerous techniques involved, which require us to pay special attention to this aspect.
Not only the safety of yourself, but we also want to ensure the safety of the \textbf{other people} we are dancing with and the group as a whole.

And of course, we are not only concerned about the physical but also the \textbf{psychological} safety which needs to be addressed.
The complexity of the social construct around behaviors and norms.
How we deal with psychological triggers, overwhelming emotions, and sexual desires.

How can we navigate through all of these areas still staying safe together?
Let's try to answer that.

\subsection{Is it safe?}\label{subsec:is-it-safe?}
%%%%%%%%%%%%%%%%%%%%%%%%%%%%%%%%%%%%%%%%%%%%%%%%%%%%%%%%%%%%%%%%%%%%%%%%%%%%%%%%%%%%%%%%%%%%%%%%%%%%%%%%%%%%%%%%%%%%%%%%

Safety, in obvious terms of ``free of injuries'', should always come first when practicing any potentially risky movement style such as CI\@.
Furthermore, knowing, respecting and adhering to the local, specific \textbf{subcultural norms} will also lead to psychological/emotional safety within the group.
Besides all these concerns about safety, it is safe to say that CI can be practiced by anyone: professionally trained dancers, recreational movers, athletes, dancers of all abilities and ages.

Yet, if you ask yourself whether CI is safe, then the short answer is simple: No.
It can be \textbf{safe enough}, though, which is what we aim for.
It is for sure less safe than Judo or MMA, but more safe than running or pottery.
Furthermore, it is a high-intense form of dancing, and also more risky due to not knowing what your partner is bringing.
Safety just can't be reached in the outside world, meeting each other; neither physically injured nor psychologically in pain.
So that's why it's important to be aware of the existing risks, and at the same time expanding the boundaries of what's possible.

\subsection{Consent}\label{sec:consent}
%%%%%%%%%%%%%%%%%%%%%%%%%%%%%%%%%%%%%%%%%%%%%%%%%%%%%%%%%%%%%%%%%%%%%%%%%%%%%%%%%%%%%%%%%%%%%%%%%%%%%%%%%%%%%%%%%%%%%%%%

Consent is mainly about being able to express as well as hearing and appropriately responding to a ``yes'' and a ``no'', which are both possibly, and in the context of CI, preferably expressed non-verbally.

Of course, this starts already when engaging in a dance with someone, but is most importantly when it comes to more advanced techniques like lifts.
A ``no'' could be expressed due to lack of experience with and thus trust into each other, fear of being lifted, considering the other person's weight as too heavy, or other reasons.
There are many elegant ways out so that we can hold ourselves responsible for withstanding our \textbf{boundaries} instead of making others accountable for crossing them, ultimately ``self-dis-empowering'' ourselves.

Whenever a gentle hint by movement does not have the desired result, we can of course always cough, or if all of those subtle cues wouldn't work, simply speak and be very explicit about our wishes by saying assertively and gently with a smile: ``No''.

\subsection{Technical Safety}\label{subsec:technical-safety}
%%%%%%%%%%%%%%%%%%%%%%%%%%%%%%%%%%%%%%%%%%%%%%%%%%%%%%%%%%%%%%%%%%%%%%%%%%%%%%%%%%%%%%%%%%%%%%%%%%%%%%%%%%%%%%%%%%%%%%%%

There are many, many things which can lead to severe injuries, especially when being performed by more inexperienced students.
For the sake of simplicity and conciseness though, we will limit the list to a few, the most common ones, here:

\vspace{10pt}

\noindent \textbf{Head above ass}: When lifting a partner, the shoulders must always be higher than the pelvis, as otherwise the person will slide down overhead with all kinds of undesirable consequences.

\vspace{5pt}
\noindent \textbf{Don't grab}: When encompassing the partner's limb, for example, that part of his body will be immovable and thus prevents him from using it as ``landing gear'' when needed, potentially leading to an accident.
It additionally removes the agency of the other person and is considered to be simply rude.
There is a fine line between ``polite manipulation'' by pushing a bit, guiding, and even offering a welcomed structure/base for the other person on the one hand, and a forceful, dominant directness on the other, which as well can be ok with some dance partners, those you already know well, and they consented to it.
To be fully safe though, especially with strangers or in unknown places, refrain from too much manipulation, or also ``pedipulation'', meaning using the feet to change the shape of your partner.

\vspace{5pt}
\noindent \textbf{Don't interlock}: When people perform a back lift and interlock the arms (and God forbid simultaneously lowering the head below the pelvis) and they fall, the flyer might get into a state of panic and tenses up his arms, while the base will fall and not being able to use his arms, thus falling on his face with the weight of the lifted partner.
I leave it up to your own imagination of the consequences of such an event.

\vspace{5pt}
\noindent \textbf{Don't jump}, especially when being lifted as a flyer.
Jumping and other techniques leading to (uncontrolled) collisions was an essential part of old-school CI, yet in today's development more emphasis is being put on safety.
The impact of jumping on a partner can lead to injuries, especially on the spine, that's why we prefer to pour our weight softly on our partner so that he can adjust, shift his weight and readjust his structure.

\vspace{5pt}
\noindent \textbf{Momentum} is debatable, yet if safety has the highest priority, then it should be avoided as it takes away reversibility because there is no control anymore in the movement.

\vspace{10pt}

For any more risky technique, it's always a good idea to have someone \textbf{spotting} you, someone we usually refer to as a ``\gls{guardianangel}'', who is trying to catch if someone falls on the ground, usually more concerned about the head and less about the feet; never grab the feet!
Don't grab, yet stay ready to engage in case something goes wrong and grab them well to make them land as softly as possible.
For the even more extreme techniques, it can be even advisable to have two guardian angels at the same time.

\subsection{Intimacy}\label{subsec:intimacy}
%%%%%%%%%%%%%%%%%%%%%%%%%%%%%%%%%%%%%%%%%%%%%%%%%%%%%%%%%%%%%%%%%%%%%%%%%%%%%%%%%%%%%%%%%%%%%%%%%%%%%%%%%%%%%%%%%%%%%%%%

The psychological aspect of safety usually revolves around the topics of intimacy, sensuality and sexuality and how to set, perceive and respect \textbf{boundaries} of each other.
Take for example, when engaging in floor work or simply using the rolling point of contact over more intimate body parts, people --and especially women-- can quickly feel uncomfortable.
From the outside (imagine people who don't know the practice) this might seem very intimate, but for the more experienced practitioners, it's pure exploration and physics, an innocent and pure way of play.

Not only on the physical level, because of the proximity in touch, but also on the sensual/sexual level, CI can feel frightening for some people at times.
Just remember, this is not a \textbf{tantra practice} or similar, so sensuality is NOT (!) at CI's core; we don't engage in therapeutic energy work, don't activate Kundalini or seek for cuddle puddles.

Sensuality is a wonderful thing, but it is not really appropriate to go into ``melting'' into your partner, falling in love, or even sexuality, and expressing this via caressing touch, cuddles and somewhat ``lustful behavior'' during CI\@.
This border can be crossed at times, as the positive bonding effects of touch can easily lead us to engage into more.
At this moment, it is advised to \textbf{separate}, to ``cool down'' and re-engage with the dance, instead of the partner.
Caressing the skin of another person, to ``cuddle up'' and all of it is great, amazing, and please do more of it, but please refrain from doing so on the CI dance floor.

There is often a partner \textbf{bodywork} exercise at the end of a class/jam; a typical thing dancers would do.
It is meant to be a more technical, medical, or sports massage, rather than a personal, emotional, intimate touch session; don't confuse those.

If sensuality/sexuality still happens, the \textbf{teacher} will usually take that off very fast, or the jam facilitator, if present.
Cuddles, and cuddle puddles, will often happen and are being tolerated most of the time, yet they are not intentional!
If your intention is to engage in cuddles, the group will notice that behavior, and they might not want to dance with you anymore because of that.
There is of course no strict rule about it, no prohibition, and in some jams they totally tolerate it, even few invite it, but mostly it just doesn't happen.

In classes, it happens very, very rarely that certain individuals will look to satisfy their sexual needs by ``rubbing their horny body on someone else's''.
The teacher will usually here as well take care of a safe space for participants, and stop any form of sexual intended or predatory behavior right away, declaring it as a clear ``no-go''.
That's also why people should go first to a class to learn the \textbf{norms and values}, instead of dropping in a jam right away.
At jams, without a supervisor, no one knows how often such behavior might occur; maybe relatively more often?
Most of the time there is an organizer though, which can repeat clear rules and guidelines in an opening circle, minimizing the probability of such non-welcome behavior.
Comparing CI to Ecstatic Dance or Tantric Dance, in CI there are for sure way, way less ``horny people'' showing up, so you should be safe.

\subsection{Etiquette}\label{subsec:etiquette}
%%%%%%%%%%%%%%%%%%%%%%%%%%%%%%%%%%%%%%%%%%%%%%%%%%%%%%%%%%%%%%%%%%%%%%%%%%%%%%%%%%%%%%%%%%%%%%%%%%%%%%%%%%%%%%%%%%%%%%%%

In every subculture, there are certain (social) \textbf{norms and rules} established which are not written down anywhere.
They somehow float in the minds of the people, the ``bigger body'', everyone carrying an incomplete piece in a slightly different variation, \textbf{implicitly} without any for of mentioning it, and sometimes even without being consciously aware of it.
Yet, if they are broken even just once, there will be consequences for such misbehavior executed by the group.

I, personally, find it useful to try to capture those and make them \textbf{explicit}, so we can avoid unintentional misbehavior and establish as much \textbf{harmony} as possible.
Just know, that every scene, maybe even every group around a specific teacher, has their very own set of normative rules, and that's why this list (as with any other attempt to ``list things'') is just a possible set of many.

\begin{itemize}
    \item \textbf{Talking} is considered to be a distraction during CI classes and especially during jams.
    If verbal communication is indeed needed, there is no obligation for total silence, but be mindful whether it is truly necessary in this moment, or whether it is merely some irrelevant chit-chat.
    In any case, try to keep it as short as possible and the volume low, to not distract your fellow practitioners.
    Otherwise, you can always step off the dance floor and speak properly a bit further away if really needed.
    \item Similarly to talking behavior, which disturbs others, please refrain from \textbf{parking} on the dance floor, meaning: Lying on the floor, maybe even with eyes closed, and being unaware of the surroundings and the other people.
    It is not only an annoyance, as it takes a lot of space (and attention), but it also adds another, yet unnecessary risk factor.
    \item The combination of both (talking and parking) among two or more people, also called \textbf{socializing}, is considered a no-go to do on the dance floor, for the reasons mentioned above.
    For that, please use the edges of the room, and if it's a smaller room, go outside, and if that's not available, just practice letting go and stay in the physical, present moment.
    \item \textbf{Stating your name} is just polite when dancing with someone new.
    At least do so at the very end, and also ask for the other person's name.
    It's just a nice way to acknowledge the person, and potentially useful to furthermore create a friendly bond.
    \item Although probably obvious, yet worth mentioning is \textbf{body hygiene}: Wear fresh clothes (and if you get sweaty easily and/or the temperature is high, also bring several shirts), make sure you have no overly intense bad breath (after having eaten a yummy ``Döner Kebab''), and have taken a shower before engaging in such an intimate movement form with other people.
    There is no need to smell like a perfume shop, as too intense scents are also uncomfortable.
    \item \textbf{Starting slowly} will help you to get to know each other, before \textit{rushing into someone's living room}.
    It's like a handshake, to get a glimpse of whom the other person is, where he is regarding skill and experience, personal preferences and pathways, and then take it from there, building it up, gaining trust, and going on an adventure together.
    \item \textbf{Eye gazing} is another wonderful tool to deeply connect with another human being, yet during CI, we refrain from doing so.
    In our world, it is a wonderful tool to scare people away, making them not liking to dance with you anymore.
    Hopefully, your partners will tell you that this is not the way we do it here.
    We use our peripheral vision to see, and usually don't constantly dance front-to-front.
    \item Speaking about front: It is totally fine to \textbf{roll over the front} side of your partner, and there is no need to avoid certain body parts, like breasts, genital area or similar.
    Once we can depersonalize the partner, seeing it as a physical object and letting go of the stories, there is no need to avoid certain body parts, and rather develop a more innocent and reality bound perspective.
    Yet, of course, we do not seek for it, and if possible, and not disturbing the dance and pathways too much, we try not to go directly for those parts.
\end{itemize}

If you participate in a class, there will be more \textbf{guidance}, and it will be easier for you to spot what's ``right and wrong'' behavior; in a \textbf{jam} though it might be a bit more difficult, especially if there is no hosting facilitator.
Take about 10 minutes at the beginning of a jam to sit, simply watch and observe the communal implicit norms.
Every jam is different, yet there is a common set of baseline rules as mentioned already above.

\subsection{Clothing}\label{subsec:clothing}
%%%%%%%%%%%%%%%%%%%%%%%%%%%%%%%%%%%%%%%%%%%%%%%%%%%%%%%%%%%%%%%%%%%%%%%%%%%%%%%%%%%%%%%%%%%%%%%%%%%%%%%%%%%%%%%%%%%%%%%%

Appropriate clothing is found in every setting where a group forms, sometimes for very rational, practical reasons, and sometimes just to establish some kind of esthetical norm to agree to, a so-called \textbf{dress code}.
If severely violated, it can even lead to the fact that people will feel opposed to dancing with you because of inappropriate clothing, and not because of you as a person!

As such, it is recommended to wear \textbf{long pants and long sleeves} to be able to slide on the ground and have enough friction, as well as sweat soaking capabilities.
Refrain from wearing clothes which limit your movements, like firm jeans, and also shirts (or pants) with buttons on them, as rolling on your partner with your full weight might hurt them, and also your pretty shirt might break; also no zippers for similar reasons.
In short: Normal, dancing, movable sports clothes.
And also plenty of them, especially shirts when the weather is warmer; the minority of us enjoy it to have a partner who is soaked in sweat and dance close with them.

In CI, we need some friction for \textbf{grip}, and for that matter, plastic fabrics will slide more (too much), and thus cotton is better.
Your beloved Adidas pants are not advised as they are too slippery; sorry.
Speaking of grip: It's better to dance \textbf{barefoot}, without socks to avoid slipping, or socks with those plastic nobs on the bottom if you got some of those.

The single utility though which could be useful would be knee pads, as there might be plenty of floor work and carrying people on your body while in the table-top position.
They shouldn't restrict your movement, so preferably choose some which are a bit thinner and made from an elastic material.

Obviously, \textbf{jewelry}, like big earrings, bracelets, rings, and basically any form of it, is not advised as they can get trapped on the other person's body and clothes, ultimately hurting someone.

Ultimately, we don't dress to impress, but rather prefer to wear our \textbf{cozy pyjamas}.
Newcomers often stand out at the beginning wearing overly sexual clothes, showing a lot of skin, or wearing their gym clothes.
All of it which isn't wrong by itself, not at all, it just might not be the right place for it.
You wouldn't wear your favorite onesie pyjama at your mother's funeral, either, would you?!
