\chapter{Safety}\label{ch:safety}

% TODO \chapterCoverImage{safety}

As a physically demanding practice, CI does not guarantee physical safety.

To avoid injuries, disproportional risks, and dangerous situations.


\subsection{Protection}\label{subsec:protection}
%%%%%%%%%%%%%%%%%%%%%%%%%%%%%%%%%%%%%%%%%%%%%%%%%%%%%%%%%%%%%%%%%%%%%%%%%%%%%%%%%%%%%%%%%%%%%%%%%%%%%%%%%%%%%%%%%%%%%%%%

Many of us exercise to keep our body \textbf{healthy}.
Yet it sometimes happens that while we are doing so, we injure our body, destroy our health, damage what we initially wanted to protect.
It seems like a terrible contradiction, that the one thing which is supposed to heal, suddenly harms.

That is why safety is of the utmost importance when practicing CI\@.
Just like acrobatics, there are sometimes some more dangerous techniques involved, which require us to pay special attention to this aspect.
Not only the safety of yourself, but we also want to ensure the safety of the \textbf{other people} we are dancing with and the group as a whole.

And of course, we are not only concerned about the physical but also the \textbf{psychological} safety which needs to be addressed.
The complexity of the social construct around behaviors and norms.
How we deal with psychological triggers, overwhelming emotions, and sexual desires.

How can we navigate through all of these areas still staying safe together?
Let's try to answer that.

\subsection{Risk}\label{subsec:risk}
%%%%%%%%%%%%%%%%%%%%%%%%%%%%%%%%%%%%%%%%%%%%%%%%%%%%%%%%%%%%%%%%%%%%%%%%%%%%%%%%%%%%%%%%%%%%%%%%%%%%%%%%%%%%%%%%%%%%%%%%

Safety, in obvious terms of ``free of injuries'', should always come first when practicing any potentially risky movement style such as CI\@.
Furthermore, knowing, respecting and adhering to the local, specific \textbf{subcultural norms} will also lead to psychological/emotional safety within the group.
Besides all these concerns about safety, it is safe to say that CI can be practiced by anyone: professionally trained dancers, recreational movers, athletes, dancers of all abilities and ages.

Yet, if you ask yourself whether CI is safe, then the short answer is simple: No.
It can be \textbf{safe enough}, though, which is what we aim for.
It is for sure less safe than Judo or MMA, but more safe than running or pottery.
Furthermore, it is a high-intense form of dancing, and also more risky due to not knowing what your partner is bringing.
Safety just can't be reached in the outside world, meeting each other; neither physically injured nor psychologically in pain.
So that's why it's important to be aware of the existing risks, and at the same time expanding the boundaries of what's possible.

\section{Safe Behavior}\label{sec:safe-behavior}
%%%%%%%%%%%%%%%%%%%%%%%%%%%%%%%%%%%%%%%%%%%%%%%%%%%%%%%%%%%%%%%%%%%%%%%%%%%%%%%%%%%%%%%%%%%%%%%%%%%%%%%%%%%%%%%%%%%%%%%%

\subsection{Rules}\label{subsec:rules}
% ----------------------------------------------------------------------------------------------------------------------

There are many, many things which can lead to severe injuries, especially when being performed by more inexperienced students.
For the sake of simplicity and conciseness though, we will limit the list to a few, the most common ones, here:

\vspace{10pt}

\noindent \textbf{Head above ass}: When lifting a partner, the shoulders must always be higher than the pelvis, as otherwise the person will slide down overhead with all kinds of undesirable consequences.

\vspace{5pt}
\noindent \textbf{Don't grab}: When encompassing the partner's limb, for example, that part of his body will be immovable and thus prevents him from using it as ``landing gear'' when needed, potentially leading to an accident.
It additionally removes the agency of the other person and is considered to be simply rude.
There is a fine line between ``polite manipulation'' by pushing a bit, guiding, and even offering a welcomed structure/base for the other person on the one hand, and a forceful, dominant directness on the other, which as well can be ok with some dance partners, those you already know well, and they consented to it.
To be fully safe though, especially with strangers or in unknown places, refrain from too much manipulation, or also ``pedipulation'', meaning using the feet to change the shape of your partner.

\vspace{5pt}
\noindent \textbf{Don't interlock}: When people perform a back lift and interlock the arms (and God forbid simultaneously lowering the head below the pelvis) and they fall, the flyer might get into a state of panic and tenses up his arms, while the base will fall and not being able to use his arms, thus falling on his face with the weight of the lifted partner.
I leave it up to your own imagination of the consequences of such an event.

\vspace{5pt}
\noindent \textbf{Don't jump}, especially when being lifted as a flyer.
Jumping and other techniques leading to (uncontrolled) collisions was an essential part of old-school CI, yet in today's development more emphasis is being put on safety.
The impact of jumping on a partner can lead to injuries, especially on the spine, that's why we prefer to pour our weight softly on our partner so that he can adjust, shift his weight and readjust his structure.

\vspace{5pt}
\noindent \textbf{Momentum} is debatable, yet if safety has the highest priority, then it should be avoided as it takes away reversibility because there is no control anymore in the movement.

\vspace{10pt}

For any more risky technique, it's always a good idea to have someone \textbf{spotting} you, someone we usually refer to as a ``\gls{guardianangel}'', who is trying to catch if someone falls on the ground, usually more concerned about the head and less about the feet; never grab the feet!
Don't grab, yet stay ready to engage in case something goes wrong and grab them well to make them land as softly as possible.
For the even more extreme techniques, it can be even advisable to have two guardian angels at the same time.
