\section{Jargon}\label{sec:jargon}

% TODO \chapterCoverImage{jargon}

There are no universally acknowledged names for techniques or exercises, but different teachers try to invent, share, and give credit to their inventions.
Over time, an \textbf{oral tradition} has been established of passing information.
The following chapter explains language and images developed in the area ``I grew up in'' and are very \textbf{local} and not universal at all.
Yet they might inspire others, or at least provide some humorist, personal benefit for you while reading it.

\subsection{Purpose}\label{subsec:purpose}
%%%%%%%%%%%%%%%%%%%%%%%%%%%%%%%%%%%%%%%%%%%%%%%%%%%%%%%%%%%%%%%%%%%%%%%%%%%%%%%%%%%%%%%%%%%%%%%%%%%%%%%%%%%%%%%%%%%%%%%%

Introduction of jargon, using technical, domain-specific terms, is useful to express an idea, as otherwise many, many sentences would be needed whereas instead a single word can suffice as well.
The whole purpose of introducing and using jargon is not to smart-ass, to show off one's own superiority, an act of ego arrogance, or to exclude others by using an exclusive language.
It is, or at least should be, solely used to increase the information exchange and the information \textbf{density and precision}.

\subsection{General}\label{subsec:general}
%%%%%%%%%%%%%%%%%%%%%%%%%%%%%%%%%%%%%%%%%%%%%%%%%%%%%%%%%%%%%%%%%%%%%%%%%%%%%%%%%%%%%%%%%%%%%%%%%%%%%%%%%%%%%%%%%%%%%%%%

\noindent \textbf{\Gls{thebox}} -- This refers to the whole area of the upper body, the torso, including everything between the shoulders and hips.

\noindent \textbf{\Gls{smalldance}} -- The process of becoming aware of the subtle, unconscious micro movements of the body to maintain its balance.

\noindent \textbf{\Gls{kinesphere}} -- The space which can be reached by the body and its limbs with any kind of movement without making a step.

\vspace{5pt}

\noindent \ldots \textit{for further general terms, see the glossary at the end of this book} \ldots

\subsection{Onomatopoeia}\label{subsec:onomatopoeia}
%%%%%%%%%%%%%%%%%%%%%%%%%%%%%%%%%%%%%%%%%%%%%%%%%%%%%%%%%%%%%%%%%%%%%%%%%%%%%%%%%%%%%%%%%%%%%%%%%%%%%%%%%%%%%%%%%%%%%%%%

Sometimes, concepts are just much too fuzzy to be properly put into definite words.
Or there might have just no proper words yet been established, which made it necessary to develop our own few words, or even better: ``sound-terms'' (or onomatopoeia\footnote{Onomatopoeia, which has its Greek roots with the meaning of ``name-making'', which is the formation or use of words such as ``buzz'' or ``murmur'' that imitate the sounds associated with the objects or actions they refer to.}) to convey a certain meaning, quality, fuzzy principle or some abstract phenomena closer to what it meant to be:

\begin{itemize}
    \item \textbf{Botsen}: A conflict which arises when the teacher's instructions lead to a resistance based on an ``internal wisdom''; when the ``inner teacher'' and the ``outer teacher'' clash, we usually opt for the inner one and trust our gut feeling; e.g.\ Being told to jump and do a roll, but it doesn't feel safe and fear/resistance comes up, so you simply don't do it.
    (actually a Dutch word, meaning to clash, to collide; ``bot'' Dutch for bone; like two fists smashed on each other)
    \item \textbf{Hoopedy Di Poop}: Techniques which look overly impressive (good to show-off), yet not necessarily show much skill though, like jumping on each other; e.g.\ fancy technical, acrobatic stuff which usually comes with a higher risk for injury.
    \item \textbf{La La Land\footnote{Yes, ``\textit{La La Land}'' is also the title of a US musical romantic comedy-drama movie from 2016 with Ryan Gosling and Emma Stone.}}: Referring to anything which is more commonly used in the area of spirituality/religion, metaphysics, esoteric, new age talk and superstitions.
    Yet, sometimes we are still using concepts from those domains nevertheless due to the lack of any other better alternatives; e.g.\ think about when talking about ``reaching beyond the physical body'', which is of course technically not possible, but it gives the right image and intention.
    \item \textbf{Mooshy Mooshy}: (pronounced ``muschi muschi'') Indicating that during a dance, a too sensual, and thus inappropriate atmosphere arises, with no clear intentions; ranging from simply moving very slowly caressing down to \textit{cuddle puddles} where several people like cuddling on the ground.
    This can especially often be observed when starting with dancing on the ground, often preceded by a body-scan; things then often end up in some pile of bodies\footnote{Hopefully your cuddle puddles don't end up as it did in the movie ``\textit{Perfume: The Story of a Murderer}'' in the end.}.
    \item \textbf{Oomph}: (pronounced ``umpf'') The preferred quality of contact between two bodies which is characterized by properly giving/sharing weight, creating a sensation of considerable amount of pressure, opposed to only slight touch; feeling each other's groundedness.
    \item \textbf{Weee}: A melodic sound usually expressed in moments of heightened levels of alertness/fear, to down-regulate one's own nervous system, counteracting fear/panic, and allowing the body and people around to relax, release tension, and calm down; also to simply express joy about a movement, making everyone smile; e.g.\ when performing more risky lifts.
\end{itemize}

\subsection{Animals}\label{subsec:animals}
%%%%%%%%%%%%%%%%%%%%%%%%%%%%%%%%%%%%%%%%%%%%%%%%%%%%%%%%%%%%%%%%%%%%%%%%%%%%%%%%%%%%%%%%%%%%%%%%%%%%%%%%%%%%%%%%%%%%%%%%

Calling things by animal names is beneficial as they have a very visual characteristic, conveying a visceral experience, because of the qualities we associate with those animal.
We mostly use their names to refer to positions, techniques and ``qualities of body''.

% TODO good gorilla
\begin{itemize}
    \item \textbf{Bear}: Similar to the koala, but while sitting on the ground, and the bear hugging around the torso of his partner.
    \item \textbf{Banana}: Not really an animal, but anyway a useful metaphor where the arms and legs are stretched out long, shaping the whole body like a banana; it can be practiced on the ground (a way of movement on the ground, rolling sideways and only the core touching the floor), but applied often while being on the back of a partner to spiral upwards.
    \item \textbf{Chicken Wing}: Using a ``semi lock'' with the arm pit on the partner while being lifted, or similar; often necessary if the centers are not properly stacked and becoming unbalanced.
    \item \textbf{Chicken Leg}: Same as the chicken wing, but with the hip being flexed instead the elbow.
    \item \textbf{Crab}: The opposite of the octopus movement quality: Rigid, sharp, direct, and staccato.
    Think of the exoskeleton of the crab, when it walks sideways with its stiff limbs.
    \item \textbf{Koala}: When being lifted (or better: jumping on the partner's shoulder) hugging the upper body of the partner sideways and thus being very close to his center, while locking with your arms and legs; either around shoulders (more common) or pelvis (less common).
    It can be used as a preliminary technique for a shoulder lift.
    \item \textbf{(Little) Elephant}: Although we do like elephants, we don't like them in our studio, as their name is used to refer to steps/walking, or landing of the feet, which make a loud sound, indicating that there was no control and/or too much stiffness.
    Landing softly with no sound indicates control of movement, reversibility, precision and awareness, and also is a strong indicator of degree of safety with a partner.
    \item \textbf{Little Monkey}: As a little animal (a.k.a.: ``table-top'') but with knees lifted (a.k.a.: ``bear position'') with a light and fluent walking movement.
    Whenever we land silently, it is done so with control and elegance, which ultimately can prevent (serious) injuries.
    \item \textbf{Little Animal}: A table-top position on all fours, yet emphasizing a more dynamic, alive quality than a regular, wooden table.
    Also, when the painter is performing some movements, the little animal is actively supporting him by shaping his back accordingly, tilting and turning, flexing and extending.
    \item \textbf{Panda}: Similar to koala (and bear), but a more specific way of hugging, with belly to belly; often used while on the ground lying, keeping the centers connected all the time.
    \item \textbf{Snake}: Similar to octopus, but with a slightly different image to connect differently; having many movements in hands and spine going on simultaneously.
    \item \textbf{Octopus}: A movement quality which indicates aliveness/relaxation in all joints/body parts, each of them being controlled by their own intelligence; fluid, soft and smooth; opposite of the crab.
\end{itemize}
