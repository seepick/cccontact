\section{Movement Qualities}\label{sec:movement-qualities}

There are different imagined \textbf{dimensions} we can play with, to tap into different qualities on how to use our body, giving it a different flavor, allowing us to do different techniques, and by knowing the pros and cons of each quality, we can apply them in the right moment to elevate our technical skill and also keep things safe for ourselves and our partner.

\subsection{Movement Patterns}\label{subsec:movement-patterns}
%%%%%%%%%%%%%%%%%%%%%%%%%%%%%%%%%%%%%%%%%%%%%%%%%%%%%%%%%%%%%%%%%%%%%%%%%%%%%%%%%%%%%%%%%%%%%%%%%%%%%%%%%%%%%%%%%%%%%%%%

Through a heightened awareness of communication through movement, touch, and sharing weight, we explore the space and the connection between, through mutual physical cooperation.
Fundamental movement patterns are:

\begin{itemize}
    \setlength\itemsep{0em}
    \item [] \textbf{Yielding}: softening/surrendering into incoming force or to gravity
    \item [] \textbf{Pushing}: expansion, taking up space
    \item [] \textbf{Reaching}: extending physically or meta-physically
    \item [] \textbf{Pulling}: contraction, until collapsing
    \item [] \textbf{Releasing}: relaxing into what's contracted before
\end{itemize}

All of those movements can be done easily with little muscular effort if basic physical forces are acknowledged and taken advantage of, such as gravity (falling), momentum, inertia, balancing, etc.
And all of those while staying in contact.

\subsection{Muscle Tone}\label{subsec:muscle-tone}
%%%%%%%%%%%%%%%%%%%%%%%%%%%%%%%%%%%%%%%%%%%%%%%%%%%%%%%%%%%%%%%%%%%%%%%%%%%%%%%%%%%%%%%%%%%%%%%%%%%%%%%%%%%%%%%%%%%%%%%%

We can play a lot with tone, which is the amount of tension we create in our muscles that makes us either more relaxed or more stiff.
In general, we prefer to maintain the least amount of effort, a minimum muscle tension; as little as possible, as much as necessary.
By being more relaxed, we are more flexible, can adapt to an ever-changing situation, and also are more receptive via our tactile sense, being able to receive more information, to listen better.
To have some images to play with, think of moving through air (well, you don't have to imagine that, as we constantly do that --duh).
Instead, think of you being a cloud, floating through the sky (sure, that's better, we usually never do that one).
Now increase tension by imagining moving through water, how it creates a small but continuous resistance, preventing you from sharp, edgy movements, from breaking a fluent pathway.
The next increase in resistance could be achieved by imagining something like honey, sticky and slowing down your movements, having you to add more muscle effort.
And as a final step, imagine being stuck in concrete, which maybe has not yet fully harden, still making you almost unmovable.

\subsubsection{Relaxation}

We usually move rather relaxed; a body which is ready for action yet open for receiving tactile stimulus, open for information.
We try to achieve that by deep breathing, by keeping a fluid movement quality (``octopus''-quality) and also avoid a staring eye gaze.

Yet, a relaxed state should not be confused with a collapsed one.
An active state is also not the same as a hyper-tensed one.
Within this spectrum of non-extremes, we ought to find the optimal amount of \textbf{muscle tone} which is appropriate for a given situation.


\subsection{Speed}\label{subsec:speed}
%%%%%%%%%%%%%%%%%%%%%%%%%%%%%%%%%%%%%%%%%%%%%%%%%%%%%%%%%%%%%%%%%%%%%%%%%%%%%%%%%%%%%%%%%%%%%%%%%%%%%%%%%%%%%%%%%%%%%%%%

Moving on to the very obvious dimension of slowness and fastness.
The slow can be extremely slow, to gain lots of information of internal sensations.
The fast can be released in an explosive manner, like a shockwave through the whole body.
Both, and everything in-between, can be alternated rapidly, to gain more control (of speed).

\subsection{Kinesphere}\label{subsec:kinesphere}
%%%%%%%%%%%%%%%%%%%%%%%%%%%%%%%%%%%%%%%%%%%%%%%%%%%%%%%%%%%%%%%%%%%%%%%%%%%%%%%%%%%%%%%%%%%%%%%%%%%%%%%%%%%%%%%%%%%%%%%%

The degree of extension of the limbs into the space --without stepping-- is called \gls{kinesphere} with which we can play with.
We could segregate it into a small (body), medium (elbows, knees), large (wrists, ankles) and extra large (fingers, toes), and something more abstract going even beyond (projecting outwards), the universe.
Each of them is creating a differently sized ball, or more like an egg shape, around us in which we are limited to move within and also want to stay in constant contact with.

\subsection{Levels}\label{subsec:levels}
%%%%%%%%%%%%%%%%%%%%%%%%%%%%%%%%%%%%%%%%%%%%%%%%%%%%%%%%%%%%%%%%%%%%%%%%%%%%%%%%%%%%%%%%%%%%%%%%%%%%%%%%%%%%%%%%%%%%%%%%

Next to extension into space, we can, of course, take different levels in vertical space: Up (standing), middle (hinged, or on hands and knees being a ``little animal'') and floor work (lying on the ground).
With the help of relaxation and tension, we can quickly change our position on the vertical axis and play with explosive dynamics.
Also, mirroring your dance partner, staying at a different (not the same as with imitating) level then he, can be an challenging, fun, and engaging experience to explore.

\subsection{Isolation}\label{subsec:isolation}
%%%%%%%%%%%%%%%%%%%%%%%%%%%%%%%%%%%%%%%%%%%%%%%%%%%%%%%%%%%%%%%%%%%%%%%%%%%%%%%%%%%%%%%%%%%%%%%%%%%%%%%%%%%%%%%%%%%%%%%%

The isolation of certain body parts can also be fun --and challenging at the same time-- to play with.
The most simple one is to divide the body into upper and lower, left and right side, same side or cross side (homo- or contra-lateral, see the anatomy section for explanation of terminology), or move only in a certain plane.

\subsection{Shapes}\label{subsec:shapes}
%%%%%%%%%%%%%%%%%%%%%%%%%%%%%%%%%%%%%%%%%%%%%%%%%%%%%%%%%%%%%%%%%%%%%%%%%%%%%%%%%%%%%%%%%%%%%%%%%%%%%%%%%%%%%%%%%%%%%%%%

The forms we are drawing in air can be another dimension.
Think of straight lines (edgy, staccato) versus roundness (flowy, fluid, air).
People who have experience with the practice of 5 Rhythms might be familiar with those concepts and make use of them.

\subsection{Combination}\label{subsec:combination}
%%%%%%%%%%%%%%%%%%%%%%%%%%%%%%%%%%%%%%%%%%%%%%%%%%%%%%%%%%%%%%%%%%%%%%%%%%%%%%%%%%%%%%%%%%%%%%%%%%%%%%%%%%%%%%%%%%%%%%%%

To bring those qualities to a next level, try to \textbf{combine} them in different ways.
Often slow, fluid and soft go together, but how about changing one of them to the other extreme?!
Or the legs are in the large kinesphere being soft and staccato, while the arms in the small kinesphere and hard and fluid.
Have a partner telling you what parts should be in which quality, to surprise yourself by finding combinations you would not have thought of alone.
