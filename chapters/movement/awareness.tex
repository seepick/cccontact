\section{Body Awareness}\label{sec:body-awareness}

Let's have a look of how we actually become aware of our bodies.
The biological and physiological base of how this fascinating organic machinery --roughly-- works.

\subsection{Vestibular System}\label{subsec:vestibular-system}
%%%%%%%%%%%%%%%%%%%%%%%%%%%%%%%%%%%%%%%%%%%%%%%%%%%%%%%%%%%%%%%%%%%%%%%%%%%%%%%%%%%%%%%%%%%%%%%%%%%%%%%%%%%%%%%%%%%%%%%%

This is our sense of \textbf{balance} and spatial orientation for the purpose of movement coordination, and is all well known to us.
It consists of two components: The (three, as there are three dimensions) \textit{semicircular canals} for rotational movements, and the \textit{otoliths} for linear accelerations.
Signals from those are sent to the muscles to keep us upright and control posture, allowing us to maintain our desired position in space.
Together with our proprioception, we can understand our body's dynamics and kinematics in any given moment.

\subsection{Proprioception}\label{subsec:proprioception}
%%%%%%%%%%%%%%%%%%%%%%%%%%%%%%%%%%%%%%%%%%%%%%%%%%%%%%%%%%%%%%%%%%%%%%%%%%%%%%%%%%%%%%%%%%%%%%%%%%%%%%%%%%%%%%%%%%%%%%%%

\Gls{proprioception} allows us --as a sort of 6th sense, the ``kinesthetic sense''-- to be consciously aware of movement, force and body \textbf{position}.
It tells us our body's position in the space, that is the relative positioning of neighboring body parts, and the strength of effort needed for movement.
Try for example to close your eyes and touch your nose; you will be able to do this without looking (in a mirror, or in complete darkness) because of little cells (little \textit{spindles} which are spring-like protein molecules which are stretched inside your muscles) in your body being aware of the \textbf{amount of stretch} they experience (joint position sense), which is then processed subconsciously by your brain giving raise to a bodily sensation.
Everyone is familiar with the knee-jerk reflex, where the patellar tendon is rapidly stretched to an extreme (usually with the help of a hammer), which leads to an immediate response (a reflex) to counteract that and protect the tissue from injury.

Related sensations are \textit{exteroception}, the perception of the outside world, and \textit{interoception}, the perception of internal sensations like pain and hunger.

\subsection{Kinesthesia}\label{subsec:kinesthesia}
%%%%%%%%%%%%%%%%%%%%%%%%%%%%%%%%%%%%%%%%%%%%%%%%%%%%%%%%%%%%%%%%%%%%%%%%%%%%%%%%%%%%%%%%%%%%%%%%%%%%%%%%%%%%%%%%%%%%%%%%

\Gls{kinesthesia} is the awareness of position and \textbf{movement} of body parts using sensory organs (proprioceptors, mechanosensory neurons) in muscles, tendons and joints.
It's crucial in muscle memory and hand-eye coordination.
Sometimes it's confused with proprioception, but those two are different concepts.
If you have an inner ear infection for example, and the sense of balance is affected, this would degrade the proprioceptive, but not the kinesthetic sense.
Moreover, proprioception is more about joint position (more subconscious cognitive awareness of your body in space and balance) whereas kinesthesia is more about awareness of joint movement (more conscious body's motion, behavioral).

\subsection{Neuronal Processing}\label{subsec:neuronal-processing}
%%%%%%%%%%%%%%%%%%%%%%%%%%%%%%%%%%%%%%%%%%%%%%%%%%%%%%%%%%%%%%%%%%%%%%%%%%%%%%%%%%%%%%%%%%%%%%%%%%%%%%%%%%%%%%%%%%%%%%%%

\textbf{Neuromuscular control} is the efferent (signal from the central nervous system to the body) response to an afferent (sensory) input, which is the functional component to movement and athletic activities that is referred to as dynamic stability.
Sensory input comes as (different types of\footnote{The four mechanoreceptors are: Meissner corpuscle for heavy pressure, Pacinian corpuscle for vibraiton, Merkel disks for light touch and Ruffini endings for skin stretch}) \textbf{mechanoreceptors} located in muscles, capsules and ligaments, allowing us awareness of joint position, movement, and acceleration.

All that information (vestibular, proprioceptive, kinesthetic) including the visual input is sent to the brain, where it is processed and integrated to allow us to create an overall representation of body position, movement and acceleration.
