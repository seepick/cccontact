\section{Notation}\label{sec:notation}

\subsection{Space Harmony}\label{subsec:space-harmony}
%%%%%%%%%%%%%%%%%%%%%%%%%%%%%%%%%%%%%%%%%%%%%%%%%%%%%%%%%%%%%%%%%%%%%%%%%%%%%%%%%%%%%%%%%%%%%%%%%%%%%%%%%%%%%%%%%%%%%%%%

This movement theory --and practice, also called \textit{Choreutics}-- was developed by the Austrian-Hungarian dancer and choreographer Rudolf Laban, to study the natural sequences of movements we follow in daily life; studying ``the art of movement'' to recognize spatial patterns.

When dancing, the term \textbf{\gls{kinesphere}} is used to refer to the space immediately reachable by your limbs without changing your place on the ground.
We can use up a lot of that space within this sphere (\textit{Far Reach Kinesphere}), just a bit (\textit{Near Reach Kinesphere}) or something in-between (\textit{Mid Reach Kinesphere}).

Furthermore, Mister Laban believed that there are three types of movers which prefer different \textbf{levels}: Those who enjoy leaping and springing off the ground move in \textit{High Level}, those with more sensuous movement enjoying the \textit{Central (Middle) Level}, and those who prefer more earth-bound movements who stayed in the \textit{Deep (Low) Level}.

Within the kinesphere we can move from one point to another through different approaches, so-called \textbf{pathways}: When movement is initiated from (or passes through) the body's center we take the \textit{Central Pathway}, along the outer limits of the kinesphere it takes a \textit{Peripheral Pathway}, and when the movement passes between center and periphery it takes a \textit{Transverse Pathway}.

There is of course much more to say about this, including the Laban Movement Analysis (LMA) which is a method and language for describing, visualizing, interpreting, and documenting movement, but this would go beyond the purpose of this book.
