\chapter{Technical}\label{ch:technical}

\chapterCoverImage{technical}

Techniques is what makes most of the people feel \textbf{impressed}.
Think of a child visiting a circus, watching the acrobats doing their tricks.
We will quickly explore the role of techniques (versus principles), and \textbf{conceptually} introduce them.
Most prominantly of course are \textbf{lifts}, the signature move of CI\@.
This is not meant to be a full listing with detailed descriptions of specific techniques.
For that, you want to visit the complementary website of this book.

\section{Small Dance}\label{sec:small-dance}

Recognizing and listening to the \gls{smalldance} is a starting exercise helping the practitioner to increase his \textbf{body awareness}.
It is usually done standing --and at the beginning, this was the only of doing it--, as it is the strongest way to balance due to the small surface.
Which position to take is not as important as the perception of reaction to the \textbf{process of balance}, which is always happening --except when completely lying down--, in any position.
Ultimately, you want to be able to figure out your own and also your partner's center, as lifts and basically, everything starts happening from there.

It could be considered as a form of \textbf{mindfulness} practice, where we focus our full attention to the sensation of standing; especially of the micro-movements in our ankles and whole body.
The process of how some automatic movements, little contractions and twitches, keeping us standing upright.
Something that is beyond our consciousness, but something we can definitely tap into by being more sensitive to it.
We can also use those unconscious micro-movements as a source of movement by amplifying it.

It is often used as a beginning of a grounding exercise, by shifting the weight, and keeping the center low.
Additionally, once the weight was totally shifted to one side, to ``double ground'' oneself to have an apparent sensation of stability and balance.

\section{Lifts}\label{sec:lifts}

The most common -- and most impressive -- ``spice'' added to a dance are lifts.
Whenever one center is located underneath another center, a lift can be easily performed by ``pouring one's weight''\footnote{Pouring weight is one of the core principles of CI, slowly and continuously increasing the pressure, instead of jumping on our partner with a potentially heavy impact and possible injury} onto the base.

There are different kinds of lifts, but the most typical ones are hip and shoulder-lifts.
When performing a \textbf{hip-lift}, the \gls{over-dancer} usually places his butt, lower belly or side on the lower back of the \gls{under-dancer}.
The \textbf{shoulder-lift} is the highest level of a lift where one ``flies'' on the shoulder of the other person.
Those lifts are usually done standing and also while being in the ``little animal'' position, and to change for example from a hip to a shoulder-lift, the over-dancer can spiral upwards the back defying gravity.

As a base, we need to ground ourselves to become more stable (and heavy) by focusing on our \gls{centergravity},
whereas the flyer, on the other hand, tries to make himself light and engage in his \gls{centerleviathan}.

Lifting might lead to potentially dangerous situations, which require us to pay special attention to the following safety rules:

\begin{itemize*}
    \item \textbf{Not grabbing} is a general rule, using less the hands and more the torso.
        With lifting this becomes even more important. Also \textbf{no interlocking} of the arms or grabbing the limbs of your flyer.
    \item Always keep a \textbf{hollow back} (a.k.a. ``\gls{goodgorilla}'') to provide a stable support for your partner instead of rounding your back, which makes him feel down and freak him out.
    \item Always keep your \textbf{head above ass}, otherwise your partner will most likely react with a fear response because of the danger of sliding down over your head (especially don't combine this with grabbing/interlocking).
\end{itemize*}

Read more about this in the~\nameref{ch:safety} chapter.

\section{Spirals}\label{sec:spirals}

We use a lot of spirally movement patterns in CI and pay special attention to how they are perceived, seen and anticipated in different movements in our own or someone else's body.
Moving in spirals is the perfect way to keep the pathway continuation which adds to a more enjoyable, fluid sensation during the dance.
Spiraling is considered a core movement pattern in CI and there is much more to say and experience about it.

Spirals can be very visibly be done between two body parts by moving from the distal parts of the body towards the more proximal parts.
There is a lot possible with playing with the axis, changing the axis, going within or outside the body, \ldots the limit is the imagination (and skill).
And ultimately spirals can also only by visualized, imagined, with pure intention/attention without any external visibility of movement.

\section{Negative Space}\label{sec:negative-space}

Dancing in the so-called \gls{negativespace} basically means entering, usually by reaching into it with a limb, the \gls{kinesphere} of another person without engaging in touch.
This is a common way to non-verbally invite someone for contact, to start a dance together, at a jam for example.

\section{Other Techniques}\label{sec:other-techniques}

\textbf{\Gls{body-surfing}} happens when one person rolls on the floor, and the other (usually) perpendicularly rolls over him.
Watch out if your partner is too heavy, to be able to release/distribute his weight, or simply drop your partner off in a gentle and polite way.
This usually evokes a lot of laughter due to the fun nature of it, its playfulness which reminds us of our childhood games we played.
For even more fun, consider doing a body-surf with more than two people and make some sort of ``body train''.

\textbf{Counter-balancing} is commonly used in partner-acrobatics, where the center is, for once, oppositional, instead of centers being shared.
We basically ``lean away'' from the partner, and by holding on to each other somehow, we balance each other out.

\section{Jargon}\label{sec:jargon}

% TODO \chapterCoverImage{jargon}

There are no universally acknowledged names for techniques or exercises, but different teachers try to invent, share, and give credit to their inventions.
Over time, an \textbf{oral tradition} has been established of passing information.
The following chapter explains language and images developed in the area ``I grew up in'' and are very \textbf{local} and not universal at all.
Yet they might inspire others, or at least provide some humorist, personal benefit for you while reading it.

\subsection{Purpose}\label{subsec:purpose}
%%%%%%%%%%%%%%%%%%%%%%%%%%%%%%%%%%%%%%%%%%%%%%%%%%%%%%%%%%%%%%%%%%%%%%%%%%%%%%%%%%%%%%%%%%%%%%%%%%%%%%%%%%%%%%%%%%%%%%%%

Introduction of jargon, using technical, domain-specific terms, is useful to express an idea, as otherwise many, many sentences would be needed whereas instead a single word can suffice as well.
The whole purpose of introducing and using jargon is not to smart-ass, to show off one's own superiority, an act of ego arrogance, or to exclude others by using an exclusive language.
It is, or at least should be, solely used to increase the information exchange and the information \textbf{density and precision}.

\subsection{General}\label{subsec:general}
%%%%%%%%%%%%%%%%%%%%%%%%%%%%%%%%%%%%%%%%%%%%%%%%%%%%%%%%%%%%%%%%%%%%%%%%%%%%%%%%%%%%%%%%%%%%%%%%%%%%%%%%%%%%%%%%%%%%%%%%

\noindent \textbf{\Gls{thebox}} -- This refers to the whole area of the upper body, the torso, including everything between the shoulders and hips.

\noindent \textbf{\Gls{smalldance}} -- The process of becoming aware of the subtle, unconscious micro movements of the body to maintain its balance.

\noindent \textbf{\Gls{kinesphere}} -- The space which can be reached by the body and its limbs with any kind of movement without making a step.

\vspace{5pt}

\noindent \ldots \textit{for further general terms, see the glossary at the end of this book} \ldots

\subsection{Onomatopoeia}\label{subsec:onomatopoeia}
%%%%%%%%%%%%%%%%%%%%%%%%%%%%%%%%%%%%%%%%%%%%%%%%%%%%%%%%%%%%%%%%%%%%%%%%%%%%%%%%%%%%%%%%%%%%%%%%%%%%%%%%%%%%%%%%%%%%%%%%

Sometimes, concepts are just much too fuzzy to be properly put into definite words.
Or there might have just no proper words yet been established, which made it necessary to develop our own few words, or even better: ``sound-terms'' (or onomatopoeia\footnote{Onomatopoeia, which has its Greek roots with the meaning of ``name-making'', which is the formation or use of words such as ``buzz'' or ``murmur'' that imitate the sounds associated with the objects or actions they refer to.}) to convey a certain meaning, quality, fuzzy principle or some abstract phenomena closer to what it meant to be:

\begin{itemize}
    \item \textbf{Botsen}: A conflict which arises when the teacher's instructions lead to a resistance based on an ``internal wisdom''; when the ``inner teacher'' and the ``outer teacher'' clash, we usually opt for the inner one and trust our gut feeling; e.g.\ Being told to jump and do a roll, but it doesn't feel safe and fear/resistance comes up, so you simply don't do it.
    (actually a Dutch word, meaning to clash, to collide; ``bot'' Dutch for bone; like two fists smashed on each other)
    \item \textbf{Hoopedy Di Poop}: Techniques which look overly impressive (good to show-off), yet not necessarily show much skill though, like jumping on each other; e.g.\ fancy technical, acrobatic stuff which usually comes with a higher risk for injury.
    \item \textbf{La La Land\footnote{Yes, ``\textit{La La Land}'' is also the title of a US musical romantic comedy-drama movie from 2016 with Ryan Gosling and Emma Stone.}}: Referring to anything which is more commonly used in the area of spirituality/religion, metaphysics, esoteric, new age talk and superstitions.
    Yet, sometimes we are still using concepts from those domains nevertheless due to the lack of any other better alternatives; e.g.\ think about when talking about ``reaching beyond the physical body'', which is of course technically not possible, but it gives the right image and intention.
    \item \textbf{Mooshy Mooshy}: (pronounced ``muschi muschi'') Indicating that during a dance, a too sensual, and thus inappropriate atmosphere arises, with no clear intentions; ranging from simply moving very slowly caressing down to \textit{cuddle puddles} where several people like cuddling on the ground.
    This can especially often be observed when starting with dancing on the ground, often preceded by a body-scan; things then often end up in some pile of bodies\footnote{Hopefully your cuddle puddles don't end up as it did in the movie ``\textit{Perfume: The Story of a Murderer}'' in the end.}.
    \item \textbf{Oomph}: (pronounced ``umpf'') The preferred quality of contact between two bodies which is characterized by properly giving/sharing weight, creating a sensation of considerable amount of pressure, opposed to only slight touch; feeling each other's groundedness.
    \item \textbf{Weee}: A melodic sound usually expressed in moments of heightened levels of alertness/fear, to down-regulate one's own nervous system, counteracting fear/panic, and allowing the body and people around to relax, release tension, and calm down; also to simply express joy about a movement, making everyone smile; e.g.\ when performing more risky lifts.
\end{itemize}

\subsection{Animals}\label{subsec:animals}
%%%%%%%%%%%%%%%%%%%%%%%%%%%%%%%%%%%%%%%%%%%%%%%%%%%%%%%%%%%%%%%%%%%%%%%%%%%%%%%%%%%%%%%%%%%%%%%%%%%%%%%%%%%%%%%%%%%%%%%%

Calling things by animal names is beneficial as they have a very visual characteristic, conveying a visceral experience, because of the qualities we associate with those animal.
We mostly use their names to refer to positions, techniques and ``qualities of body''.

% TODO good gorilla
\begin{itemize}
    \item \textbf{Bear}: Similar to the koala, but while sitting on the ground, and the bear hugging around the torso of his partner.
    \item \textbf{Banana}: Not really an animal, but anyway a useful metaphor where the arms and legs are stretched out long, shaping the whole body like a banana; it can be practiced on the ground (a way of movement on the ground, rolling sideways and only the core touching the floor), but applied often while being on the back of a partner to spiral upwards.
    \item \textbf{Chicken Wing}: Using a ``semi lock'' with the arm pit on the partner while being lifted, or similar; often necessary if the centers are not properly stacked and becoming unbalanced.
    \item \textbf{Chicken Leg}: Same as the chicken wing, but with the hip being flexed instead the elbow.
    \item \textbf{Crab}: The opposite of the octopus movement quality: Rigid, sharp, direct, and staccato.
    Think of the exoskeleton of the crab, when it walks sideways with its stiff limbs.
    \item \textbf{Koala}: When being lifted (or better: jumping on the partner's shoulder) hugging the upper body of the partner sideways and thus being very close to his center, while locking with your arms and legs; either around shoulders (more common) or pelvis (less common).
    It can be used as a preliminary technique for a shoulder lift.
    \item \textbf{(Little) Elephant}: Although we do like elephants, we don't like them in our studio, as their name is used to refer to steps/walking, or landing of the feet, which make a loud sound, indicating that there was no control and/or too much stiffness.
    Landing softly with no sound indicates control of movement, reversibility, precision and awareness, and also is a strong indicator of degree of safety with a partner.
    \item \textbf{Little Monkey}: As a little animal (a.k.a.: ``table-top'') but with knees lifted (a.k.a.: ``bear position'') with a light and fluent walking movement.
    Whenever we land silently, it is done so with control and elegance, which ultimately can prevent (serious) injuries.
    \item \textbf{Little Animal}: A table-top position on all fours, yet emphasizing a more dynamic, alive quality than a regular, wooden table.
    Also, when the painter is performing some movements, the little animal is actively supporting him by shaping his back accordingly, tilting and turning, flexing and extending.
    \item \textbf{Panda}: Similar to koala (and bear), but a more specific way of hugging, with belly to belly; often used while on the ground lying, keeping the centers connected all the time.
    \item \textbf{Snake}: Similar to octopus, but with a slightly different image to connect differently; having many movements in hands and spine going on simultaneously.
    \item \textbf{Octopus}: A movement quality which indicates aliveness/relaxation in all joints/body parts, each of them being controlled by their own intelligence; fluid, soft and smooth; opposite of the crab.
\end{itemize}

