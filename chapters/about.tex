\chapter{About}\label{ch:about}

\chapterCoverImage{about}

So what is this book about, and where is its emphasis?
For whom is it written, and what is the person's background who has written it?
Finally, I also want to provide some general remarks as a form of a disclaimer before you continue reading.

\section{Book}\label{sec:book}

\begin{displayquote}
    ``\textit{Alles was gelernt werden kann, ist nicht wert gelehrt zu werden.}'' -- my older brother
\end{displayquote}

% History; intention, motivation
The following pages were initially written for the sole purpose of taking \textbf{personal notes} of my own experience, thoughts and insights.
After some time, it started to grow bigger and bigger, until it reached a level where it could also be of use to others.
Interesting also, that most of the questions being asked from beginners are the same, which means it is easily possible to satisfy that curiosity with a predefined set of answers.
Over many years, collecting from those sources, a structure would emerge, and that's how this book got its shape and (physical) manifestation.

% Emphasis; goal, USP
Looking at the other -few- books about Contact Improvisation (or short: CI) on the market, it seemed to me there was something missing for me.
A \textbf{structured approach} to the technicalities, the rules and principles explained in details, using less of the imaginary language but something an engineering mind can better relate to.
Furthermore, I had the need to make the \textbf{implicit norms and rules} in a delicate, social context such as created during a CI event, explicit.
This would allow people reading this book to safely navigate through the space without bumping and crossing on or the other boundary, without crash-landing, neither physically nor psychologically.

% Target group
Whether you consider yourself a \textbf{beginner or advanced} practitioner, I feel certain that you'll find something of value to you.
Being interested in getting more acquainted with the \textbf{theoretical background} -- next to your regular practice in the studio -- of this fine art is the only prerequisite.

\subsection{Disclaimer}\label{subsec:disclaimer}

The book uses the \textbf{masculine version} when using examples for the sake of simplicity, in case both can be applied, and of course, always implying that the female version could have been equally used as well.

The book is \textbf{incomplete}, as any form of completeness can never be achieved anyway; we are not dealing with a hard science here.
I'd like to ask you to be gentle when you encounter an attempt of enumerating possibilities where you know that something is missing.
If you do so, please consider contacting me, contributing to this little project, and bringing it closer to completion and perfection, yet knowing we will never reach it, an asymptotic goal.

Whatever is written on the following pages does not claim to hold any absolute, objective truth.
Some parts are summaries of sources found along a path of exploration, whether they might be (direct or indirect) oral teachings, or written information found in books or on the internet.
Most of it is just a very \textbf{personal, subjective} collection of experiences, thoughts, associations, observations and opinions which were gathered along a single person's journey.
The content is, of course, also biased, although I made my best to free myself from my own limitations of reality-perception.
The content, thus, will sometimes be colored by my own personal background.
I shall be forgiven for my flawed, limited human nature.

\begin{displayquote}
    ``\textit{You are entitled to your opinion but you are not entitled to your own facts.}'' -- Daniel Patrick Moynihan
\end{displayquote}

It is said that there are no books existing written by a true master.
The reason being, that a master knows that it is impossible to capture the essence of a system in words, in a linear expression.
It happens sometimes, though, that students try to do that for the master, transcribing his teachings themselves.
I shall be forgiven for being such a student, desperate enough to achieve the unachievable in my own ignorance.

\section{Myself}\label{sec:myself}

To understand a system, we first need to understand its context and history.
If you want to understand psychoanalysis, for example, you first have to understand Sigmund Freud and the sociocultural environment he grew up in.
We are all just children of our time, and our views are shaped by it.
My own background will severely influence my approaches and values, ultimately also the content in this book.

\begin{displayquote}
    ``\textit{Good Kung-Fu looks bad, only bad Kung-Fu looks good.}'' -- my Aikido teacher
\end{displayquote}

My background is mainly rooted in the internal \textbf{martial arts} from Asia, and as such, my focus lies more on a practical approach (\textit{form follows function}).
For me, it is less about the aesthetics of movement, which might be more important for someone who is performing for others on stage, but more about the technical aspects.
``Right'' is what works in the most \textbf{pragmatic} way, meaning efficient in time and space, going into subjects like physics and biomechanics.
Whatever is within the principles of a system could also be considered is ``right'', relative to those principles, yet not necessarily ``better'' in every regard.

Besides those more physical aspects, the \textbf{psychological aspect} also has an important role to me.
The benefit for one's mental health, the ability to get to know oneself and others more deeply.
And, of course, a more philosophical/spiritual path which can also be walked with the help of this deep art.

As a body worker, I emphasize the importance of the non-verbal communication aspect which is happening while two people moving together.
The slowness, the gentleness in establishing a \textbf{deep connection} through listening, and the expressions of our personalities through this practice.

If you want to contact me, please feel free to do so by emailing me: \href{mailto:christoph@crashcoursecontact.org}{christoph@crashcoursecontact.org}
