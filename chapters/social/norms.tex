\section{Social Norms}\label{sec:social-norms}

\subsection{Etiquette}\label{subsec:etiquette}
%%%%%%%%%%%%%%%%%%%%%%%%%%%%%%%%%%%%%%%%%%%%%%%%%%%%%%%%%%%%%%%%%%%%%%%%%%%%%%%%%%%%%%%%%%%%%%%%%%%%%%%%%%%%%%%%%%%%%%%%

In every subculture, there are certain (social) \textbf{norms and rules} established which are not written down anywhere.
They somehow float in the minds of the people, the ``bigger body'', everyone carrying an incomplete piece in a slightly different variation, \textbf{implicitly} without any for of mentioning it, and sometimes even without being consciously aware of it.
Yet, if they are broken even just once, there will be consequences for such misbehavior executed by the group.

I, personally, find it useful to try to capture those and make them \textbf{explicit}, so we can avoid unintentional misbehavior and establish as much \textbf{harmony} as possible.
Just know, that every scene, maybe even every group around a specific teacher, has their very own set of normative rules, and that's why this list (as with any other attempt to ``list things'') is just a possible set of many.

\begin{itemize}
    \item \textbf{Talking} is considered to be a distraction during CI classes and especially during jams.
    If verbal communication is indeed needed, there is no obligation for total silence, but be mindful whether it is truly necessary in this moment, or whether it is merely some irrelevant chit-chat.
    In any case, try to keep it as short as possible and the volume low, to not distract your fellow practitioners.
    Otherwise, you can always step off the dance floor and speak properly a bit further away if really needed.
    \item Similarly to talking behavior, which disturbs others, please refrain from \textbf{parking} on the dance floor, meaning: Lying on the floor, maybe even with eyes closed, and being unaware of the surroundings and the other people.
    It is not only an annoyance, as it takes a lot of space (and attention), but it also adds another, yet unnecessary risk factor.
    \item The combination of both (talking and parking) among two or more people, also called \textbf{socializing}, is considered a no-go to do on the dance floor, for the reasons mentioned above.
    For that, please use the edges of the room, and if it's a smaller room, go outside, and if that's not available, just practice letting go and stay in the physical, present moment.
    \item \textbf{Stating your name} is just polite when dancing with someone new.
    At least do so at the very end, and also ask for the other person's name.
    It's just a nice way to acknowledge the person, and potentially useful to furthermore create a friendly bond.
    \item Although probably obvious, yet worth mentioning is \textbf{body hygiene}: Wear fresh clothes (and if you get sweaty easily and/or the temperature is high, also bring several shirts), make sure you have no overly intense bad breath (after having eaten a yummy ``Döner Kebab''), and have taken a shower before engaging in such an intimate movement form with other people.
    There is no need to smell like a perfume shop, as too intense scents are also uncomfortable.
    \item \textbf{Starting slowly} will help you to get to know each other, before \textit{rushing into someone's living room}.
    It's like a handshake, to get a glimpse of whom the other person is, where he is regarding skill and experience, personal preferences and pathways, and then take it from there, building it up, gaining trust, and going on an adventure together.
    \item \textbf{Eye gazing} is another wonderful tool to deeply connect with another human being, yet during CI, we refrain from doing so.
    In our world, it is a wonderful tool to scare people away, making them not liking to dance with you anymore.
    Hopefully, your partners will tell you that this is not the way we do it here.
    We use our peripheral vision to see, and usually don't constantly dance front-to-front.
    \item Speaking about front: It is totally fine to \textbf{roll over the front} side of your partner, and there is no need to avoid certain body parts, like breasts, genital area or similar.
    Once we can depersonalize the partner, seeing it as a physical object and letting go of the stories, there is no need to avoid certain body parts, and rather develop a more innocent and reality bound perspective.
    Yet, of course, we do not seek for it, and if possible, and not disturbing the dance and pathways too much, we try not to go directly for those parts.
\end{itemize}

If you participate in a class, there will be more \textbf{guidance}, and it will be easier for you to spot what's ``right and wrong'' behavior; in a \textbf{jam} though it might be a bit more difficult, especially if there is no hosting facilitator.
Take about 10 minutes at the beginning of a jam to sit, simply watch and observe the communal implicit norms.
Every jam is different, yet there is a common set of baseline rules as mentioned already above.

\subsection{Clothing}\label{subsec:clothing}
%%%%%%%%%%%%%%%%%%%%%%%%%%%%%%%%%%%%%%%%%%%%%%%%%%%%%%%%%%%%%%%%%%%%%%%%%%%%%%%%%%%%%%%%%%%%%%%%%%%%%%%%%%%%%%%%%%%%%%%%

Appropriate clothing is found in every setting where a group forms, sometimes for very rational, practical reasons, and sometimes just to establish some kind of esthetical norm to agree to, a so-called \textbf{dress code}.
If severely violated, it can even lead to the fact that people will feel opposed to dancing with you because of inappropriate clothing, and not because of you as a person!

As such, it is recommended to wear \textbf{long pants and long sleeves} to be able to slide on the ground and have enough friction, as well as sweat soaking capabilities.
Refrain from wearing clothes which limit your movements, like firm jeans, and also shirts (or pants) with buttons on them, as rolling on your partner with your full weight might hurt them, and also your pretty shirt might break; also no zippers for similar reasons.
In short: Normal, dancing, movable sports clothes.
And also plenty of them, especially shirts when the weather is warmer; the minority of us enjoy it to have a partner who is soaked in sweat and dance close with them.

In CI, we need some friction for \textbf{grip}, and for that matter, plastic fabrics will slide more (too much), and thus cotton is better.
Your beloved Adidas pants are not advised as they are too slippery; sorry.
Speaking of grip: It's better to dance \textbf{barefoot}, without socks to avoid slipping, or socks with those plastic nobs on the bottom if you got some of those.

The single utility though which could be useful would be knee pads, as there might be plenty of floor work and carrying people on your body while in the table-top position.
They shouldn't restrict your movement, so preferably choose some which are a bit thinner and made from an elastic material.

Obviously, \textbf{jewelry}, like big earrings, bracelets, rings, and basically any form of it, is not advised as they can get trapped on the other person's body and clothes, ultimately hurting someone.

Ultimately, we don't dress to impress, but rather prefer to wear our \textbf{cozy pyjamas}.
Newcomers often stand out at the beginning wearing overly sexual clothes, showing a lot of skin, or wearing their gym clothes.
All of it which isn't wrong by itself, not at all, it just might not be the right place for it.
You wouldn't wear your favorite onesie pyjama at your mother's funeral, either, would you?!
