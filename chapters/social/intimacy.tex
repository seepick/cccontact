\section{Intimacy}\label{sec:intimacy}

The psychological aspect of safety usually revolves around the topics of intimacy, sensuality and sexuality and how to set, perceive and respect \textbf{boundaries} of each other.
Take for example, when engaging in floor work or simply using the rolling point of contact over more intimate body parts, people --and especially women-- can quickly feel uncomfortable.
From the outside (imagine people who don't know the practice) this might seem very intimate, but for the more experienced practitioners, it's pure exploration and physics, an innocent and pure way of play.

Not only on the physical level, because of the proximity in touch, but also on the sensual/sexual level, CI can feel frightening for some people at times.
Just remember, this is not a \textbf{tantra practice} or similar, so sensuality is NOT (!) at CI's core; we don't engage in therapeutic energy work, don't activate Kundalini or seek for cuddle puddles.

Sensuality is a wonderful thing, but it is not really appropriate to go into ``melting'' into your partner, falling in love, or even sexuality, and expressing this via caressing touch, cuddles and somewhat ``lustful behavior'' during CI\@.
This border can be crossed at times, as the positive bonding effects of touch can easily lead us to engage into more.
At this moment, it is advised to \textbf{separate}, to ``cool down'' and re-engage with the dance, instead of the partner.
Caressing the skin of another person, to ``cuddle up'' and all of it is great, amazing, and please do more of it, but please refrain from doing so on the CI dance floor.

There is often a partner \textbf{bodywork} exercise at the end of a class/jam; a typical thing dancers would do.
It is meant to be a more technical, medical, or sports massage, rather than a personal, emotional, intimate touch session; don't confuse those.

If sensuality/sexuality still happens, the \textbf{teacher} will usually take that off very fast, or the jam facilitator, if present.
Cuddles, and cuddle puddles, will often happen and are being tolerated most of the time, yet they are not intentional!
If your intention is to engage in cuddles, the group will notice that behavior, and they might not want to dance with you anymore because of that.
There is of course no strict rule about it, no prohibition, and in some jams they totally tolerate it, even few invite it, but mostly it just doesn't happen.

In classes, it happens very, very rarely that certain individuals will look to satisfy their sexual needs by ``rubbing their horny body on someone else's''.
The teacher will usually here as well take care of a safe space for participants, and stop any form of sexual intended or predatory behavior right away, declaring it as a clear ``no-go''.
That's also why people should go first to a class to learn the \textbf{norms and values}, instead of dropping in a jam right away.
At jams, without a supervisor, no one knows how often such behavior might occur; maybe relatively more often?
Most of the time there is an organizer though, which can repeat clear rules and guidelines in an opening circle, minimizing the probability of such non-welcome behavior.
Comparing CI to Ecstatic Dance or Tantric Dance, in CI there are for sure way, way less ``horny people'' showing up, so you should be safe.

\subsection{Consent}\label{sec:consent}
%%%%%%%%%%%%%%%%%%%%%%%%%%%%%%%%%%%%%%%%%%%%%%%%%%%%%%%%%%%%%%%%%%%%%%%%%%%%%%%%%%%%%%%%%%%%%%%%%%%%%%%%%%%%%%%%%%%%%%%%

Consent is mainly about being able to express as well as hearing and appropriately responding to a ``yes'' and a ``no'', which are both possibly, and in the context of CI, preferably expressed non-verbally.

Of course, this starts already when engaging in a dance with someone, but is most importantly when it comes to more advanced techniques like lifts.
A ``no'' could be expressed due to lack of experience with and thus trust into each other, fear of being lifted, considering the other person's weight as too heavy, or other reasons.
There are many elegant ways out so that we can hold ourselves responsible for withstanding our \textbf{boundaries} instead of making others accountable for crossing them, ultimately ``self-dis-empowering'' ourselves.

Whenever a gentle hint by movement does not have the desired result, we can of course always cough, or if all of those subtle cues wouldn't work, simply speak and be very explicit about our wishes by saying assertively and gently with a smile: ``No''.
