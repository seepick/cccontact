\chapter{Mastery}\label{ch:mastery}

% TODO \chapterCoverImage{mastery}

\section{Becoming Good Yourself}\label{sec:becoming-good-yourself}

What makes a good CI practitioner good can be answered in many ways.
It is for sure not only the obviously physical/technical aspect, but also having the right intention for the dance, a proper attitude and mindset which is led by \textbf{curiosity} -- the antidote to being judgmental.
An advanced practitioner is pleased exploring the smallest thing, being able to keep a beginner's mind open, and he knows what he knows, and he especially also knows what he doesn't know.

For many, dancing with total beginners and total advanced is most enjoyable as they show and remind us about the different aspects of the dance.
Both of them are usually happy to dance with each other as well.
On the other hand, intermediates are usually not happy to dance with beginners; they prefer to only stay with their own level or higher.
They are also usually the ``dangerous ones'', as they know the pathways and the tricks, the form and the looks, but they don't know what they don't know.
They lack the listening ability.
If something happens which was not predicted in their pathways, like quick changes, they don't know how to handle that.
They also often go faster than their level of attention, and especially very often go faster than their beginner-partner's level of attention.
Having that said: Always \textbf{respect} what your body can do according to level, age, and constitution.

An essential tool you must gain to get good is the ability to \textbf{listen} to small and subtle sensations in the body, the change of movement and weight.
An increased awareness of incoming information, inside as well as outside the body via the peripheral nervous system, the so-called \gls{skinesphere}.
It leads us to be able to detect what's going on, thus becoming ``well-informed'' and leading us to become much better at CI.

More generally speaking, we can state that the following personality aspects lead to becoming good (those apply of course also to any other form or practice):

\begin{itemize*}
    \item [] \textbf{Kindness}: Smile, be gentle and soft, with yourself and others.
    \item [] \textbf{Patience}: Let beginners make their own mistakes, gaining experience.
    \item [] \textbf{Humbleness}: By not showing off how many years you are doing this-or-that, and not arrogantly teaching beginners.
\end{itemize*}

Everybody learns and everybody teaches; everybody is a student and teacher at the same time.
There are no certificates.
If you have a lot of knowledge, you will share that by your dancing.

Years of practice itself are not even necessarily a guarantee for expertise or even mastery.
Some people get stuck at a certain level; even after many, many years.
Lots of experience itself doesn't mean your level of technical practice is going forward.
Always keep a \textbf{beginner's mind}: Keep on doing the very same thing, like lifts and spirals, yet stay with curiosity and explore deeper, like the small things and changes, where to focus now, to put your attention and intention to.
Even if someone's technical level is very advanced: The moment where you stop posing questions, is where you stop developing.
