\chapter{Principles}\label{ch:principles}

\chapterCoverImage{principles}

Many different systems --like with sports, dances or martial arts-- are focused around dozens or even hundreds of \textbf{techniques} which are given to the student to learn by hard, including their names, and precise definitions of what's the right way of doing it.
This is an approach which might work for many instances, but obviously has some serious disadvantages when it comes to quickly responding (picking the right technique from many within a split of a second) and more importantly the ability for individual expression (using the system as an art-form to ``say the unsayable'').

In contrast to that, CI --along with many other systems as well-- is centered around a few core principles, and every technique which might be taught, studied and practiced is a manifestation of those core principles and are based upon them.
There are therefore no real moves to be learned, but more principles to be embodied and applied in any given moment, and adjusted to whatever needs are present while staying within the boundaries of the principles of that specific system.
Once they are well understood, you can free yourself from the limitations of specific techniques, and questions like whether something is ``right or wrong'' can be easily answered by asking your understanding.
Yet, as it is with the mastery of any art: Once the principles are fully understood, they can be broken if desired so, as: ``\textit{You can do whatever you want, as long as you know what you are doing.}''

\section{In Short}\label{sec:in-short}

As CI is an improvised partner dance --usually, but not necessarily done with two people/bodies--, it encourages the exploration together with the ground, while staying in constant physical contact.
The dance is supposed to move by itself, according to the participants aims and wishes.

In short, these are the basic principles used in CI, whereas the first two could be considered the physical essential ones and the other more about attitude and technique:

\begin{itemize}
    \item \textbf{Sharing Weight}
    \item \textbf{Rolling Point of Contact}
    \item Exploration of \textbf{Physical Forces}
    \item \textbf{Spirals} (and other related movement patterns)
\end{itemize}

Once all those principles are embodied, they will show up and surprise you when they happen and change your pathways.
They will also very much show up when in high velocity, when going into a risk engaged dance, dancing with a super high level of alertness and attention, jumping on each other, yet landing safely back on the ground.

\section{Grounding}\label{sec:grounding}

With grounding we are referring to some kind of sensation of (light) heaviness in the body, which makes the stance more \textbf{stable}, more robust and thus more connected to the ground.
Imaginary language like ``rooting'', and similar, are often used to describe this internal sensation, with its very realistic impact on the external.
This quality is the beginning of it all, without it, we can't go any further, as without a firm foundation, there is no house we can build upon.
To help to improve our groundedness we can use \textbf{visualizations} (roots growing into the ground), focusing our attention to where the sole of the feet have contact with the floor, breathing out and relaxing the muscular tension without collapsing in one's structure, and simply thinking about words which are associated with a grounded, firm, or stable quality.

It should not be confused with stiffness or rigidity, which so often lead to the illusion of groundedness and is achieved by simply contracting all muscles; something we don't want to do as it will remove the ability to adapt to the moment, remove our flexibility.

Lastly, because of the interconnection between \textbf{body-mind}, the fact that you become a more grounded mover physically, you also become a more grounded person mentally (emotionally stable, more resilient, durable).

\section{Small Dance}\label{sec:small-dance}

Recognizing and listening to the \gls{smalldance} is a starting exercise helping the practitioner to increase his \textbf{body awareness}.
It is usually done standing --and at the beginning, this was the only of doing it--, as it is the strongest way to balance due to the small surface.
Which position to take is not as important as the perception of reaction to the \textbf{process of balance}, which is always happening --except when completely lying down--, in any position.
Ultimately, you want to be able to figure out your own and also your partner's center, as lifts and basically, everything starts happening from there.

It could be considered as a form of \textbf{mindfulness} practice, where we focus our full attention to the sensation of standing; especially of the micro-movements in our ankles and whole body.
The process of how some automatic movements, little contractions and twitches, keeping us standing upright.
Something that is beyond our consciousness, but something we can definitely tap into by being more sensitive to it.
We can also use those unconscious micro-movements as a source of movement by amplifying it.

It is often used as a beginning of a grounding exercise, by shifting the weight, and keeping the center low.
Additionally, once the weight was totally shifted to one side, to ``double ground'' oneself to have an apparent sensation of stability and balance.

\section{Pouring Weight}\label{sec:pouring-weight}

Once we have succeeded to ``gain some weight by grounding'', we can use that to pour it into another person's body.
The emphasis here is to \textbf{slowly and gradually} increase the amount of pressure where the bodies have contact, instead of a quick and sudden shift, which will be difficult and fear evoking movement for your partner; ultimately even dangerous.
Instead, we want to ``announce'' that there is some weight approaching so that our partner can adjust and adapt posture and internal tension/structure to that poured weight.

\section{Sharing Weight}\label{sec:sharing-weight}

The first and most important principle is trying to seek a deep --core-- connection between two bodies, something we call in our classes as an ``\textit{oomph}''-quality.
The body is \textbf{stable and grounded}, yet its limbs and joints are \textbf{soft and relaxed}; \textit{like an iron stick wrapped in cotton wool}.
Also, the contact is primarily on \gls{thebox}, the upper body, and less on the arms and legs.
It is different from actively pushing with muscular force, and also slightly different from leaning, by which you shift your center of gravity beyond a point of no-return.

The contact should lead to a sensation of the \textbf{ground underneath} your partner's feet, passing through their center, originating from the single point of contact which can be even as far from the ground as a hand.
We constantly try to search for the \gls{centergravity} of the other person's body, which might sound familiar to people with a background in martial arts such as Taijiquan.
This is also called a \textit{contact quality}, a result of grounding plus sharing weight, instead of a simple \textit{touch quality}, a soft feather stroking like a butterfly.
The ultimate goal is to maintain this quality throughout (almost) all time, and therefore also leading to acquiring the skill of \textbf{recognizing weight}.

Unfortunately, this is also usually one of the \textbf{most difficult} skills to acquire for beginners.
Reasons could be such as fear of falling, fear of imposing one's own weight on another person, being a burden and thoughts popup like: ``\textit{Am I too much? Am I too heavy for the other person?}''
A handshake or a tap on the shoulder is common in our society, leaning on someone not.
It's important to learn this principle, yet without stopping to breathe and without tensing up, which can be a massive struggle, especially for beginners.
Another very scary aspect for many people is when going to the ground.
Having lots of weight on you or giving (lots of) weight to that person on the ground is something which needs to be trained.
We don't only share our weight, but we are also sharing our balance, in order not to be off-balanced but ``\textbf{shared-balanced}''.

The more advanced you get, the more you have already acquired that skill, the more often you can choose to deliberately not give weight in the right moment.

\section{Rolling Point of Contact}\label{sec:rolling-point-of-contact}

We use spiraling and rotating movement patterns to always \textbf{maintain contact} and amount of pressure in a rolling fashion, instead of sliding or even jumping the point of contact, e.g. directly from a hand to a shoulder, not passing ``through'' the whole arm.
Sliding or jumping (point of contact) is by no means wrong, but it is added later on deliberately by more advanced practitioners.

The contact will thereby follow a \textbf{predictable trajectory}, a pathway, which means both partners can anticipate the very next movement, which furthermore leads to a more ``fluid sensation'' in the dance.
There is no disengaging or re-engaging of the point of contact (at least not at the beginning), which sometimes can feel like little bumps during the dance, breaking this fluid sensation.
For this to happen, it is required to have a more agile body, bulging out body parts and bending/flexing wherever necessary to keep a clear rolling point of contact; something we refer to as an ``octopus''-quality.
It is also used to correct each other to find balance, to readjust and realign.

This is the second most important and also the second most often principle with which beginners struggle with.
Be aware that we don't aim for \textbf{intimate body-parts} (buttocks, breasts, genitals), yet we roll over them ``coincidentally'' without staying there.
For example, when my head would be at those parts, I will still try to avoid that if possible, even if we know each other very well.
We try to desexualize the human body --not the partner, but the body--, something which can be indeed difficult in our hyper-sexualized society.
The touch should be without any sexual intention, even if intimate parts are being rolled over, which will make us both feel comfortable; the ``how'' is more important than the ``what''.
We play with each other, just like kids are playing ``rough and tumble'' in the playground, before encountering the wonderful world of sexuality.
Just like a doctor touching breast tissue intending to find signs of cancer and without any sexual intention or gain some personal advantage; that's why it feels safe for the patient to receive that touch.

\section{Pathway Continuation}\label{sec:pathway-continuation}

To be aligned with the physical law of \textbf{inertia}, we should never break an already moving momentum; exceptions again made for those who have mastered it.
Once spiraling in one direction it should be maintained; it opens up the possibility for anticipation, predictability and therefore trust on a psychological level, but also a mere reason of energy efficiency on the physical level.

\section{Movement Patterns}\label{sec:movement-patterns}

Through a heightened awareness of communication through movement, touch, and sharing weight, we explore the space and the connection between, through mutual physical cooperation.
Fundamental movement patterns are:

\begin{itemize}
    \item [] \textbf{Yielding}: softening/surrendering into incoming force or to gravity
    \item [] \textbf{Pushing}: expansion, taking up space
    \item [] \textbf{Reaching}: extending physically or meta-physically
    \item [] \textbf{Pulling}: contraction, until collapsing
    \item [] \textbf{Releasing}: relaxing into what's contracted before
\end{itemize}

All of those movements can be done easily with little muscular effort if basic physical forces are acknowledged and taken advantage of, such as gravity (falling), momentum, inertia, balancing, etc.
And all of those while staying in contact.

\section{Relaxation}\label{sec:relaxation}

We usually move rather relaxed; a body which is ready for action yet open for receiving tactile stimulus, open for information.
We try to achieve that by deep breathing, by keeping a fluid movement quality (``octopus''-quality) and also avoid a staring eye gaze.

Yet, a relaxed state should not be confused with a collapsed one.
An active state is also not the same as a hyper-tensed one.
Within this spectrum of non-extremes, we ought to find the optimal amount of \textbf{muscle tone} which is appropriate for a given situation.
