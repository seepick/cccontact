\chapter{Principles}\label{ch:principles}

\chapterCoverImage{principles}

This chapter is maybe the most important one in the whole book.
It captures the \textbf{essence} of CI, as without the principles it is not it anymore.
We will have a look at the few core principles, and the requirements to be able to apply them.
We will also further include some side principles and \textbf{movement patterns} usually mentioned in the same sentence.

\section{Grammar versus Vocabulary}\label{sec:grammar-versus-vocabulary}

Many different systems --like with sports, dances or martial arts-- are focused around dozens or even hundreds of \textbf{techniques}, or call them words/vocabulary.
They are given to the student to learn by hard, including their names, and a precise definitions of what's the right way of doing it.
This is an approach that might work for many instances, but obviously has some serious disadvantages.
Think of when it comes to quickly responding, picking the right technique from many within a split of a second.
And more importantly is the ability for individual expression: using the system as an art-form to ``say the unsayable''.

In contrast to that, CI --along with many other systems as well-- is centered around a few principles, or call them the grammar.
Every technique that might be taught, studied and practiced, is a manifestation of those core principles.
There are no real moves to be learned, but more principles to be embodied and applied in any given moment.
They need to be adjusted to whatever needs are present while staying within the boundaries of the system.
Once they are well understood, you can free yourself from the limitations of specific techniques, and questions like whether something is ``right or wrong'' can be easily answered by asking your deeper understanding.
Yet, as it is with the mastery of any art: Once the principles are fully understood, they can be broken if desired so, as: ``\textit{You can do whatever you want, as long as you know what you are doing.}''
% TODO move this last sentence into the motto chapter

When you meet CI people from another culture, they might speak a different dialect, even using foreign words unknown to you.
But as long as you both use the same grammar, you will still understand and be able to interact with each other.

\section{In Short}\label{sec:in-short}

As CI is an improvised partner dance --usually, but not necessarily done with two people/bodies--, it encourages the exploration together with the ground, while staying in constant physical contact.
The dance is supposed to move by itself, according to the participants aims and wishes.

In short, these are the basic principles used in CI, whereas the first two could be considered the physical essential ones and the other more about attitude and technique:

\begin{itemize}
    \setlength\itemsep{0em}
    \item \textbf{Sharing Weight}
    \item \textbf{Rolling Point of Contact}
    \item Exploration of \textbf{Physical Forces}
    \item \textbf{Spirals} - and other related movement patterns
\end{itemize}

Once all those principles are embodied, they will show up and surprise you when they happen and change your pathways.
They will also very much show up when in high velocity, when going into a risk engaged dance, dancing with a super high level of alertness and attention, jumping on each other, yet landing safely back on the ground.

\section{Grounding}\label{sec:grounding}

With grounding we are referring to some kind of sensation of (light) heaviness in the body, which makes the stance more \textbf{stable}, more robust and thus more connected to the ground.
Imaginary language like ``rooting'', and similar, are often used to describe this internal sensation, with its very realistic impact on the external.
This quality is the beginning of it all, without it, we can't go any further, as without a firm foundation, there is no house we can build upon.
To help to improve our groundedness we can use \textbf{visualizations} (roots growing into the ground), focusing our attention to where the sole of the feet have contact with the floor, breathing out and relaxing the muscular tension without collapsing in one's structure, and simply thinking about words which are associated with a grounded, firm, or stable quality.

It should not be confused with stiffness or rigidity, which so often lead to the illusion of groundedness and is achieved by simply contracting all muscles; something we don't want to do as it will remove the ability to adapt to the moment, remove our flexibility.

Lastly, because of the interconnection between \textbf{body-mind}, the fact that you become a more grounded mover physically, you also become a more grounded person mentally (emotionally stable, more resilient, durable).

\section{Small Dance}\label{sec:small-dance}

Recognizing and listening to the \gls{smalldance} is a starting exercise helping the practitioner to increase his \textbf{body awareness}.
It is usually done standing --and at the beginning, this was the only of doing it--, as it is the strongest way to balance due to the small surface.
Which position to take is not as important as the perception of reaction to the \textbf{process of balance}, which is always happening --except when completely lying down--, in any position.
Ultimately, you want to be able to figure out your own and also your partner's center, as lifts and basically, everything starts happening from there.

It could be considered as a form of \textbf{mindfulness} practice, where we focus our full attention to the sensation of standing; especially of the micro-movements in our ankles and whole body.
The process of how some automatic movements, little contractions and twitches, keeping us standing upright.
Something that is beyond our consciousness, but something we can definitely tap into by being more sensitive to it.
We can also use those unconscious micro-movements as a source of movement by amplifying it.

It is often used as a beginning of a grounding exercise, by shifting the weight, and keeping the center low.
Additionally, once the weight was totally shifted to one side, to ``double ground'' oneself to have an apparent sensation of stability and balance.

\section{Pouring Weight}\label{sec:pouring-weight}

Once we have succeeded to ``gain some weight by grounding'', we can use that to pour it into another person's body.
The emphasis here is to \textbf{slowly and gradually} increase the amount of pressure where the bodies have contact, instead of a quick and sudden shift, which will be difficult and fear evoking movement for your partner; ultimately even dangerous.
Instead, we want to ``announce'' that there is some weight approaching so that our partner can adjust and adapt posture and internal tension/structure to that poured weight.

\section{Sharing Weight}\label{sec:sharing-weight}

The first and most important principle is trying to seek a deep --core-- connection between two bodies, something we call in our classes as an ``\textit{oomph}''-quality.
The body is \textbf{stable and grounded}, yet its limbs and joints are \textbf{soft and relaxed}; \textit{like an iron stick wrapped in cotton wool}.
Also, the contact is primarily on \gls{thebox}, the upper body, and less on the arms and legs.
It is different from actively pushing with muscular force, and also slightly different from leaning, by which you shift your center of gravity beyond a point of no-return.

The contact should lead to a sensation of the \textbf{ground underneath} your partner's feet, passing through their center, originating from the single point of contact which can be even as far from the ground as a hand.
We constantly try to search for the \gls{centergravity} of the other person's body, which might sound familiar to people with a background in martial arts such as Taijiquan.
This is also called a \textit{contact quality}, a result of grounding plus sharing weight, instead of a simple \textit{touch quality}, a soft feather stroking like a butterfly.
The ultimate goal is to maintain this quality throughout (almost) all time, and therefore also leading to acquiring the skill of \textbf{recognizing weight}.

Unfortunately, this is also usually one of the \textbf{most difficult} skills to acquire for beginners.
Reasons could be such as fear of falling, fear of imposing one's own weight on another person, being a burden and thoughts popup like: ``\textit{Am I too much? Am I too heavy for the other person?}''
A handshake or a tap on the shoulder is common in our society, leaning on someone not.
It's important to learn this principle, yet without stopping to breathe and without tensing up, which can be a massive struggle, especially for beginners.
Another very scary aspect for many people is when going to the ground.
Having lots of weight on you or giving (lots of) weight to that person on the ground is something which needs to be trained.
We don't only share our weight, but we are also sharing our balance, in order not to be off-balanced but ``\textbf{shared-balanced}''.

The more advanced you get, the more you have already acquired that skill, the more often you can choose to deliberately not give weight in the right moment.

\section{Rolling Point of Contact}\label{sec:rolling-point-of-contact}

We use spiraling and rotating movement patterns to always \textbf{maintain contact} and amount of pressure in a rolling fashion, instead of sliding or even jumping the point of contact, e.g. directly from a hand to a shoulder, not passing ``through'' the whole arm.
Sliding or jumping (point of contact) is by no means wrong, but it is added later on deliberately by more advanced practitioners.

The contact will thereby follow a \textbf{predictable trajectory}, a pathway, which means both partners can anticipate the very next movement, which furthermore leads to a more ``fluid sensation'' in the dance.
There is no disengaging or re-engaging of the point of contact (at least not at the beginning), which sometimes can feel like little bumps during the dance, breaking this fluid sensation.
For this to happen, it is required to have a more agile body, bulging out body parts and bending/flexing wherever necessary to keep a clear rolling point of contact; something we refer to as an ``octopus''-quality.
It is also used to correct each other to find balance, to readjust and realign.

This is the second most important and also the second most often principle with which beginners struggle with.
Be aware that we don't aim for \textbf{intimate body-parts} (buttocks, breasts, genitals), yet we roll over them ``coincidentally'' without staying there.
For example, when my head would be at those parts, I will still try to avoid that if possible, even if we know each other very well.
We try to desexualize the human body --not the partner, but the body--, something which can be indeed difficult in our hyper-sexualized society.
The touch should be without any sexual intention, even if intimate parts are being rolled over, which will make us both feel comfortable; the ``how'' is more important than the ``what''.
We play with each other, just like kids are playing ``rough and tumble'' in the playground, before encountering the wonderful world of sexuality.
Just like a doctor touching breast tissue intending to find signs of cancer and without any sexual intention or gain some personal advantage; that's why it feels safe for the patient to receive that touch.

\section{Pathway Continuation}\label{sec:pathway-continuation}

To be aligned with the physical law of \textbf{inertia}, we should never break an already moving momentum; exceptions again made for those who have mastered it.
Once spiraling in one direction it should be maintained; it opens up the possibility for anticipation, predictability and therefore trust on a psychological level, but also a mere reason of energy efficiency on the physical level.

\section{Movement Patterns}\label{sec:movement-patterns}

Through a heightened awareness of communication through movement, touch, and sharing weight, we explore the space and the connection between, through mutual physical cooperation.
Fundamental movement patterns are:

\begin{itemize}
    \setlength\itemsep{0em}
    \item [] \textbf{Yielding}: softening/surrendering into incoming force or to gravity
    \item [] \textbf{Pushing}: expansion, taking up space
    \item [] \textbf{Reaching}: extending physically or meta-physically
    \item [] \textbf{Pulling}: contraction, until collapsing
    \item [] \textbf{Releasing}: relaxing into what's contracted before
\end{itemize}

All of those movements can be done easily with little muscular effort if basic physical forces are acknowledged and taken advantage of, such as gravity (falling), momentum, inertia, balancing, etc.
And all of those while staying in contact.

\section{Relaxation}\label{sec:relaxation}

We usually move rather relaxed; a body which is ready for action yet open for receiving tactile stimulus, open for information.
We try to achieve that by deep breathing, by keeping a fluid movement quality (``octopus''-quality) and also avoid a staring eye gaze.

Yet, a relaxed state should not be confused with a collapsed one.
An active state is also not the same as a hyper-tensed one.
Within this spectrum of non-extremes, we ought to find the optimal amount of \textbf{muscle tone} which is appropriate for a given situation.

\section{Mottos}\label{sec:mottos}

Mottos are like universal \textbf{guidelines}; like a line that helps us to keep track in case we need it when we are lost.
They are not rules, but there are more like \textbf{invitations}: There will be no consequences once they are not adhered.
And it is up to us to face to them when we need them, and they don't come to us once we disregard them.
They help us improve our technique, implement the principles, and embody a quality more conforming to the core of CI\@.

%You could also call them --universal-- guidelines, which would be semantically more specific than principles, yet not as specific as concrete techniques.
%They are kept short, so we can remember them quickly, and can act as part of our vocabulary.
%Once we mention it to someone who is also familiar with that saying, we immediately have a common understanding within a very short period of time; something that is identical to the benefit of introducing a~\nameref{ch:jargon}.

\newcommand{\motto}[1]{
    \vspace{5pt}
    \begin{adjustwidth}{30pt}{30pt}
    \begin{center}
    \textit{#1}
    \end{center}
    \end{adjustwidth}
}

\motto{Tension masks sensation.}

Image your nerves like little pipes running throughout your whole body.
Now imagine whenever you tense up, your muscles contract which surround those ``nerve pipes''.
By contracting so much, they start to squeeze the nerves so that they can't transmit any information anymore; they are blocked.

The more \textbf{relaxed} we are, the more sensitive our skin is to touch, pressure and weight.
Of course, we want to relax, yet not totally collapse; any extreme is to be avoided.
We want to engage with the minimum effort possible to stay stable; as little as possible, as much as necessary.

\motto{Keep on breathing.}
When getting tensed up, physically and/or emotionally, we usually have the tendency to automatically hold our breath.
This is rather counterproductive for staying sharp, focused and relaxed.
Instead, we continuously try to remind ourselves to breathe.
Especially emphasizing the (deep) out-breath, as the exhalation activates our parasympathetic nervous system.
Do that through your mouth, relaxing the jaw, and don't suppress, but also don't force, to make a sound.

Breathe out when you notice that your partner becomes stiff and holds his breath.
It is an effective way to non-verbally co-regulate your partner into taking a big sigh as well.
It's a little bit like with the phenomena of yawning: Due to our social nature, equipped with mirror neurons, those behaviors tend to be contagious, and we can't help it but imitate it.

\motto{We try not to fall in love with our partner,\\we fall in love with the dance.}
Try to see your partner as a mere physical object and by that explore the physical realm instead of the psychological, the interpersonal one.
It's also not a personal dance, it's a physicality; the experience is because of the practice, not because of your partner necessarily; don't attach that to a specific person, just like the teachings of Tantra tell us as well.
Even when you had a remarkable dance with someone, once the dance is over, I'm going to say ``Thank you, and bye bye''.
It is nevertheless possible, of course, to talk to that person later on, but please don't linger and dance the entire class or jam with that single person.

\motto{Keep eyes open and ``wide''.}
Sometimes people tend to close their eyes, or focus them constantly, fierce-fully on their partner.
Instead, we want to keep an open gaze, perceiving everything around us, staying in connection with all the people in the room and the room itself.
Once we start to gaze at the floor, this is usually an indication of a state of hyper-focus, which potentially closes our perception.
Remind yourself to keep your vision leveled, stay aware and attentive.

\motto{Dance at the edge of your level of attention,\\and don't cross the level of your partner's attention.}
Try to \textbf{expand} what you can do regarding movement, attention, speed, techniques, and pathways.
And also vital, listen to what your partner is able as well, and \textbf{respect} that with patience and compassion.

\subsection{Inspirational Mistakes}\label{subsec:inspirational-mistakes}

As in general with any improvised art, mistakes are only seen as such, as soon as we declare them as being mistakes.

Imagine two improvisation actors on stage.
One says ``mouse'', the other says ``house''.
And then again, the same thing: house, then mouse.
They actually intended to go on with different rhyming words, but for whatever reason (too nervous?) they are stuck and can't come up with something new, and because they can fake it as a deliberate decision (not admitting it being a mistake, something which was not their initial desired goal; when things don't go according to a fixed plan), people in the audience might be amazed by the ``post-modern acting skills'' and interpret something into it which is actually not there.

Once you can let go of any plan, and be truly in the \textbf{present moment} with whatever is happening; once you can fully comprehend that whenever there is another person involved, any desire for control is futile \ldots then you will be able to surrender and use any happening as a source of inspiration.
Ultimately, being able to \textbf{surprise yourself}, and be fascinated by what happened to you once you let go.

