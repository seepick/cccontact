\section{Principles}\label{sec:principles}

\begin{wrapfigure}{R}{0.3\textwidth}
    \centering
    \includegraphics[width=0.25\textwidth]{images/principles}
\end{wrapfigure}

Many systems (sports, martial arts, \ldots) are focused around dozens or even hundreds of \textbf{techniques} which are given to the student to learn by hard, including their names, and precise definitions of what's the right way of doing it.
This is an approach which might work for many instances, but obviously has some serious disadvantages when it comes to quickly responding (picking the right technique from many within a split of a second) and more importantly the ability for individual expression.

In contrast to that, CI (along with many other sports/martial arts systems as well) is centered around a few core \textbf{principles}, and every technique which might be taught, studied and practiced is a manifestation of those core principles.
The techniques present are based on the principles, the principles are the core.
There are therefore no real moves to be learned, but more principles to be embodied and applied in any given moment.
Once the principles are well understood, one can free oneself from the limitations of specific techniques, and questions like whether something is ``right or wrong'' can be easily answered by asking those principles.
Yet, as it is with the mastery of any art: Once the principles are fully understood, they can be broken if desired so, as: ``\textit{You can do whatever you want, as long as you know what you are doing.}''

\subsection{In Short}\label{subsec:in-short}

CI being an improvised partner dance (usually, but not necessarily done with two people/bodies), it encourages the exploration together with the ground, while staying in constant physical contact.
The dance is supposed to move by itself, according to the participants aims and wishes.

In short, these are the basic principles used in CI, whereas the first two could be considered the physical essential ones:

\begin{itemize}
    \item Sharing Weight
    \item Rolling Point of Contact
    \item Exploration of physical forces
    \item Spirals and other related movement patterns
\end{itemize}

Later, once all those principles are embodied, they show up and will surprise you when they happen and change your pathways.
Also in high velocity, when going into a risk engaged dance, dancing with a super, super high level of alertness and attention, jumping on each other, yet landing safely back on the ground.

\subsection{Grounding}\label{subsec:grounding}

With grounding we are referring to some kind of sensation (light) heaviness in the body, which makes the stance more stable, more robust and thus more connected to the ground.
Imaginary language like ``rooting'', and similar, are often used to describe this internal sensation, with its very realistic impact on the external.
This quality is the beginning of it all, without it, we can't go any further, as without a firm foundation there is no house we can build upon it.
To help improving our groundedness we can use visualizations (roots growing into the ground), focusing our attention to where the sole's of the feet have contact with the floor, breathing out and relaxing the muscular tension without collapsing in one's structure, and simply thinking about words which are associated with a grounded, firm, or stable quality.

It should not be confused with stiffness, which so often lead to the illusion of groundedness, which is achieved by simply contracting all muscles; something we don't want to do as it will remove the ability to adapt at the moment, our flexibility.

Lastly, because of the interconnection between body-mind, the fact that one becomes a more grounded dancer, one also becomes a more grounded person.

\subsubsection{Small Dance}

Recognizing and listening to the \gls{smalldance} is a (warming up) exercise helps the practitioner to increase one's body awareness.
It is classically done only standing, as it is the strongest way to balance due to the small surface, that's how it started.
Which position to take is not as important as the perception of reaction to the process of balance, which is always happening (except when completely lying down), in any position.
Ultimately we want to be able to figure out one's own and also your partner's center, as lifts are happening from there.

It could be considered as some form of mindfulness practice, where we focus our full attention to the sensation of standing; especially of the micro movements in our ankles and whole body.
How some automatic movements, little contractions and twitches, keeping us standing upright.
Something that is beyond our consciousness, but something we can definitely tap into by being more sensitive to it.
We can use these unconscious micro movements as a source of movement by amplifying it.

It is also often used as a beginning of a grounding exercise, by shifting the weight, and keeping the center low.
Additionally, once the weight was totally shifted to one side, to ``double ground'' oneself to have a very clear sensation of stability and balance.

\subsection{Pouring Weight}\label{subsec:pouring-weight}

Once we have established to ``gain some weight'' by grounding, we can use that to pour it into another person's body.
The emphasis here is to slowly increase the amount of pressure where the body's have contact, instead of a quick and sudden shift, which will be difficult and fear evoking movement for your partner; ultimately even potentially dangerous.
Instead, we want to ``announce'' that there is some weight approaching, so that our partner can adjust and adapt posture and internal tension/structure to that poured weight.

\subsection{Sharing Weight}\label{subsec:sharing-weight}

The first and most important principle is trying to seek a deep (core) connection between two bodies, sometimes also called ``\textit{umpf}'' in our classes.
The body is stable and grounded, yet its limbs and joints are soft and relaxed; like an iron stick wrapped in cotton wool.
Also the contact is primarily on \gls{thebox}, the upper body, and less on the arms and legs.
It is different from actively pushing with muscular force, and also slightly different from leaning by which one shifts one's center of gravity beyond a point of no-return.

The sensation should lead to a feeling of the ground beneath the partner's feet, right through their center, solely through the single point of contact which can be even as far from the ground as a hand.
We constantly try to search for the \gls{centergravity} of the other person's body, which might sound familiar experienced Taijiquan practitioners where the given aim is identical.
This is also called a \textit{contact quality} (a result of grounding plus sharing weight) instead of a simple \textit{touch quality} (soft feather stroke like a butterfly).
The ultimate goal is to maintain this quality throughout (almost) all time, and therefor also leading to acquiring the skill of recognizing weight.

Unfortunately, this is also usually one of the most difficult skills to acquire for beginners.
Reasons could be such as fear of falling, fear of imposing one's own weight on another person (``Am I too much/heavy?'', being a burden).
A handshake or a tap on the shoulder is common in our society, to lean on someone not.
It's important to learn this principle, yet without stopping to breathe and without tensing up, which is a very big struggle for beginners.

Another very scary aspect for many people is when going to the ground.
Having lots of weight on you or giving (lots of) weight to that person on the ground is something which need to be trained.

We don't only share our weight, but we are also sharing our balance, in order not to be ``off-balanced'' but ``shared-balanced''.

The more advanced people, who acquired that skill, can choose to deliberately not give weight (or jump contact-points).

\subsection{Rolling Point of Contact}\label{subsec:rolling-point-of-contact}

We use spiraling and rotating movement pattern to always maintain contact (and amount of pressure) in a rolling fashion, instead of sliding of even jumping the point of contact, e.g. directly from a hand to a shoulder, not passing ``through'' the whole arm.
Sliding or jumping is by no means wrong, but it is added later on deliberately by more advanced dancers.

The contact follows a predictable trajectory, a pathway, which means both partners can anticipate the very next movement, which furthermore leads to a more ``fluid sensation'' in the dance.
There is no disengaging or re-engaging of the point of contact (at least not at the beginning), which sometimes can feel like little bumps during the dance, breaking this fluid sensation.
For this to happen, it is required to have a more agile body, bulging out body parts and bending/flexing wherever necessary to keep a clear rolling point of contact.
It is also used to correct each other to find balance, to readjust and realign.

The second most important and also second most often principle with which beginners struggle with.
Be aware that we don't aim for ``intimate body-parts'' (buttocks, breasts, genitals), yet we roll over them ``coincidentally'' without staying there.
For example when my head would be at those parts, I will still try to avoid that if possible, even if we know each other very well.
We try to desexualize the human body (not the partner, but the body), someting which can be difficult in our hyper-sexualized society.
The touch should be without any sexual intention, even if intimate parts are being rolled over (the ``how'' is more important than the ``what'') which will make us both feel comfortable.
We play with each other, just like kids are playing ``rough and tumble'' in the playground;
before encountering the wonderful world of sexuality.
Just like a doctor touching breast tissue with the intention of finding signs of cancer and without any sexual intention;
that's why it feels safe for the patient.

\subsection{Pathway Continuation}\label{subsec:pathway-continuation}

According to the physical law of inertia, and to be in accordance with it, we should never break an already moving momentum (exceptions for the master applied here).
Once spiraling in one direction it should be maintained; it opens up possibility for anticipation, predictability and therefore trust on a psychological level, but also a mere reason of energy efficiency on the physical level.

\subsection{Movement Patterns}\label{subsec:movement-patterns}

Through a heightened awareness of communication through movement, touch and sharing weight, we explore the space and the connection between through mutual physical cooperation.
Fundamental movement patterns are:

\begin{itemize}
    \item \textbf{Yielding}: softening/surrendering into incoming force or to gravity
    \item \textbf{Pushing}: expansion, taking up space
    \item \textbf{Reaching}: extending physically or meta-physically
    \item \textbf{Pulling}: contraction, up til collapsing
    \item \textbf{Releasing}: relaxing into what's contracted before
\end{itemize}

All of those movements can be done easily with little muscular effort if basic physical forces are acknowledged and taken advantage of, such as: gravity (falling), momentum, inertia, balancing and others.
And all of those while staying in contact.

\subsection{Relaxation}\label{subsec:relaxation}

We move usually rather relaxed; a body which is ready for action yet open for receiving tactile stimulus, open for information.
We try to achieve that by deep breathing, by keeping a fluid movement quality (``octopus quality'') and also avoid a staring eye gaze.

Yet, a relaxed state should not be confused with a collapsed one.
An active state is also not the same as a hyper-tensed one.
Within this spectrum of non-extremes, we ought to find the optimal amount of muscle tonus which is appropriate for a given situation.
