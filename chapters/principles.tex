\section{Principles}\label{sec:principles}

\begin{wrapfigure}{R}{0.3\textwidth}
\centering
\includegraphics[width=0.25\textwidth]{images/principles}
\end{wrapfigure}

Many systems (sports, martial arts, \...) are focused around dozens or even hundreds of techniques which are thrown at a student to learn by hard, including their names, and precise definitions of what's right and what's wrong.
This is an approach which might work, but obviously has some serious disadvantages when it comes to quickly responding (picking the right technique from many within a split of a second) and more importantly the ability for individual expression.

In contrast to that, CI is centered around a few core principles, and every technique which might be taught, studied and practiced is a manifestation of those core principles.
There are therefore no real moves to be learned, but more principles to be embodied and applied in any given moment.
Once the principles are well understood, one can free oneself from the limitations of specific techniques, and questions like whether something is ``right or wrong'' can be easily answered by asking those principles.
Yet, as it is with the mastery of any art: Once the principles are fully understood, they can be broken if desired so, as: ``\textit{You can do whatever you want, as long as you know what you are doing.}''

\subsection{Grounding}\label{subsec:grounding}

With grounding we are referring to some kind of sensation (light) heaviness in the body, which makes the stance more stable, more robust and thus more connected to the ground.
Imaginary language like ``rooting'', and similar, are often used to describe this internal sensation, with its very realistic impact on the external.
This quality is the beginning of it all, without it we can't go any further, as without a firm foundation there is no house we can build upon it.
To help improving our groundedness we can use visualizations (roots growing into the ground), focusing our attention to where the sole's of the feet have contact with the floor, breathing out and relaxing the muscular tension without collapsing in one's structure, and simply thinking about words which are associated with a grounded, firm, or stable quality.

It should not be confused with stiffness, which so often lead to the illusion of groundedness, which is achieved by simply contracting all muscles; something we don't want to do as it will remove the ability to adapt in the moment, our flexibility.

\subsection{Pouring Weight}\label{subsec:pouring-weight}

Once we have established to ``gain some weight'' by grounding, we can use that to pour it into another person's body.
The emphasize here is to slowly increase the amount of pressure where the body's have contact, instead of a quick and sudden shift, which will be difficult and fear evoking movement for your partner; ultimately even potentially dangerous.
Instead, we want to ``announce'' that there is some weight approaching, so that our partner can adjust and adapt posture and internal tension/structure to that poured weight.

\subsection{Sharing Weight}\label{subsec:sharing-weight}

The first and most important principle is trying to seek a deep connection between two bodies, sometimes also called ``\textit{umpf}''.
It is different from actively pushing with muscular force, and also different from leaning by which one shifts one's center of gravity beyond a point of no-return.
The sensation should lead to a feeling of the ground beneath the partner's feet.
The body is stable and grounded, yet its limbs and joints are soft and relaxed; like an iron stick wrapped in cotton wool.
The ultimate goal is to maintain this quality throughout (almost) all time.

\subsection{Rolling Point of Contact}\label{subsec:rolling-point-of-contact}

Instead of sliding or jumping (point of contact), by using a spiraling and rotating movement pattern, the contact (and amount of pressure) is always maintained and follows a predictable trajectory, which means both partners can anticipate the very next movement, which furthermore leads to a more ``fluid sensation'' in the dance.
For this to happen, it is required to have a more agile body, bulging out body parts and bending and flexing whenever necessary.

\subsection{Pathway Continuation}\label{subsec:pathway-continuation}

According to the physical law of inertia, and to be in accordance with it, we should never break an already moving momentum (exceptions for the master applied here).
Once spiraling in one direction it should be maintained; possibility for anticipation, predictability and therefore trust on a psychological level, but also a mere reason of energy efficiency on the physical level.

\subsection{Movement Patterns}\label{subsec:movement-patterns}

Through a heightened awareness of communication through movement, touch and sharing weight, we explore the space and the connection between through mutual physical cooperation.
Fundamental movement patterns are:

\begin{itemize}
    \item \textbf{Yielding}: softening/surrendering into incoming force or to gravity
    \item \textbf{Pushing}: expansion, taking up space
    \item \textbf{Reaching}: physically or meta-physically
    \item \textbf{Pulling}: contraction, up til collapsing
    \item \textbf{Releasing}: relaxing into what's contracted before
\end{itemize}

All of those movements can be done easily with little muscular effort if basic physical forces are acknowledged and taken advantage of, such as: gravity (falling), momentum, inertia, balancing and others.
And all of those while staying in contact.
