\section{Principles}

\begin{wrapfigure}{R}{0.3\textwidth}
\centering
\includegraphics[width=0.25\textwidth]{images/principles.jpg}
\end{wrapfigure}

Many skills are focused around dozens of techniques which are thrown at a student to learn by hard, with names, and precise definitions of what's right and what's wrong. This is an approach which might work, but obviously has some serious disadvantages when it comes to quick responding (picking the right technique from many within a split of a second) and ability for individual expression.

In contrast to that, CI is centered around a few core principles, and every technique which might be taught, studied and practiced is a manifestation of those principles. Once the principles are well understood, one can free oneself from the limitations of specific techniques, and questions whether something is "right or wrong" can be easily answered by asking the principles. Yet, as it is as a master of any art, once the principles are fully understood and embodied, they can be broken at will, as: "\textit{You can do whatever you want, as long as you know what you are doing.}"

\subsection{Sharing Weight}

The first and most important principle is trying to seek a deep connection between two bodies, sometimes also called "\textit{umpf}". It is different from actively pushing with muscular force, and also different from leaning by which one shifts one's center of gravity beyond a point of no-return. The sensation should lead to a feeling of the ground beneath the partner's feet. The body is stable and grounded, yet its limbs and joints are soft and relaxed; like an iron stick wrapped in cotton wool. The ultimate goal is to maintain this quality throughout (almost) all time.

\subsection{Rolling Point of Contact}

Instead of sliding (point of contact), by using a spiraling and rotating movement pattern, the contact is always maintained and follows a more predictable trajectory, which means both partners can anticipate the very next movement, which furthermore leads to a more "fluid sensation" in the dance. For this to happen, it is required to have a more agile body, bulging out body parts and bending and flexing whenever necessary.

\subsection{Pathway Continuation}

According to the physical law of inertia, and to be in accordance with it, we should never break an already moving momentum (exceptions for the master applied here). Once spiraling in one direction it should be maintained; possibility for anticipation, predictability and therefore trust on a psychological level, but also a mere reason of energy efficiency on the physical level.