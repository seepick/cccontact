\section{Myself}\label{sec:myself}

To understand a system, we first need to understand its context and history.
If you want to understand psychoanalysis, for example, you first have to understand Sigmund Freud and the sociocultural environment he grew up in.
We are all just children of our time, and our views are shaped by it.
My own background will severely influence my approaches and values, ultimately also the content in this book.

\begin{displayquote}
    ``\textit{Good Kung-Fu looks bad, only bad Kung-Fu looks good.}'' -- my Aikido teacher
\end{displayquote}

My background is mainly rooted in the internal \textbf{martial arts} from Asia, and as such, my focus lies more on a practical approach (\textit{form follows function}).
For me, it is less about the aesthetics of movement, which might be more important for someone who is performing for others on stage, but more about the technical aspects.
``Right'' is what works in the most \textbf{pragmatic} way, meaning efficient in time and space, going into subjects like physics and biomechanics.
Whatever is within the principles of a system could also be considered is ``right'', relative to those principles, yet not necessarily ``better'' in every regard.

Besides those more physical aspects, the \textbf{psychological aspect} also has an important role to me.
The benefit for one's mental health, the ability to get to know oneself and others more deeply.
And, of course, a more philosophical/spiritual path which can also be walked with the help of this deep art.

As a body worker, I emphasize the importance of the non-verbal communication aspect which is happening while two people moving together.
The slowness, the gentleness in establishing a \textbf{deep connection} through listening, and the expressions of our personalities through this practice.

If you want to contact me, please feel free to do so by emailing me: \href{mailto:christoph@crashcoursecontact.org}{christoph@crashcoursecontact.org}
