\section{Community}\label{sec:community}

The term ``community'' is in some areas used for people living in the same place, something hippies would do who strive for a more ``natural'' approach to life, people who drop out the system, sharing knowledge about each other, rather than about a specific \textbf{practice}, and helping each other in times of need.
And there is even a current trend in our society that companies are using phrases like ``Come, join the family!'' which even deludes the word family very much.

We use the term ``community'' here for a group of people \textbf{sharing} the same passion for this practice, and getting curious about the other person, from a very basic physical connection.
Sometimes this curiosity goes to their life, experiences, and ideas, which sometimes can lead to afterward also sharing food; from there, friendship can happen.
Sometimes it is also just a great dance partner, but personally, we might have nothing in common, and we also do not want to become friends.
We see each other as ``only dance partners'', and the depth of care we feel is a bit less than, for example, for our beloved neighbor.

In the CI ``community'' this is a bit off though (compared to regular dance scenes), as when someone discontinues to come to classes or jams after a while, we might not be real friends, yet we would still reach out for each other, asking where he is, \textbf{helping} each other out (with health issues, financially) and support if needed.
This happens especially in smaller classes, where there is more of a sense of community.
These silos also happen according to the attitude of the teacher or the jam, where people will converge into specific places; norms will differ based on that.

People who practice or teach CI believe in what they do, in the goodness that it gives the world.
They want to expand the pool to reach more people, and that's why you will see little competition and more \textbf{collaboration} in the scene.
Maybe there will be a bit more competition if it's one's only source of income, a scarce modality; instead of CI being a livelihood, a hobby, a passion.
Some teachers will invite guest teachers, or send their students to another teacher, to a different city, or a teacher who is more technical or in some other way different.
There is not much ``fighting'' among CI practitioners happening, but also collaboration are not so often happening, as it is a very \textbf{individual practice}.
There are festivals, where (abroad) teachers are being invited, but usually when invited, then not by other teachers but by schools, institutions, or organizers.
The norm is nevertheless a ``city-based'' teacher who offers 2--3 classes a week, and sometimes there are ``travelling teachers''.
And there is also a win-win-win in having different student-teacher relationships in one's own workshop.

\subsection{Jams}\label{sec:jams}
%%%%%%%%%%%%%%%%%%%%%%%%%%%%%%%%%%%%%%%%%%%%%%%%%%%%%%%%%%%%%%%%%%%%%%%%%%%%%%%%%%%%%%%%%%%%%%%%%%%%%%%%%%%%%%%%%%%%%%%%

There is a broad, global community that regularly organizes social dances, so-called ``\gls{jam}s''.
You will see that many of those people also often overlap with the Ecstatic Dance communities\footnote{Many people from Ecstatic Dance say that they have done ``contact'' but what they are doing and their idea has little to nothing to do with CI. This confusion is about ``touch-based partner dance'' versus the art-form of CI (sharing weight, etc.). Those people might also bring in a slightly different (sensual-sexual, maybe even Tantric) attitude which is not welcome in CI.}.
Jams are \textbf{social gatherings} without a leader, yet sometimes with someone who facilitates it; similar to jazz jam sessions or Milongas in Tango.
It's an opportunity to practice \textbf{freely}, where people can meet and negotiate together their dance, or observe the practice of their partners.
It's an occasion to meet other fellow practitioners, friends or strangers, old, young, experienced, and novice, and has a bit of a taste of being a hybrid between bodily meditation, psycho-kinesthetic therapy, sports training and a dance improvised session.
They can be regularly in a studio for a few hours, or longer retreat jams for several days in spring resorts where it can be practiced at any hour of the day.

Sometimes you will encounter that the \textbf{facilitator} takes a bit more the lead, by hosting sharing circles at the beginning and/or at the end, and by providing some warm-up exercises.
Sometimes you will have a playlist providing musical background throughout, sometimes you might even encounter a live musician.
Yet, traditionally jams are being kept \textbf{silent}; to let fully unfold of what is alive presently between the dancing partners; to not let the music dictate externally but allow for full expression of what's present, as one of my dance teachers once said:

\begin{displayquote}
    ``\textit{Music makes a dancer stupid.}'' -- my dance teacher's teacher
\end{displayquote}

At the end (also sometimes in classes/workshops), you will sometimes participate in an exercise called ``\textbf{\gls{roundrobin}}''.
Usually two, three or four couples (or sometimes half of the people) are constantly in the center of the room, dancing.
The rest of the people are sitting on the edge, holding space, being present, giving in their energy.
As an active witness, you might learn something, or at least get inspired by simply observing.
When you enter and join a couple, another person is leaving, so the number of people stays constant throughout.

\subsection{Stereotype}\label{sec:stereotype}
%%%%%%%%%%%%%%%%%%%%%%%%%%%%%%%%%%%%%%%%%%%%%%%%%%%%%%%%%%%%%%%%%%%%%%%%%%%%%%%%%%%%%%%%%%%%%%%%%%%%%%%%%%%%%%%%%%%%%%%%

There is no clear stereotypical, \textbf{physical appearance} for so-called ``CI people'', as we might have when thinking of people who do Yoga, Hip Hop, or Tantra.
CI practitioners are nevertheless quickly spotted on the dance floor, not so much based on their looks but how they move, how they engage, their behavior with weight.
There appears to be a mutual, unconscious ability to spot each other on the dance floor, which makes them find each other rather quickly -like magic.

A stereotypical CI practitioner is often wearing common \textbf{dance clothes}, nothing too casual or street clothes, but also nothing too fancy.
The age range is rather big, yet for the technical part it's usually around 25--45.
Socioeconomically, it's a little bit on the higher end, meaning people who are \textbf{financially} a bit better off.
Teachers might be willing to give a discount to people who can't easily afford to pay for the classes/workshops, so in case needed, feel free to go ahead and ask, as the worst thing you could get is a ``no'' and in the best case you can afford to go for your dream.
Also, strictly \textbf{religious} people seem not to be attracted by it, most likely because of the touch and the physical proximity.

CI people have no need to attach themselves to a \textbf{subculture} and shape their personality around it; there is no need for ``another stamp'' as is seen so often in other practices.
CI is about the \textbf{physicality}, and less a ``spiritual practice for personal growth'', or a community, as stated above.
Still, the way we move might influence our personality; similar to the increased sensuality when moving Tango, but that's not a big thing in CI\@.
It's very \textbf{egalitarian}, meaning no roles, and a plain physical form.
It's more \textbf{technical}, more raw, and no (moral) values are being conveyed -whereas the egalitarian approach might be a value in itself.
Not only that, but it is \textbf{independent} of sex, gender, age, and body; there is no leading nor following based on sex or gender -just play.
Furthermore, it doesn't convince others of the greatness of CI, although everyone is invited and welcome to join -it's fun.
