\section{Definition}\label{sec:definition}


CI is something that doesn't want to be defined, and thus confined, limited in what it can be.
It wants to stay open, to be able to be what the people need it to be.
It's by design something that you can't grasp properly, leaving you with the feeling you might have after watching a movie with an open ending.
It stays a mystery.
Yet we will try in this chapter to reveal the secret, take you backstage, and tell you the secret of the magician, how he's doing it.

\subsection{Essence}\label{sec:essence}
%%%%%%%%%%%%%%%%%%%%%%%%%%%%%%%%%%%%%%%%%%%%%%%%%%%%%%%%%%%%%%%%%%%%%%%%%%%%%%%%%%%%%%%%%%%%%%%%%%%%%%%%%%%%%%%%%%%%%%%%

Without a proper definition of what something is, without capturing its essence, it can't be properly talked about.
How does it relate and especially separate itself from other similar systems?
What are its \textbf{goals and principles}, so we can always check whether a specific method or direction is in alignment with these goals and principles.
Not necessarily that it would be wrong, but just to be aware when we step out of our system, and step into something (completely) different.
Communication is usually loaded with misunderstandings, and by having a \textbf{clear, mutual understanding} of what it is we are actually referring to, we can create more harmony in our interaction through clarity (of words).

So what is CI (sometimes also called \textit{contact} or \textit{contact dance} in the community) \textbf{compared} to other systems?
A dance, a sport, a (visceral) art-form or post-modern art-sport, partner acrobatics or martial arts?
Well, maybe all of it or none of it, or perhaps depending on how you approach it and what your background is and where you put the emphasis on.
Let's take a look at the different views on it, and by also looking at its history, it might get clearer what it is, or what it can be for you.

\subsection{By Comparison}\label{sec:by-comparison}
%%%%%%%%%%%%%%%%%%%%%%%%%%%%%%%%%%%%%%%%%%%%%%%%%%%%%%%%%%%%%%%%%%%%%%%%%%%%%%%%%%%%%%%%%%%%%%%%%%%%%%%%%%%%%%%%%%%%%%%%

Most of the emphasis is put on the following areas:

\begin{itemize*}
    \item [] \textbf{Experimental dance} -- practice-based research in dance labs
    \item [] \textbf{Theatrical form} -- improvised performances and lectures
    \item [] \textbf{Educational tool} -- training for dancers in improvisation
    \item [] \textbf{Awareness practice} -- listening skills for the subtleties in contact
    \item [] \textbf{Social dancing} -- at informal gatherings called ``jams''
\end{itemize*}

At its core, it involves \textbf{mindfulness}, sensing and collecting information which requires full presence and attention.
Thus, it could be considered a ``relational mindfulness practice in movement''.

\subsubsection{Dance}\label{subsec:dance}

It could be considered a form of improvised partner dancing, embedded in contemporary and (post-)modern dance.
Dance improvisation existed already before CI, but it didn't have the contact aspect as CI has, and especially not the ``\textit{shared point of contact}'' principle.
The development in dance history started with ballet, to modern (Martha Gram, John Cage, Cunningham), and then towards contemporary -- improvisation became not only a research tool, but a performance itself.
Those are mostly vertical, meaning standing, without any floor work involved; whereas in CI, we roll over each other on the floor, something which would be seen as rather unusual in other styles.
We use a \textbf{three-dimensional} spherical space, where every body-part can be a foot -- and be pushed against the ground.
We dance by ourselves, with a partner, or with multiple partners; we engage, re-engage, and dis-engage at our own will without clear structures.

% IMPROVISATION
CI is an \textbf{improvised} ``art-form'' (or ``art-sport''), thus without any predefined choreography.
Sometimes CI is used in choreographies, which is then called ``partnering''.
CI is an exploration, a way to try to ``break the system'' as it is improvised.
Nevertheless, CI is not something you would see in the professional dance world, a world where they are walking with very ``nice movements''.
The improvised nature of CI shifts the attention very much on listening and avoiding forcing a plan.
There is usually not enough time to follow along a plan, as the window closes very quickly again for a potential move to happen.
And yes, it is even related to other improvisation practices like improvisation acting: the ``Yes, and \ldots''-concept expressed just physically.

It is not so much that as a CI practitioner, you are serving the \textbf{audience} (in a performance), but rather that the audience is joining you.
It's about getting a visceral reaction from the audience of an improvisation art-form, that they want to join.
Ultimately, there is no difference between dancers and the audience, no hierarchy exists.

% MUSIC
The connection is also mostly with the partner, and less with the \textbf{music} being played.
Sometimes, and originally, there is even no music at all playing.
With regular partner dancing, both synchronize together via the help of music, having the focus on the music and musicality.
In CI we sync with each other through the connection we establish through deep listening, and the focus is on the moment, and the interaction.
If there are songs being played, they are usually more in the background, lacking vocals and a prominent drumset.
We let the people be the navigator and the driver, a free form expression without the restrictions, limitations or manipulations of the music's quality.

% GENDERS
It is distinct from other partner dance styles, as those are more focused on the social aspect, whereas CI comes from the art world.
Many other dance forms have also very strict \textbf{gender roles}, where usually the man would lead, and the woman would follow; exceptions exist, like, for example, Lindy Hop, a form of Swing, the mother of Rock'n'Roll.
In CI, everyone dances with everyone, without any clearly defined role, a totally egalitarian system.

% PHYSICAL
CI has a strong focus on the \textbf{physical} aspect: How to survive a high-impact volume, which leads to either a crash or a fall?
Lindy Hop has, for example, also some aspects of acrobatics, which Zouk also has a little bit.
Thinking about taking/lifting people on your body, on your core, your center.

% CONTACT
Next to the obvious improvised nature of CI, the contact quality plays the other important role.
Question being: Does contact always require constant, physical contact, as in touching?
In more old-school CI there was lots of play with impact, with distance and closeness.
Today, there is more the tendency to stay permanently in contact, and if distance happens, it would usually mean the end of the dance.
Sometimes you can question and challenge this attitude.
Keep dancing in the \gls{negativespace}; engaging, disengaging; maintaining a more mental connection with your partner in the distance.

% AESTHETICS
There is also a difference in styling, as many \textbf{aesthetically} pleasing dance styles would ``dress to impress''.
In CI, we would typically come wearing our pyjamas (pajamas), sloppy, cozy, comfy, pragmatic.
As we have no audience, there is no performance, thus there is no urge to perform.
We are who we are, whether it looks good or not is not of relevance.
Maybe even not whether it has ``internal aesthetics'', whether it feels good, but just to feel itself with a non-judgmental mindset, welcoming what is, and experience some sort of satisfaction instead in this depth.

\subsubsection{Martial Art}\label{subsec:martial-art}

It is also combined with some minor aspects of eastern philosophy and a few things from the 1960s hippie culture.
The reason is that some founders of CI had their background in the Japanese martial art of \textbf{Aikido}, which is similar to Brazilian Ju-Jitsu.
Gently receiving incoming force and deflect it in circular movements, not resisting but rather listen, perceive, interpret and work with what's there at the moment.
Connecting with one's weight, one's center (''\textit{hara}`` in Japanese) and acting from that place.
By listening and using ``what is'', we can move with less effort.
Acting below the level of voluntary muscular contraction, a preliminary state, being prepared.
The joints and the whole state of being stimulated, readied for action.

Yet of course, CI totally lacks the intention of it being even remotely applied in a contest or let alone in a street fight.
There is no emphasis put on practicality during a physical conflict, still having skills in, for example, rolling (Judo) can be extremely useful.

We could also use the metaphysical, philosophical concept of Ki (or ``Qi'' in Chinese), which expresses itself in the quality and potential of connections.
It's active in an ideokinetic way, as an image-effect, which influences the course in which events might develop.
It is the potential of the extension-principle, a source, or better ``a radiation'' of energy within the body.

\subsubsection{Sports}\label{subsec:sports}

If we define sports of any form of physical activity, then CI (as like any other dance form), is for sure a sport.
Yet, sports usually has an ego-benefitting aspect by being able to measure performance and even having competitions where people/teams compete against each other -something totally missing in CI, yet not in most dance forms.
Sports also usually lacks the depth, the possibilty for ``soul exploration'' and spiritual growth which also has often an emphasis in CI.

\subsubsection{Acrobatics}\label{subsec:acrobatics}

The main founder, Steve Paxton himself, was a \textbf{gymnast}, and that might be the reason why still some of CI resembles more \textbf{partner acrobatics} than dancing.
The many (demanding) lifts, and the high velocity and high-impact movements can require a high level of fitness of the practitioners, just like acrobats do.
Whereas proper partner acrobatics goes way beyond what CI is doing, in terms of physical demands and risky techniques.

Acrobatics is Greek (\textit{akrobatéō}) and means to ``walk on tiptoe'', to strut.
The geometry of aerobatics is a combination of aeroplane and acrobatics, to fly and to tiptoe.
We use bodies, instead of airplanes, thus it would be more correct to call it ``bodatics'' or ``somatics'' maybe?

\subsubsection{Applied Physics}\label{subsec:applied-physics}

CI is an exploration / research practice of the \textbf{physical forces} of one's body in relationship to others by using the principle of sharing weight, touch, and movement awareness, while moving in contact and on the ground.
To play with the artistry of falling unbalanced and counterbalance, with gravity and physics.
To learn the mechanics of the body to handle someone's weight or to be lifted, along with breathing techniques.
Alternatively, it can be stated:

\begin{center}
    ``\textit{CI is a three-body-problem with two dancers, being shaped and moved by momentum, gravity, and force, dealing with the third body, the ground.}''
\end{center}

\subsection{Historically}\label{sec:historically}
%%%%%%%%%%%%%%%%%%%%%%%%%%%%%%%%%%%%%%%%%%%%%%%%%%%%%%%%%%%%%%%%%%%%%%%%%%%%%%%%%%%%%%%%%%%%%%%%%%%%%%%%%%%%%%%%%%%%%%%%

In the early beginnings, in the 1970s, every time the founding members practiced CI, they would redefine it, asking the very same question over and over again: ``What is CI \textit{now}?'' An early definition by \textbf{Steve Paxton} and others was (source: CQ Vol. 5:1, Fall 1979):

\begin{displayquote}
    ``A continuously evolving system of improvised movement.
    Two bodies, communicating with each other in physical contact, creating a relationship with the physical laws (motion, gravity, momentum, inertia).
    Sensitive, thus relaxed of unnecessary muscular tension, and willing to experience a natural flow.
    Techniques may include: Rolling, falling, being upside-down, following, supporting and giving weight.

    A physical dialogue ranging from stillness to highly energetic exchange.
    Alert enough to stay in an energetic state of physical disorientation, trusting your survival instincts.
    A free play seeking for balanced movements, leading to a physical and emotional truth, shared currently, leaving you informed, centered, and enlivened.''
\end{displayquote}

\subsection{Possible Definitions}\label{sec:possible-definitions}
%%%%%%%%%%%%%%%%%%%%%%%%%%%%%%%%%%%%%%%%%%%%%%%%%%%%%%%%%%%%%%%%%%%%%%%%%%%%%%%%%%%%%%%%%%%%%%%%%%%%%%%%%%%%%%%%%%%%%%%%

Definitions, and the same with any terminology, are time-, space- and person-dependent.
Just as with the improvised nature of CI, it is a constantly moving target.
By asking the same question a week later, the answer you receive will already be different.
And as such, the answer you will give to yourself, what CI is for you, will and most likely also should change, as you yourself are in constant change.

There is still a rough \textbf{body of agreement} -- a congruent understanding -- of what it is, and what it is not, leading to a definition stable across time, space, and practitioner.
It's like a tree: The trunk which are the agreed principles, and it's branches and leaves (the styles/adaptations) representing the edges where you can play freely.

\textbf{Steve Paxton} himself stated in 1979:
\begin{displayquote}
    ``The exigencies\footnote{exigency = a pressing or urgent situation, requirement, or need} of the form dictate a mode of movement which is relaxed, constantly aware and prepared, and on flowing.
    As a basic focus, the dancers remain in physical touch, mutually supportive and innovative, meditating upon the physical laws relating to their masses: gravity, momentum, inertia, and friction.
    They do not strive to achieve results, but rather, to meet the constantly changing physical reality with appropriate placement and energy.''
\end{displayquote}

\textbf{Nancy Stark Smith} once mentioned:
\begin{displayquote}
    ``It resembles other familiar duet forms, such as the embrace, wrestling, surfing, martial arts, and the Jitterbug (Lindy Hop and swing dances), encompassing a wide range of movement from stillness to highly athletic.''
\end{displayquote}

\textbf{Daniel Lepkoff} states about the core of CI:
\begin{displayquote}
    ``To put focus on bodily awareness and physical reflexes, rather than consciously controlled movements.
    Precedence of body experience first, and mindful cognition second, is an essential distinction between CI and other approaches to dance.''
\end{displayquote}

\textbf{Ray Chung} once announced in a workshop 2009 that:
\begin{displayquote}
    ``CI is an open-ended exploration of the kinesthetic possibilities of bodies moving through contact.
    Sometimes wild and athletic, sometimes quiet and meditative, it is a form open to all bodies and inquiring minds.''
\end{displayquote}

\subsection{Beyond CI}\label{sec:beyond-ci}
%%%%%%%%%%%%%%%%%%%%%%%%%%%%%%%%%%%%%%%%%%%%%%%%%%%%%%%%%%%%%%%%%%%%%%%%%%%%%%%%%%%%%%%%%%%%%%%%%%%%%%%%%%%%%%%%%%%%%%%%

CI is for sure not a pure \textbf{martial art}, as it has no claim to have any practical fighting application.
Nor is it a competitive \textbf{sport} in any way, as there are no competitions and due to its artistic nature it would be difficult, or at least without much meaning, to judge one as being better than the other.
It has many aspects of partner \textbf{acrobatics}, but lacks many technical possibilities due to the nature of its principles.

It is also not really your regular \textbf{dance} form like Salsa or Tango, due to several reasons: We don't dress (nor dance) to impress but rather show up with our authentic selves; we don't move to only look good or aesthetically (to others); we focus more on the internal and interpersonal aspects than the external ones; we often dance without music; there are no real techniques which can be learned but only guiding principles from which some specific movements can emerge.

As it is the case with so many (or maybe even all?) disciplines: Once the rule has been understood, and you know what you are doing, it can be broken by the student, thus \textbf{becoming a master} of it.
Finally, it has to be mentioned that a system\footnote{In this case we talk about CI as a system, but the forementioned wisdome can be applied to our world's system, politically or economically, to technology, to our endless todo lists \ldots} is supposed to be of service to the user (not the other way round), and its boundaries and dogma should not limit but enrich the applicant.
Whenever the purpose is hindered by the system, the system shall be left behind, and we should remember the original goal which was there in the first place, and not to serve the gods we created.
