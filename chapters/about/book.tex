\section{Book}\label{sec:book}

\begin{displayquote}
    ``\textit{Alles was gelernt werden kann, ist nicht wert gelehrt zu werden.}'' -- my older brother
\end{displayquote}

% History; intention, motivation
The following pages were initially written for the sole purpose of taking \textbf{personal notes} of my own experience, thoughts and insights.
After some time, it started to grow bigger and bigger, until it reached a level where it could also be of use to others.
Interesting also, that most of the questions being asked from beginners are the same, which means it is easily possible to satisfy that curiosity with a predefined set of answers.
Over many years, collecting from those sources, a structure would emerge, and that's how this book got its shape and (physical) manifestation.

% Emphasis; goal, USP
Looking at the other -few- books about Contact Improvisation (or short: CI) on the market, it seemed to me there was something missing for me.
A \textbf{structured approach} to the technicalities, the rules and principles explained in details, using less of the imaginary language but something an engineering mind can better relate to.
Furthermore, I had the need to make the \textbf{implicit norms and rules} in a delicate, social context such as created during a CI event, explicit.
This would allow people reading this book to safely navigate through the space without bumping and crossing on or the other boundary, without crash-landing, neither physically nor psychologically.

% Target group
Whether you consider yourself a \textbf{beginner or advanced} practitioner, I feel certain that you'll find something of value to you.
Being interested in getting more acquainted with the \textbf{theoretical background} -- next to your regular practice in the studio -- of this fine art is the only prerequisite.

\subsection{Disclaimer}\label{subsec:disclaimer}

The book uses the \textbf{masculine version} when using examples for the sake of simplicity, in case both can be applied, and of course, always implying that the female version could have been equally used as well.

The book is \textbf{incomplete}, as any form of completeness can never be achieved anyway; we are not dealing with a hard science here.
I'd like to ask you to be gentle when you encounter an attempt of enumerating possibilities where you know that something is missing.
If you do so, please consider contacting me, contributing to this little project, and bringing it closer to completion and perfection, yet knowing we will never reach it, an asymptotic goal.

Whatever is written on the following pages does not claim to hold any absolute, objective truth.
Some parts are summaries of sources found along a path of exploration, whether they might be (direct or indirect) oral teachings, or written information found in books or on the internet.
Most of it is just a very \textbf{personal, subjective} collection of experiences, thoughts, associations, observations and opinions which were gathered along a single person's journey.
The content is, of course, also biased, although I made my best to free myself from my own limitations of reality-perception.
The content, thus, will sometimes be colored by my own personal background.
I shall be forgiven for my flawed, limited human nature.

\begin{displayquote}
    ``\textit{You are entitled to your opinion but you are not entitled to your own facts.}'' -- Daniel Patrick Moynihan
\end{displayquote}

It is said that there are no books existing written by a true master.
The reason being, that a master knows that it is impossible to capture the essence of a system in words, in a linear expression.
It happens sometimes, though, that students try to do that for the master, transcribing his teachings themselves.
I shall be forgiven for being such a student, desperate enough to achieve the unachievable in my own ignorance.
