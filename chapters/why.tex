\section{Why}\label{sec:why}

``\textit{If meditation is the process of being in the present moment, in the so-called here and now, then contact improvisation is the highest form of meditation.}''

Besides the obvious reason for dancers to expand their skill-set, non-professionals are most likely to start with CI for either simply pleasure of doing it, or for personal development purposes.

\subsection{The Power of Touch}\label{subsec:the-power-of-touch}

\begin{wrapfigure}{R}{0.3\textwidth}
\centering
\includegraphics[width=0.25\textwidth]{images/why}
\end{wrapfigure}

The tactile experience is most prominent during practicing CI. To be specific, it's the activation of nerve fibers in the skin, so called C-tactile afferents, through gentle and slow stroking with body-like temperature (``warm touch'' leads to release of serotonin, a hormone which makes us feel happy, regulates our emotions and associated with sympathy and interpersonal affection; ``cold touch'' is experienced when being socially excluded, as the skin temperature decreases).
This increases activity in the insular brain region, which is responsible for interoception (perception of the current body state and facilitation of the sense of self).
Through this touch, which can be soft to slide, or deep to the bones, is used to establish a nonverbal, two-way dialogue process, a form of communication.
In the UK alone, about 54\% (among 40,000 participants) want more touch in their lives!
We live in a low-touch society, and mistrusting strangers became a default.

This touch reduces stress (lowered cortisol levels, which is a biomarker for stress-related diseases), increases oxytocin levels (attachment, connectedness) and lower blood pressure.
If a person lacks touch, disorders like depression, a distorted body image, low self-esteem and heightened aggression and self-injurious behavior might occur.
It counteracts a feeling of loneliness and isolation, which is the case in many forms of mental disorders.
Touch releases endogenous opioids, which make us feel relaxed, good, and more resilient.

Yet, touch needs to happen with consent in order to be experienced as something positive.
To want to touch, and want to be touched.
Respecting each other's boundaries, and the possibility to step out of touch at any given moment.
A good CI facilitator will address those things, by, for example, introducing ``no'' exercises, and safety rules at an event.
Especially beginners, who are easily overwhelmed by touch and lust trust in others due to previous bad experiences can benefit from a gender-based group and sharing circles in a safe space to reflect upon.

\subsection{Psychological Health Benefits}\label{subsec:psychological-health-benefits}

It is said to have positive effects on stress relief, relaxation, well-being, happiness, joy, connectedness, empowerment and feeling more fearless.
Also a clearer experience of a ``sense of self'' and the body; being aware of one's own existence.

The massage effect of ``sharing weight'' seems beneficial against anxiety, depression, ADHD, eating disorders (better sense of one's own body), autoimmune diseases, chronic fatigue syndrome, and more.
It also helps with the activation of the parasympathetic nervous system (``\textit{rest and digest}'').

As the attention is attracted to the ``shared zones of contact'', it brings us to the present moment, and thus becomes a mindfulness practice, distracting us from stressful thoughts.
There is no intention to reach a certain goal, but simply exploring movement and touch itself.
This requires us to be attentive, aware and present.

Also the personality gets strengthened, by being more aware of one's own needs, and distinguish oneself from the others.
Patterns can be broken, and a positive impact in resilience-building might occur.

\subsection{Social Bonding}\label{subsec:social-bonding}

CI helps us to relate better to others, to create strong social connections.
The experience of giving and receiving support via ``sharing weight'' for example.
The sensation goes beyond the pure physically into the emotional realm.
To trust each other, to feel safe again.
Presenting our vulnerabilities and instabilities, ultimately leading to a state of intimacy.

Reported experiences during a dance can be: playfulness, surprises, curiosity, flow state, feeling free, being alive; vitality, nourished, energized; connectedness; trust, closeness, deep communication, safe, secure.
The sensation of connectedness feels like being one new, ``third entity'', as it is not clear who gave the impulse to move in a certain way (quote Paxton).
By feeling the other person more, I feel my own boundaries better, thus I feel myself more, ``feeling more home in my own body''.
Yet, negative sensations might occur as well of course, like: boredom, insecurities, doubts, exhaustion, strenuous and being at one's own mercy.
