\section{Movement}\label{sec:movement}

\begin{wrapfigure}{R}{0.3\textwidth}
    \centering
    \includegraphics[width=0.25\textwidth]{images/movement}
\end{wrapfigure}

This chapter is based on the previous ones about anatomy and physics, and how these concepts give raise to higher concepts we encounter while practicing CI.
It's about the science of movement, movement theory and a bit of biomechanics and also dealing with more abstract views on movement as maybe known from the dancing world.
For more experienced (improvisation) dancers, a lot of the things mentioned here might sound familiar, for the more inexperienced among us this can open a totally new window to movement.

\subsection{Movement Qualities}\label{subsec:movement-qualities}

There are different ``dimensions'' we can play with to tap into different qualities on how we use our body, giving it a different touch, allowing us to do different techniques, and by knowing the pros and cons of each quality, we can apply them in the right moment to improve our technical skill and also keep things safe for ourselves and our partner.

We can play a lot with \textbf{muscle tonus}, the amount of tension we create in our muscles which make us either more relaxed or stiff.
In general, we prefer to maintain the least amount of effort, minimum muscle tension; as little as possible, as much as necessary.
By being more relaxed, we are more flexible, can adapt to an ever-changing situation, and also are more receptive via our tactile sense, being able to receive more information, to listen better.
In order to have some images to play with, think of moving through air; well, you don't have to imagine that, as we constantly do that; duh!
Instead, think of you being a cloud, floating through the sky.
Increase tension by imagining moving through water, how it creates a small, continuous resistance, preventing you from sharp/edgy movements, breaking the pathway.
The next level could be something like honey, sticky and slowing down your movements, having you to add more muscle effort.
And as a final step imagine being stuck in concrete which maybe has not yet fully harden, still making you almost unmovable.

The obvious one being \textbf{speed}: moving on a dimension of slow and fast.
The slow can be extremely slow, in order to gain extremely lots of information of internal sensations.
And the fast can be released in an explosive manner, like a shockwave through the whole body.
Both, and everything in-between, can be alternated very quickly, to gain more control of speed.

The degree of extension of the limbs into the space (without stepping) is called \gls{kinesphere} with which we can play with.
We could segregate it into a small (body), medium (elbows, knees), large (wrists, ankles) and extra large (fingers, toes), and something more abstract going even beyond, the universe (projecting outwards).
Each of them creating a differently sized ball, or more like an egg shape, around us in which we are limited to move within and also want to stay in constant contact with.

Next to extension into space, we can of course take different \textbf{levels} in space: Up (standing), middle (hinged, or on hands and knees being a ``little animal'') and floorwork (lying on the ground).
With the help of relaxation and tension, we can quickly change our position on the vertical axis and play with explosive dynamics.
Also mirroring your dance partner, staying in a different level then he is can be an interesting field to explore.

The \textbf{isolation} of certain body parts can also be fun to play with.
The most simple one being dividing the body into upper and lower, left and right side, same side or cross side (homo- or contra-lateral, see the anatomy section for more).

The \textbf{shapes} we are drawing in air can be yet another dimension.
Think of straight lines (edgy, staccato) versus roundness (flowy, fluid, air).
People who have experience with the practice of 5 rhythms might be familiar with those concepts.

To bring those qualities to a next level, try to combine them in different ways.
Often slow, fluid and soft goes together, but how about changing one of them to the other extreme?!
Or the legs are in the large kinesphere being soft and staccato, while the arms in the small kinesphere and hard and fluid.
Have a partner telling you what parts should be in which quality to challenge yourself and find combinations you would not bring yourself into by yourself.

\subsection{Partner Dance}\label{subsec:partner-dance}

Disclaimer: There is lots to tell about dancing with one (or more) partner(s), and this subsection will be far from mentioning all the most relevant there is.

Humans are social animals, and as such we have lots of neuronal networks dedicated for processing social information, hardwired bonding tendencies leading to the ability to read each other, get in-tuned with each other, ultimately leading to harmony in groups.
Think of basic empathy, which is associated to mirror neurons in our brain, which for example leads us to yawn once we see someone else yawning, or feel the pain we observe in that funny video clip where someone falls from a skateboard.
If we want to feel how someone else is feeling, we might simply want to put our full presence with that person (or even a group) and feel inside our bodies what's happening; there is a high chance what we experience is the other person's state of being.
Using a more flowerly we could say that this is the ability of ``feeling the other person's energy'', whatever is supposed to be meant by (mis)using the word ``energy'' here.

As human beings we also have the innate need to be seen by others; being ignored as one of the worst punishments we can experience leading to feelings of isolation, exclusion and loneliness.
The biggest present therefor we can give each other is to mirror each other's movements; explicitly acknowledging the other person's existence, putting one's attention to that person, copying his movements, which is also called ``kinesthetic resonance''.
Of course counter-mirroring has the same effect (you go down, I go up; you go left, I go right), as it still maintains a form of connection we can play with.

\subsection{Body Awareness}\label{subsec:body-awareness}

The \textbf{vestibular system} is our sense of balance and spatial orientation for the purpose of movement coordination, and is all well known to us.
It consists of two components: The (three, as there are three dimensions) semicircular canals (for rotational movements) and the otoliths (for linear accelerations).
Signals from those are being sent to the muscles to keep us upright and control posture, allowing us to maintain our desired position in space.
Together with proprioception, we are able to understand our body's dynamics and kinematics in any given moment.

\textbf{\Gls{proprioception}} allows us, as a sort of 6th sense, the ``kinesthetic sense'', the sense of self-movement, force and body position.
It makes us aware of our body's position in the space (the relative positioning of neighboring body parts), and the strength of effort needed for movement.
Try for example to close your eyes and touch your nose; you will be able to do this without looking (in a mirror, or in complete darkness) because of little cells (little \textit{spindles} which are spring-like protein molecules which get stretched inside your muscles) in your body being aware of the amount of stretch they experience (joint position sense), which is then processed subconsciously by your brain giving raise to a bodily sensation.
Everyone is familiar with the knee-jerk reflex, where the patellar tendon is rapidly stretched to an extreme, which leads to an immediate response (a reflex) to counteract that and protect the tissue from injury.

Related sensations are \textit{exteroception}, the perception of the outside world, and \textit{interoception}, the perception of internal sensations like pain and hunger.

\textbf{\Gls{kinesthesia}} is the awareness of position and movement of body parts using sensory organs (proprioceptors, mechanosensory neurons) in muscles, tendons and joints; it's crucial in muscle memory and hand-eye coordination.
It's different as proprioception, yet people sometimes use it (wrongly) interchangeably.
If one has an inner ear infection for example and the sense of balance is affected, this would degrade the proprioceptive, but not the kinesthetic sense.
Moreover, proprioception is more about joint \textbf{position} (more subconscious cognitive awareness of your body in space and balance) whereas kinesthesia is more about awareness of joint \textbf{movement} (more conscious body's motion, behavioral).

\textbf{Neuromuscular control} is the efferent (signal from the central nervous system to the body) response to an afferent (sensory) input, which is the functional component to movement and athletic activities that is referred to as dynamic stability.
Sensory input comes as (different types of\footnote{The four mechanoreceptors are: Meissner corpuscle for heavy pressure, Pacinian corpuscle for vibraiton, Merkel disks for light touch and Ruffini endings for skin stretch}) \textbf{mechanoreceptors} located in muscles, capsules and ligaments, allowing us awareness of joint position, movement, and acceleration.

All that information (vestibular, proprioception, kinesthesia) including the visual input is sent to the brain, processed, integrated to allow us to create an overall representation of body position, movement and acceleration.

\subsection{Space Harmony}\label{subsec:space-harmony}

This movement theory, and practice, also called \textit{Choreutics} was developed by the Austrian-Hungarian dancer/choreographer Rudolf Laban, to study the natural sequences of movements we follow in daily life, studying ``the art of movement'' to recognize spatial patterns.

When dancing, the term \textbf{\gls{kinesphere}} is being used to refer to the space immediately reachable by our limbs without changing our place; we can use up a lot of that space within this sphere (Far Reach Kinesphere), just a bit (Near Reach Kinesphere) or something in-between (Mid Reach Kinesphere).

\begin{wrapfigure}{R}{0.3\textwidth}
    \centering
    \includegraphics[width=0.25\textwidth]{images/kinsphere}
    \caption{The kinesphere is the sphere around the body whose periphery can be reached by easily extended limbs.}
\end{wrapfigure}

Furthermore, Mister Laban believed that there are three types of movers which prefer different \textbf{levels}: Those who enjoy leaping and springing off the ground move in \textit{High Level}; those with more sensuous movement enjoy the \textit{Central (Middle) Level}; and those who prefer more earth-bound movements who stay in the \textit{Deep (Low) Level}.

Within the kinesphere we can move from one point to another through different approaches, so-called \textbf{pathways}: When movement is initiated from (or passes through) the body's center we take the \textit{Central Pathway}; along the outer limits of the Kinesphere it takes a \textit{Peripheral Pathway}; and when the movement passes between center and periphery it takes a \textit{Transverse Pathway}.

There is of course much more to say about this, including the Laban Movement Analysis (LMA) which is a method and language for describing, visualizing, interpreting, and documenting human movement; but this would go beyond the purpose of this booklet.
