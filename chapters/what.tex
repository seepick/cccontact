\section{What}\label{sec:what}

\begin{wrapfigure}{R}{0.3\textwidth}
\centering
\includegraphics[width=0.25\textwidth]{images/what}
\end{wrapfigure}

So what is CI? A dance, a sport, acrobatics or martial arts?
Well, all of it.

It could be considered a form of improvised partner dancing, a contemporary dance form, an ``art-sport''.
An exploration of one's body in relationship to others by using principles of sharing weight, touch, and movement awareness.
To play with the artistry of falling off balance and counterbalance.
To learn the mechanics of the body to handle someone's weight or to be lifted, along with breathing techniques.
At its core it involves mindfulness, sensing and collecting information.

Emphasize is put on:
\begin{itemize}
	\item \textbf{Experimental dance}: practice-based research in dance laboratories
	\item \textbf{Theatrical form}: improvised performances and lectures-demonstrations
	\item \textbf{Educational tool}: training for professional dancers in improvisation
	\item \textbf{Awareness practice}: being able to listen to the subtleties in contact
	\item \textbf{Social dancing}: at informal gatherings called ``jams''
\end{itemize}

This art form is not only for the young and well-trained, as there is no real requirement for acrobatic performance.
The body just needs to be mobile and the bones bear some weight.

There is a broad global community, which organizes social dances, so-called ``jams'', and practitioners often overlap with the ecstatic dance communities.


\subsection{Definitions}\label{subsec:definitions}

\textbf{Steve Paxton} himself stated in 1979: \textit{The exigencies of the form dictate a mode of movement which is relaxed, constantly aware and prepared, and onflowing. As a basic focus, the dancers remain in physical touch, mutually supportive and innovative, meditating upon the physical laws relating to their masses: gravity, momentum, inertia, and friction. They do not strive to achieve results, but rather, to meet the constantly changing physical reality with appropriate placement and energy.}

\textbf{Nancy Start Smith} once mentioned, it ``\textit{resembles other familiar duet forms, such as the embrace, wrestling, surfing, martial arts, and the Jitterbug (Lindy Hop and swing dances), encompassing a wide range of movement from stillness to highly athletic}''.

\textbf{Daniel Lepkoff} states about the core of CI, to ``\textit{put focus on bodily awareness and physical reflexes, rather than consciously controlled movements. Precedence of body experience first, and mindful cognition second, is an essential distinction between CI and other approaches to dance.}''

\subsection{Beyond CI}\label{subsec:beyond-ci}

CI is for sure not a pure martial art, as it has no claim to have any real fighting application.
Nor is it a competitive sport in any way, as there are no competitions and due to it's artistic nature it would be difficult (impossible?) to judge one as being better than the other.
It has many aspects of partner acrobatics, but lacks many possibilities due to its confining principles (e.g.\ grabbing/manipulation).

It is also not really your regular dance form like salsa or tango, due to several reasons: We don't dress to impress but rather show up with our pyjamas; we don't even necessarily dance in order to look good or aesthetically; we focus more on the internal and interpersonal aspects than external ones; we dance often without music; there are no real techniques which can be learned but only guiding principles from which some specific movements can emerge; and so forth.

As it is the case with so many (all?) disciplines, once the rule has been understood, it can be broken by the student, thus becoming a master of it.
Furthermore, a system is supposed to be of service to the user, and its boundaries and dogma should not limit but enrich the applicant.
Whenever the purpose is hindered by the system, the system shall be left behind and we should remember the original goal which was there in the first place, and not to serve the gods we created.
