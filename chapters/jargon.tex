\section{Jargon}\label{sec:jargon}

\begin{wrapfigure}{R}{0.3\textwidth}
    \centering
    \includegraphics[width=0.25\textwidth]{images/jargon}
\end{wrapfigure}

There are no universal names for techniques or qualities, but different teachers try to invent, share, and give credits to their inventions.
Over the time, an oral tradition has been established of passing information.
The following chapter is a rather class specific one, as the language and images developed in the scene ``I grew up in'' are very local and not universal at all.
Yet, they might inspire others, or at least provide some humorist benefit.

Introduction of a jargon, using technical, domain-specific terms, is useful to express an idea, as otherwise many, many sentences would be needed.
The whole purpose of introducing and using jargon is not to smart-ass, to show off and exclude others, but for increased information exchange and increase in information density.

\subsection{General}\label{subsec:general}

\begin{itemize}
    \item \gls{smalldance} -- The subtle, unconscious micro movements of the body trying to maintain its balance.
    \item \gls{kinesphere} -- The space which can be reached by the body/limbs without taking a step.
    \item \ldots see the glossary \ldots
\end{itemize}

\subsection{Onomatopoeia}\label{subsec:onomatopoeia}

Sometimes concept are too complex, too difficult to verbalize them fully in their precise definition.
Or simply there have been no proper words established yet in our (English) language, which made it necessary to develop our own few words, or better ``sound-terms'' to convey a certain meaning, qualities, fuzzy principles and abstract phenomena more efficiently:

\begin{itemize}
    \item \textbf{botsen}: A conflict which arises when the teacher's instructions lead to a resistance based on an internal wisdom; when the ``inner teacher'' and the ``outer teacher'' clash, we usually opt for the inner one and trust our gut feeling; e.g.\ Jump and do a roll, but it doesn't feel safe but fear/resistance, so you don't do it.
    (Dutch word: to clash/collide; bot=bone; like two fists smashed on each other)
    \item \textbf{hoopity poop}: Techniques which look overly impressive, yet not necessarily showing much skill though, like jumping on each other; fancy technical, acrobatic stuff.
    \item \item \textbf{lala land}: Referring to anything which is more commonly used in the area of spirituality/religion, metaphysics, esoteric new age talk and superstitions, yet using it nevertheless due to lack of better words; e.g.\ when talking about reaching beyond the physical body.
    \item \textbf{muschi muschi}: Indicating that during a dance a too sensual and thus inappropriate atmosphere arises, with non-clear intentions; ranging from simply moving very slowly and caressing to cuddle puddles.
    Usually when starting with flat dances, preceded by a body-scan, where things can even end up in a sort of ``cuddle puddle''.
    \item \textbf{umpf}: The preferred quality of contact between two bodies which is characterized by properly giving/sharing weight; contact is established throughout the whole body (-part) and there is a considerable amount of pressure, not only slight touch; e.g.\ when both partners are in table-top position, and they feel firmly each other's groundedness.
    \item \textbf{wee}: A scream/sung sound usually expressed only in moments of heightened levels of alertness/fear, to down-regulate one's own nervous system, counteracting fear/panic, and allowing the body/atmosphere to relax, release tension, calm down; also to simply express joy at the moment about a movement, making everyone smile; e.g.\ when performing more risky lifts.
\end{itemize}

\subsection{Animals}\label{subsec:animals}

Animals are beneficial as they have a very visual characteristic, conveying a visceral experience, the qualities that animal embodies.
We mostly use their names to refer to positions, techniques and ``qualities of body''.

\begin{itemize}
    \item \textbf{Bear}: Similar to the Koala, but while sitting on the ground and the bear hugging around the torso.
    \item \textbf{Banana}: Not really an animal, but anyway a useful metaphor of a way of movement on the ground, rolling sideways and only the core touching the floor, arms and legs stretched out long, shaping the whole body like a banana; it can be practiced on the ground, but applied often while being on the back of a partner.
    \item \textbf{Chicken Wing}: Using a ``semi lock'' with the arm pit on the partner while being lifted, or similar; often necessary if the centers are not stacked and becoming unbalanced.
    \item \textbf{Chicken Leg}: Same as the chicken wing but with the hip flexed.
    \item \textbf{Chicken Leg}: Same as the chicken wing but with the hip flexed.
    \item \textbf{Crab}: The opposite of the octopus movement quality: Rigid, sharp, direct, staccato.
    \item \textbf{Koala}: When being lifted hugging the upper body of the partner sideways and thus being very close to his center; either around shoulders or pelvis.
    \item \textbf{(Little) Elephant}: Although we do like elephants, but we don't like them in the dance studio, as their name is used to refer to steps/walking, or landing of the feet, which make a loud sound, indicating that there was no control and/or too much stiffness.
    \item \textbf{Little Monkey}: As a little animal (a.k.a.: ``table-top'') but with knees lifted (a.k.a.: ``bear position'') with a light and fluent walking movement.
    Whenever we land silently, it is done so with control and elegance, which ultimately can prevent (serious) injuries.
    \item \textbf{Little Animal}: A table-top position on all fours, yet emphasizing a more dynamic, alive quality than a regular, wooden table.
    \item \textbf{Panda}: Similar to koala, but a more specific way of hugging, with belly to belly; often used while on the ground and keep the centers connected all the time.
    \item \textbf{Snake}: Similar to octopus, but with another image to connect differently: Lots of movement hands/spine in joints.
    \item \textbf{Octopus}: A movement quality which indicates aliveness/relaxation in all joints/body parts, each of them being controlled by their own intelligence, fluid, soft and smooth.
\end{itemize}
