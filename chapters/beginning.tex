\section{Beginning}\label{sec:beginning}

As a beginner it is sometimes a bit difficult to get the information to decide whether this art can be something for you or not.
Furthermore, we need to find some information to see what it is, and whether it actually fits us -in our needs but also in regard to the physical requirements on the body.
How to find a school, how to find a teacher, and what to look out for when doing so.
This section should provide you with some directions on doing your first steps into the world of CI.

\subsection{Where to begin?}\label{subsec:where-to-begin?}

To start it's always nice to check some YouTube videos (see the Resources chapter), but please don't be intimidated right away from what you are seeing.
The people you will see on those videos are usually already on a very high level, and that's not what you are going to encounter at first.

Join a recommended teacher and go first to some classes (see ``How to spot a good teacher'' further below).
It is not recommended to immediately go to jams; only after 10-20 classes, to have the basic principles embodied.
And don't give up if your very first experience seems bad; maybe you want to just try another class or another teacher.
Be also aware that CI is not only physical, thus ask yourself: Why and what are you doing?
This will be very beneficial and inspiring for your first steps with CI.

\subsection{Requirements}\label{subsec:requirements}

You might ask yourself for whom is CI fitted?
For whom is it, or is it not a good idea to join?
Are whether there are any physical requirements and how to deal with if one feels a bit touch-averse towards strangers.

In general, it can be said that every body is welcome, but not necessarily every behaviour though.
It is not about age, as even people 70 plus can join, neither about body ability.
Yet some techniques might not be possible to do, or would need some adaption.
Once a student with 72 years would even be able to do shoulder lifts, and he was doing it for many years.
Some of the more crazy things, he mentioned, was better to keep for ``the next life''.

Sharing weight, being in contact with another body, exploring what's happening, where the weight is, the unconscious reaction to balance, \ldots this, anybody can do.
Even with a partially disabled body, with some physical disability like people in a wheelchair can do it, together with physically fit people.
Or also blind people, as they are often also super in tune with weight and different other aspects.
It leads to a very different and interesting kind of exploration, and requires us to renegotiate, to re-explore what kind of communication we can play with.

Even with a huge weight difference, we just need to be more careful about which kinds of lifts to do, and who is carrying whom how.
We need to negotiate what's possible, and sometimes this means no shoulder lifts, and/or no rolling over.
Always respect the abilities of the dance and both bodies by checking each other out, slowly (!) and step-by-step, then it becomes \textit{safer}, yet not necessarily safe!
When about to engage in more advanced (crazy, dangerous) techniques, always do it with a lot of care and a \gls{guardianangel} who will spot you.

In theory, in an ideal world, everyone is welcome, yet it is not advised for people with severe touch aversion to jump into CI, as a solution to their trauma in that field.
It is recommended to better ``tippie toe'' in different kind of forms, and only then see if they want to continue with CI.
Mental challenges sometimes make people join only one (or even half a) class, or sometimes the teacher might even have doubts and expresses that.
People who take too many risks (the ``dangerous ones''), which are throwing themselves on unknown partners, won't be denied access to classes or jams, but dancers will most likely keep their distance to them, as it won't feel safe.

\subsection{How to spot a good teacher}\label{subsec:how-to-spot-a-good-teacher}

There are three realms to consider:

\begin{enumerate}
    \item The \textbf{physical} aspect: The teacher knows what he is teaching, and also being honest about his limitations in that knowledge; or in short: humbleness.
    \item The \textbf{psychological} aspect: This includes the ability to hold space, in case when processes might happen.
    Further, safety is created by the nature of his intention: Does he want to inflate his ego, making himself bigger?
    Does he claim to have a status of a master or guru, or is he ``one of us'' and going for a drink with the students, taking off his teacher's hat.
    Does he want to gain something from you outside the class, besides monetary compensation in a formally agreed transaction?
    \item The \textbf{spiritual} aspect: Watch out for the presence of competition.
    Does he genuinely want the student to be as good or even better as himself?
    Basically, does he have a pure intention to teach, sharing his knowledge (whereas this sharing is also always happening two-way though).
\end{enumerate}

\subsection{Becoming Good}\label{subsec:becoming-good}

What makes a good CI practitioner good can be answered in many ways.
It is for sure not only the obviously physical aspect, but also having the right intention for the dance, a proper attitude and mindset which is led by curiosity, the antidote to being judgmental.
An advanced practitioner is really happy exploring the smallest thing, being able to keep a beginner's mind open, and they know what they know, and they especially also know what they don't know.

For many, dancing with total beginners and total advanced is most enjoyable as they reminder us, showing us the different aspects of the dance.
Both of them are usually very happy to dance with each other as well.
On the other hand, intermediates are usually not happy to dance with beginners, they prefer to only stay with their own level or higher.
They are also usually the ``dangerous ones'', as they know the pathways and the tricks, the form and the looks, but they don't know what they don't know.
They are lacking the listening ability, if something happens which was not predicted in their pathways, like quick changes, they don't know how to handle that.
They also often go faster than their level of attention, and especially very often go faster than their beginner-partner's level of attention.
Having that said: Always respect what your body is able to do, in regard to level, age, and constitution.

A very important tool one should have is also the ability to listen to small/subtle sensations in the body, the change of movement and weight.
An increased awareness of incoming information, inside as well as outside the body via the peripheral nervous system, the so-called \gls{skinesphere}.
It leads us to be able to detect what's going on, thus becoming ``well-informed'' and leading us to become a much better dancer.

More general speaking, which applies to any form or practice:

\begin{itemize}
    \item \textbf{Kindness}: Smile, be gentle and soft.
    \item \textbf{Humbleness}: By not showing off how many years you are doing this-or-that, and not arrogantly teaching beginners.
    \item \textbf{Patience}: Let beginners do their own mistakes, and thus gain their own experience.
\end{itemize}

Everybody learns and everybody teaches.
There are no certificates.
If you have a lot of knowledge, you will share that by your dancing.

Years of practice itself are not necessarily a guarantee for expertise or even mastery!
Some people get stuck at a certain level even after many, many years.
Even lots of experience itself doesn't mean your level of technical practise is going forward.
Even if someone is technical level very advanced: The moment where one stops posing questions, is where one stops developing.
Always keep a beginner's mind: Keep on doing the very same thing, like lifts and spirals, yet stay with curiosity and explore deeper, like the small things and changes, where to focus now, to put your attention and intention to.
