\chapter{Techniques}\label{ch:techniques}

\chapterCoverImage{techniques}

Techniques is what makes most of the people feel \textbf{impressed}.
Think of a child visiting a circus, watching the acrobats doing their tricks.
We will quickly explore the role of techniques (versus principles), and \textbf{conceptually} introduce them.
Most prominantly of course are \textbf{lifts}, the signature move of CI\@.
This is not meant to be a full listing with detailed descriptions of specific techniques.
For that, you want to visit the complementary website of this book.

\section{Technique or Principle}\label{sec:technique-or-principle}

CI is a \textbf{principle-based system}, and as such it doesn't have a defined set of techniques which are universally taught and practiced.
Nevertheless, there are certain reoccurring movements (you could call them ``tricks'') which could be considered as techniques.
They should not be regarded as something to be followed too literally.
As long as you stay within the framework of the principles, any adoption can be judged as correct relative to the system.

The difference between technique and principle of CI is like the difference between \textbf{grammar} (universal and abstract) and \textbf{vocabulary} (specific and concrete) of language.
Even in the United States we will use the same grammar, but the words/vocabulary, and especially style/dialect will be slightly different from in the United Kingdom, Australia or anywhere else in the world; expect small hiccups to happen.
Nevertheless, we will be able to \textbf{understand} each other as the principles, the grammar stays the same.
And often it will be necessary to stay a few hours with a new teacher to embody his body language, from which the vocabulary comes.
Vocabulary, like lifts, is a rather \textbf{regional expression} of the application of the commonly shared principles.

\section{Lifts}\label{sec:lifts}

The most common -- and most impressive -- ``spice'' added to a dance are lifts.
Whenever one center is located underneath another center, a lift can be easily performed by ``pouring one's weight''\footnote{Pouring weight is one of the core principles of CI, slowly and continuously increasing the pressure, instead of jumping on our partner with a potentially heavy impact and possible injury} onto the base.

There are different kinds of lifts, but the most typical ones are hip and shoulder-lifts.
When performing a \textbf{hip-lift}, the \gls{over-dancer} usually places his butt, lower belly or side on the lower back of the \gls{under-dancer}.
The \textbf{shoulder-lift} is the highest level of a lift where one ``flies'' on the shoulder of the other person.
Those lifts are usually done standing and also while being in the ``little animal'' position, and to change for example from a hip to a shoulder-lift, the over-dancer can spiral upwards the back defying gravity.

As a base, we need to ground ourselves to become more stable (and heavy) by focusing on our \gls{centergravity},
whereas the flyer, on the other hand, tries to make himself light and engage in his \gls{centerleviathan}.

Lifting might lead to potentially dangerous situations, which require us to pay special attention to the following safety rules:

\begin{itemize*}
    \item \textbf{Not grabbing} is a general rule, using less the hands and more the torso.
        With lifting this becomes even more important. Also \textbf{no interlocking} of the arms or grabbing the limbs of your flyer.
    \item Always keep a \textbf{hollow back} (a.k.a. ``\gls{goodgorilla}'') to provide a stable support for your partner instead of rounding your back, which makes him feel down and freak him out.
    \item Always keep your \textbf{head above ass}, otherwise your partner will most likely react with a fear response because of the danger of sliding down over your head (especially don't combine this with grabbing/interlocking).
\end{itemize*}

Read more about this in the~\nameref{ch:safety} chapter.

\section{Spirals}\label{sec:spirals}

We use a lot of spirally movement patterns in CI and pay special attention to how they are perceived, seen and anticipated in different movements in our own or someone else's body.
Moving in spirals is the perfect way to keep the pathway continuation which adds to a more enjoyable, fluid sensation during the dance.
Spiraling is considered a core movement pattern in CI and there is much more to say and experience about it.

Spirals can be very visibly be done between two body parts by moving from the distal parts of the body towards the more proximal parts.
There is a lot possible with playing with the axis, changing the axis, going within or outside the body, \ldots the limit is the imagination (and skill).
And ultimately spirals can also only by visualized, imagined, with pure intention/attention without any external visibility of movement.

\section{Negative Space}\label{sec:negative-space}

Dancing in the so-called \gls{negativespace} basically means entering, usually by reaching into it with a limb, the \gls{kinesphere} of another person without engaging in touch.
This is a common way to non-verbally invite someone for contact, to start a dance together, at a jam for example.

\section{Other Techniques}\label{sec:other-techniques}

\textbf{\Gls{body-surfing}} happens when one person rolls on the floor, and the other (usually) perpendicularly rolls over him.
Watch out if your partner is too heavy, to be able to release/distribute his weight, or simply drop your partner off in a gentle and polite way.
This usually evokes a lot of laughter due to the fun nature of it, its playfulness which reminds us of our childhood games we played.
For even more fun, consider doing a body-surf with more than two people and make some sort of ``body train''.

\textbf{Counter-balancing} is commonly used in partner-acrobatics, where the center is, for once, oppositional, instead of centers being shared.
We basically ``lean away'' from the partner, and by holding on to each other somehow, we balance each other out.
