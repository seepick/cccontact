\section{Techniques}\label{sec:techniques}

\begin{wrapfigure}{R}{0.3\textwidth}
    \centering
    \includegraphics[width=0.25\textwidth]{images/techniques}
\end{wrapfigure}

Although CI is a principle-based system which doesn't follow a strict collection of well-defined techniques, there are certain reoccurring movements which could be considered as techniques.
Yet, they should not be regarded as something to be followed literally, and as long as one stays within the framework of the CI principles, any adaption is judged as correct.

\subsection{Lifts}\label{subsec:lifts}

The most common, and most impressive, ``spice'' added to a dance are lifts.
Whenever one center is lifted above another person's, a lift can be easily executed by ``pouring the weight''.
There are different kinds of lifts; typical ones are:

\begin{itemize}
    \item \textbf{Hip-lift}: Usually ones butt, lower belly or side is placed on the lower back of the other person.
    \item \textbf{Shoulder-lift}: The highest level of a lift when flying on the shoulder of another person.
\end{itemize}

\subsection{Spirals}\label{subsec:spirals}

To keep the pathway continuation, spirals are the perfect tool to doing so.
