\section{Techniques}\label{sec:techniques}

\begin{wrapfigure}{R}{0.3\textwidth}
    \centering
    \includegraphics[width=0.25\textwidth]{images/techniques}
\end{wrapfigure}

Although CI is a principle-based system and has those principles at its core, meaning it doesn't follow a strict collection of well-defined techniques, there are certain reoccurring movements which could be considered as techniques.
Yet, they should not be regarded as something to be followed literally, and as long as one stays within the framework of the CI principles, any adaption is judged as correct.

\subsection{Technique or Principle?}\label{subsec:technique-or-principle?}

The difference between this is like the difference between grammar (principles; universal and abstract) and vocabulary (techniques; specific and concrete) of a language.
Even in the United States we will use the same grammar, but the words/vocabulary, and especially style/dialect will be slightly different; expect small hiccups to happen.
Nevertheless, we will be able to understand each other as the principles stay the same.
And often it will be necessary to stay a few hours with a new teacher to embody his body language, from which the vocabulary comes.
Vocabulary, like lifts, is a rather regional expression of the application of the commonly shared principles.

\subsection{Lifts}\label{subsec:lifts}

The most common, and most impressive, ``spice'' added to a dance are lifts.
Whenever one center is lifted above another person's, a lift can be easily executed by ``pouring the weight'', one of the core principles, instead of jumping on our partner with a potentially heavy impact.
There are different kinds of lifts; typical ones are:

\begin{itemize}
    \item \textbf{Hip-lift}: Usually ones butt, lower belly or side is placed on the lower back of the other person.
    \item \textbf{Shoulder-lift}: The highest level of a lift when flying on the shoulder of another person.
\end{itemize}

As a base we tend to ground ourselves to become more grounded and stable, by focusing on our \gls{centergravity},
whereas the flyer tries to make himself light and engage in his \gls{centerleviathan}.

Lifting might lead to potentially dangerous situations, which require us to pay especially attention to the following safety rules:

\begin{itemize}
    \item Not grabbing is a general guideline, using less the hands and more the torso, yet with lifting this becomes especially important; thus no interlocking of the arms.
    \item Allows keep a hollow back (a.k.a. ``good gorilla'') to provide a stable support for your partner.
    \item Always keep your head above your ass, otherwise your partner will feel unsafe and slide down to the front.
\end{itemize}

Read more about this in the chapter ``\nameref{sec:safety}''.

\subsection{Spirals}\label{subsec:spirals}

We use a lot of spirals in CI and how they are perceived, seen and anticipated in different movements in other or one's own body.
Moving in spiral patterns is the perfect way to keep the pathway continuation which adds to a more enjoyable, fluid sensation during the dance.
Spiraling is considered a core movement pattern in CI and there is much more to say and experience about it.

Spirals can be very visibly be done between two body parts by moving from the distal parts of the body towards the more proximal parts.
There is a lot possible with playing with the axis, changing the axis, going within or outside the body, \ldots the limit is the imagination (and skill).
And ultimately spirals can also only by visualized, imagined, with pure intention/attention without any external visibility of movement.

\textbf{Exercise:} Connect two points on the body, and visual and move along a connecting line between those by using spirals around that line.
Feel free to also play with different planes.

\subsection{Negative Space}\label{subsec:negative-space}

Dancing in the so-called \gls{negativespace} basically means entering, usually by reaching into it with a limb, the \gls{kinesphere} of another person without, yet engaging in touch.
This is a common way to non-verbally invite someone for contact, to dance together, at a jam for example.

\subsection{Others}\label{subsec:others}

\begin{itemize}
    \item \textbf{Body-surfing}: One person rolls on the floor, the other (usually) perpendicularly rolling over.
    \item \textbf{Counter-balance}: As also done in partner acrobatics, where the center is, for once, oppositional, instead of centers being shared.
\end{itemize}